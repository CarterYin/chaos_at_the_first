% 若编译失败,且生成 .synctex(busy) 辅助文件,可能有两个原因:
% 1. 需要插入的图片不存在:Ctrl + F 搜索 'figure' 将这些代码注释/删除掉即可
% 2. 路径/文件名含中文或空格:更改路径/文件名即可

% ------------------------------------------------------------- %
% >> ------------------ 文章宏包及相关设置 ------------------ << %
% 设定文章类型与编码格式
\documentclass[UTF8]{report}		

% 本文特殊宏包
    \usepackage{siunitx} % 埃米单位

% 本文的特殊宏定义
\def\Im{\mathrm{\,Im\,}}
\def\Re{\mathrm{\,Re\,}}
\def\Ln{\mathrm{\,Ln\,}}
\def\Arg{\mathrm{\,Arg\,}}
\def\Arccos{\mathrm{\,Arccos\,}}
\def\Arcsin{\mathrm{\,Arcsin\,}}
\def\Arctan{\mathrm{\,Arctan\,}}

% 通用宏定义
\def\N{\mathbb{N}}
\def\F{\mathbb{F}}
\def\Z{\mathbb{Z}}
\def\Q{\mathbb{Q}}
\def\R{\mathbb{R}}
\def\C{\mathbb{C}}
\def\T{\mathbb{T}}
\def\S{\mathbb{S}}
\def\A{\mathbb{A}}
\def\I{\mathscr{I}}
\def\d{\mathrm{d}}
\def\p{\partial}


% 导入基本宏包
    \usepackage[UTF8]{ctex}     % 设置文档为中文语言
    \usepackage[colorlinks, linkcolor=blue, anchorcolor=blue, citecolor=blue, urlcolor=blue]{hyperref}  % 宏包:自动生成超链接 (此宏包与标题中的数学环境冲突)
    % \usepackage{docmute}    % 宏包:子文件导入时自动去除导言区,用于主/子文件的写作方式,\include{./51单片机笔记}即可。注:启用此宏包会导致.tex文件capacity受限。
    \usepackage{amsmath}    % 宏包:数学公式
    \usepackage{mathrsfs}   % 宏包:提供更多数学符号
    \usepackage{amssymb}    % 宏包:提供更多数学符号
    \usepackage{pifont}     % 宏包:提供了特殊符号和字体
    \usepackage{extarrows}  % 宏包:更多箭头符号
    \usepackage{multicol}   % 宏包:支持多栏 
    \usepackage{graphicx}   % 宏包:插入图片
    \usepackage{float}      % 宏包:设置图片浮动位置
    %\usepackage{article}    % 宏包:使文本排版更加优美
    \usepackage{tikz}       % 宏包:绘图工具
    %\usepackage{pgfplots}   % 宏包:绘图工具
    \usepackage{enumerate}  % 宏包:列表环境设置
    \usepackage{enumitem}   % 宏包:列表环境设置
    \usepackage{wrapfig}    % 宏包:图文绕排

% 文章页面margin设置
    \usepackage[a4paper]{geometry}
        \geometry{top=1in}  % 1 inch= 2.46 cm, 0.75 inch = 1.85 cm
        \geometry{bottom=1in}
        \geometry{left=0.75in}
        \geometry{right=0.75in}   % 设置上下左右页边距
        \geometry{marginparwidth=1.75cm}    % 设置边注距离(注释、标记等)

% 配置数学环境
    \usepackage{amsthm} % 宏包:数学环境配置
    % theorem-line 环境自定义
        \newtheoremstyle{MyLineTheoremStyle}% <name>
            {11pt}% <space above>
            {11pt}% <space below>
            {}% <body font> 使用默认正文字体
            {}% <indent amount>
            {\bfseries}% <theorem head font> 设置标题项为加粗
            {:}% <punctuation after theorem head>
            {.5em}% <space after theorem head>
            {\textbf{#1}\thmnumber{#2}\ \ (\,\textbf{#3}\,)}% 设置标题内容顺序
        \theoremstyle{MyLineTheoremStyle} % 应用自定义的定理样式
        \newtheorem{LineTheorem}{Theorem.\,}
    % theorem-block 环境自定义
        \newtheoremstyle{MyBlockTheoremStyle}% <name>
            {11pt}% <space above>
            {11pt}% <space below>
            {}% <body font> 使用默认正文字体
            {}% <indent amount>
            {\bfseries}% <theorem head font> 设置标题项为加粗
            {:\\ \indent}% <punctuation after theorem head>
            {.5em}% <space after theorem head>
            {\textbf{#1}\thmnumber{#2}\ \ (\,\textbf{#3}\,)}% 设置标题内容顺序
        \theoremstyle{MyBlockTheoremStyle} % 应用自定义的定理样式
        \newtheorem{BlockTheorem}[LineTheorem]{Theorem.\,} % 使用 LineTheorem 的计数器
    % definition 环境自定义
        \newtheoremstyle{MySubsubsectionStyle}% <name>
            {11pt}% <space above>
            {11pt}% <space below>
            {}% <body font> 使用默认正文字体
            {}% <indent amount>
            {\bfseries}% <theorem head font> 设置标题项为加粗
            { \indent}% <punctuation after theorem head>
            {0pt}% <space after theorem head>
            {\textbf{#3}}% 设置标题内容顺序
        \theoremstyle{MySubsubsectionStyle} % 应用自定义的定理样式
        \newtheorem{definition}{}

%宏包:有色文本框(proof环境)及其设置
    \usepackage[dvipsnames,svgnames]{xcolor}    %设置插入的文本框颜色
    \usepackage[strict]{changepage}     % 提供一个 adjustwidth 环境
    \usepackage{framed}     % 实现方框效果
        \definecolor{graybox_color}{rgb}{0.95,0.95,0.96} % 文本框颜色。修改此行中的 rgb 数值即可改变方框纹颜色,具体颜色的rgb数值可以在网站https://colordrop.io/ 中获得。(截止目前的尝试还没有成功过,感觉单位不一样)(找到喜欢的颜色,点击下方的小眼睛,找到rgb值,复制修改即可)
        \newenvironment{graybox}{%
        \def\FrameCommand{%
        \hspace{1pt}%
        {\color{gray}\small \vrule width 2pt}%
        {\color{graybox_color}\vrule width 4pt}%
        \colorbox{graybox_color}%
        }%
        \MakeFramed{\advance\hsize-\width\FrameRestore}%
        \noindent\hspace{-4.55pt}% disable indenting first paragraph
        \begin{adjustwidth}{}{7pt}%
        \vspace{2pt}\vspace{2pt}%
        }
        {%
        \vspace{2pt}\end{adjustwidth}\endMakeFramed%
        }

% 外源代码插入设置
    % matlab 代码插入设置
    %\usepackage{matlab-prettifier}
    %    \lstset{
    %        style=Matlab-editor,  % 继承matlab代码颜色等
    %    }
    %\usepackage[most]{tcolorbox} % 引入tcolorbox包 
    %\usepackage{listings} % 引入listings包
    %    \tcbuselibrary{listings, skins, breakable}
    %    \newfontfamily\codefont{Consolas} % 定义需要的 codefont 字体
    %    \lstdefinestyle{matlabstyle}{
    %        language=Matlab,
    %        basicstyle=\small\ttfamily\codefont,    % ttfamily 确保等宽 
    %        breakatwhitespace=false,
    %        breaklines=true,
    %        captionpos=b,
    %        keepspaces=true,
    %        numbers=left,
    %        numbersep=15pt,
    %        showspaces=false,
    %        showstringspaces=false,
    %        showtabs=false,
    %        tabsize=2
    %    }
    %    \newtcblisting{matlablisting}{
    %        arc=2pt,        % 圆角半径
    %        top=-5pt,
    %        bottom=-5pt,
    %        left=1mm,
    %        listing only,
    %        listing style=matlabstyle,
    %        breakable,
    %        colback=white   % 选一个合适的颜色
    %    }
% table 支持
    \usepackage{booktabs}   % 宏包:三线表
    \usepackage{tabularray} % 宏包:表格排版
    \usepackage{longtable}  % 宏包:长表格


%figure 设置
%    \usepackage{graphicx}  % 支持 jpg, png, eps, pdf 图片 
%    \usepackage{svg}       % 支持 svg 图片
%        \svgsetup{
%             指向 inkscape.exe 的路径
%            inkscapeexe = C:/aa_MySame/inkscape/bin/inkscape.exe, 
%            inkscapeexe = C:/aa_MySame/inkscape/bin/inkscape.exe, 
%             一定程度上修复导入后图片文字溢出几何图形的问题
%            inkscapelatex = false                 
%        }
%    \usepackage{subcaption} % subfigure 子图支持

%图表进阶设置
%    \usepackage{caption}    % 图注、表注
%        \captionsetup[figure]{name=图}  
%        \captionsetup[table]{name=表}
%        \captionsetup{labelfont=bf, font=small}
%    \usepackage{float}     % 图表位置浮动设置 

% 圆圈序号自定义
    \newcommand*\circled[1]{\tikz[baseline=(char.base)]{\node[shape=circle,draw,inner sep=0.8pt, line width = 0.03em] (char) {\small \bfseries #1};}}   % TikZ solution

% 列表设置
%    \usepackage{enumitem}   % 宏包:列表环境设置
%        \setlist[enumerate]{itemsep=0pt, parsep=0pt, topsep=0pt, partopsep=0pt, leftmargin=3.5em} 
%        \setlist[itemize]{itemsep=0pt, parsep=0pt, topsep=0pt, partopsep=0pt, leftmargin=3.5em}
%        \newlist{circledenum}{enumerate}{1} % 创建一个新的枚举环境  
%        \setlist[circledenum,1]{  
%            label=\protect\circled{\arabic*}, % 使用 \arabic* 来获取当前枚举计数器的值,并用 \circled 包装它  
%            ref=\arabic*, % 如果需要引用列表项,这将决定引用格式(这里仍然使用数字)
%            itemsep=0pt, parsep=0pt, topsep=0pt, partopsep=0pt, leftmargin=3.5em
%        }  

% 其它设置
    % 脚注设置
        \renewcommand\thefootnote{\ding{\numexpr171+\value{footnote}}}
    % 参考文献引用设置
        \bibliographystyle{unsrt}   % 设置参考文献引用格式为unsrt
        \newcommand{\upcite}[1]{\textsuperscript{\cite{#1}}}     % 自定义上角标式引用
    % 文章序言设置
        \newcommand{\cnabstractname}{序言}
        \newenvironment{cnabstract}{%
            \par\Large
            \noindent\mbox{}\hfill{\bfseries \cnabstractname}\hfill\mbox{}\par
            \vskip 2.5ex
            }{\par\vskip 2.5ex}

% 文章默认字体设置
    \usepackage{fontspec}   % 宏包:字体设置
        \setmainfont{SimSun}    % 设置中文字体为宋体字体
        \setCJKmainfont[AutoFakeBold=3]{SimSun} % 设置加粗字体为 SimSun 族,AutoFakeBold 可以调整字体粗细
        \setmainfont{Times New Roman} % 设置英文字体为Times New Roman

% 各级标题自定义设置
    \usepackage{titlesec}   
        \titleformat{\chapter}[hang]{\normalfont\huge\bfseries\centering}{第\,\thechapter\,章}{20pt}{}
        \titlespacing*{\chapter}{0pt}{-20pt}{20pt} % 控制上方空白的大小
        % section标题自定义设置 
        \titleformat{\section}[hang]{\normalfont\Large\bfseries}{§\,\thesection\,}{8pt}{}
        % subsubsection标题自定义设置
        %\titleformat{\subsubsection}[hang]{\normalfont\bfseries}{}{8pt}{}

% >> ------------------ 文章宏包及相关设置 ------------------ << %
% ------------------------------------------------------------- %

% ----------------------------------------------------------- %
% >> --------------------- 文章信息区 --------------------- << %
% 页眉页脚设置
    \usepackage{fancyhdr}   %宏包:页眉页脚设置
        \pagestyle{fancy}
        \fancyhf{}
        \cfoot{\thepage}
        \renewcommand\headrulewidth{1pt}
        \renewcommand\footrulewidth{0pt}
        \lhead{2024.8 -- 2025.1} 
        \chead{yinchao050313@gmail.com}    
        \rhead{yinchao23@mails.ucas.ac.cn}
%文档信息设置
    \title{信号与系统}
    \author{尹超\\ \footnotesize 中国科学院大学,北京 100049\\ Carter Yin \\ \footnotesize University of Chinese Academy of Sciences, Beijing 100049, China}
    \date{\footnotesize 2024.8 -- 2025.1}
% >> --------------------- 文章信息区 --------------------- << %
% ----------------------------------------------------------- %

% 开始编辑文章

\begin{document} 
\zihao{5}             % 设置全文字号大小, -4 为小四, 5 为五号

% --------------------------------------------------------------- %
% >> --------------------- 封面序言与目录 --------------------- << %
% 封面
    \maketitle\newpage  
    \pagenumbering{Roman} % 页码为大写罗马数字
    \thispagestyle{fancy}   % 显示页码、页眉等

% 序言
    \begin{cnabstract}\normalsize 
        本笔记为信号与系统的笔记整理\par
        讲课教师:刘智勇(国科大人工智能学院
        中国科学院自动化研究所)\par
        助教:\par
        张思琦:自动化所直博生第五年\par
        林镇阳:人工智能学院直博生第四年\par
        庄新宇:自动化所直博生第二年\par
\end{cnabstract}    
\addcontentsline{toc}{chapter}{序言} % 手动添加为目录

% 目录
    \setcounter{tocdepth}{4}                % 目录深度(为1时显示到section)
    \tableofcontents                        % 目录页
    \addcontentsline{toc}{chapter}{目录}    % 手动添加此页为目录
    \thispagestyle{fancy}                   % 显示页码、页眉等 

% 收尾工作
    \newpage    
    \pagenumbering{arabic} 


    
% >> --------------------- 封面序言与目录 --------------------- << %
% --------------------------------------------------------------- %

\chapter{信号与系统}

\section{连续时间和离散时间信号}

\begin{definition}
    几个概念
    \begin{itemize}
        \item 模拟信号:时间和函数值均为连续的信号
        \item 抽样/采样信号:时间是离散的,函数值是连续的
        \item 数字信号:时间和函数值均为离散的
    \end{itemize}

    信号的能量与功率
    对于一个连续时间信号 $x(t)$,在时间间隔 $t_1 \leq t \leq t_2$ 内的总能量定义为
    \[
    E = \int_{t_1}^{t_2} |x(t)|^2 \, dt
    \]
    $x$ 为 $x$(可能为复数)的模,其平均功率为
    \[
    P = \frac{1}{t_2 - t_1} \int_{t_1}^{t_2} |x(t)|^2 \, dt
    \]

    对于一个离散时间信号 $x[n]$,在时间间隔 $n_1 \leq n \leq n_2$ 内的总能量定义为
    \[
    E = \sum_{n=n_1}^{n_2} |x[n]|^2
    \]
    平均功率为
    \[
    P = \frac{1}{n_2 - n_1 + 1} \sum_{n=n_1}^{n_2} |x[n]|^2
    \]

    在无穷区间($-\infty < t < +\infty$ 或 $-\infty < n < +\infty$)连续和离散时间信号 $x(t)$、$x[n]$ 总能量为
    \[
    E_\infty \triangleq \lim_{T \to \infty} \int_{-T}^{T} |x(t)|^2 \, dt = \int_{-\infty}^{+\infty} |x(t)|^2 \, dt
    \]
    \[
    E_\infty \triangleq \lim_{N \to \infty} \sum_{n=-N}^{N} |x[n]|^2 = \sum_{n=-\infty}^{+\infty} |x[n]|^2
    \]

    在无穷区间($-\infty < t < +\infty$ 或 $-\infty < n < +\infty$)连续和离散时间信号 $x(t)$、$x[n]$ 的平均功率为
    \[
    P_\infty \triangleq \lim_{T \to \infty} \frac{1}{2T} \int_{-T}^{T} |x(t)|^2 \, dt
    \]
    \[
    P_\infty \triangleq \lim_{N \to \infty} \frac{1}{2N + 1} \sum_{n=-N}^{N} |x[n]|^2
    \]
\end{definition}

\begin{definition}
    能量信号与功率信号
    \begin{itemize}
        \item 能量信号:能量有限的信号为能量信号,即 $E_\infty < \infty$
        \item 功率信号:功率有限的信号为功率信号,即 $P_\infty < \infty$
        \item 非能量非功率信号:信号的能量和功率均无界
    \end{itemize}
    \begin{itemize}
        \item 能量信号的(无限区间的平均)功率为零
        \item 一般周期信号为功率信号
        \item 时限信号(即仅在有限时间段内不为零的非周期信号)为能量信号
    \end{itemize}
\end{definition}

\section{信号的自变量变换}

\begin{definition}
    \textbf{\textcolor{red}{信号的时间平移(时移)}}\par
    \vspace{1em}
    将自变量由 $t \rightarrow t - t_0$ 或 $n \rightarrow n - n_0$,即由 $x(t) \rightarrow x(t - t_0)$ 或 $x[n] \rightarrow x[n - n_0]$
    \begin{itemize}
        \item 当 $t_0$ 或 $n_0$ 大于 0,表示信号右移 $t_0$ 或 $n_0$ 个单位,也即延时
        \item 当 $t_0$ 或 $n_0$ 小于 0,表示信号左移 $t_0$ 或 $n_0$ 个单位,也即超前
    \end{itemize}

    \textbf{\textcolor{red}{信号的时间反转(反折、反褶)}}\par
    \vspace{1em}
    将自变量由 $t \rightarrow -t$ 或 $n \rightarrow -n$,即由 $x(t) \rightarrow x(-t)$ 或 $x[n] \rightarrow x[-n]$
    \begin{itemize}
        \item 将信号沿 $t = 0$ 或 $n = 0$ 轴反转得到
        \item 没有物理器件可实现此功能,但在数字信号处理中可以,如堆栈的“先进后出”
    \end{itemize}

    \textbf{\textcolor{red}{信号的时间尺度变换(展缩)}}\par
    \vspace{1em}
    将自变量由 $t \rightarrow at$ 或 $n \rightarrow an$,即由 $x(t) \rightarrow x(at)$ 或 $x[n] \rightarrow x[an]$
    \begin{itemize}
        \item 连续:当 $0 < a < 1$,信号沿着横坐标扩展;当 $a > 1$,信号沿着横坐标压缩
        \item 离散:$an$ 为整数才有意义;当 $0 < a < 1$,内插;当 $a > 1$,抽取
        \item 物理意义:以磁带播放为例,$0 < a < 1$ 慢放;$a > 1$ 快放;$a = -1$ 倒放
    \end{itemize}

    \textbf{\textcolor{red}{信号的混合变换}}\par
    \vspace{1em}
    将自变量由 $t \rightarrow \alpha t + \beta$,即由 $x(t) \rightarrow x(\alpha t + \beta)$
    \begin{itemize}
        \item 混合了平移、反转、展缩等变换;运算时次序可以任意,但要注意一切变换都是对 $t$ 开展的
        \item 信号将保持 $x(t)$ 的形状
        \item 操作中可以遵循“平移 – 展缩 – 反转 – 验证”的步骤
    \end{itemize}

    \textbf{\textcolor{red}{周期信号}}\par
    \vspace{1em}
    对于任意的 $t$ 或 $n$,如果有 $x(t) = x(t + T)$ 或 $x[n] = x[n + N]$,其中 $T$ 为某正值,$N$ 为某正整数,则称 $x(t)$ 和 $x[n]$ 为周期信号,周期为 $T$ 或 $N$
    \begin{itemize}
        \item 任意的 $mT$ 或 $mN$ 也是上述信号的周期,$m$ 为任意整数
        \item 满足上述定义的最小的 $T$ 或 $N$ 称为信号的基波周期(fundamental period)
        \item $x(t) = c$ 常数信号也是周期信号,但没有基波周期,因为任意小的 $T$ 都成立;$x[n] = c$ 的基波周期为 1
        \item 如果两个连续时间的周期信号 $x_1(t)$ 和 $x_2(t)$ 的基波周期 $T_1, T_2$ 之比 $T_1/T_2$ 是有理数,那么 $x(t) = x_1(t) + x_2(t)$ 为周期信号,其基波周期为 $T_1, T_2$ 的最小公倍数
        \item 两个离散时间的周期序列 $x_1[n]$ 和 $x_2[n]$ 之和一定是周期序列
    \end{itemize}

    \textbf{\textcolor{red}{偶信号与奇信号}}\par
\vspace{1em}
    考虑实信号
    \begin{itemize}
        \item 如果有 $x(t) = x(-t)$ 或者 $x[n] = x[-n]$,则为偶信号
        \item 如果有 $-x(t) = x(-t)$ 或者 $-x[n] = x[-n]$,则为奇信号
        \item 奇信号必过零点(如果在 $t = 0$ 或 $n = 0$ 有定义),即 $x(0) = 0$ 或 $x[0] = 0$
        \item 一个实信号可以分解为偶信号与奇信号之和
        \[
        x(t) = \text{Even}(x(t)) + \text{Odd}(x(t))
        \]
        \[
        \text{Even}(x(t)) = \frac{1}{2} [x(t) + x(-t)]
        \]
        \[
        \text{Odd}(x(t)) = \frac{1}{2} [x(t) - x(-t)]
        \]
    \end{itemize}

    共轭偶对称和共轭奇对称信号
考虑复信号
\begin{itemize}
    \item 如果有 $x(t) = x^*(-t)$ 则为共轭偶对称信号,也称为共轭对称
    \begin{itemize}
        \item 实部是偶函数、虚部为奇函数:设 $x(t) = a(t) + jb(t)$,可直接证明
        \item 模为偶函数、相位为奇函数:相位 $\nless x(-t) = \text{atan} \left( \frac{b(-t)}{a(-t)} \right) = \text{atan} \left( \frac{-b(t)}{a(t)} \right) = -\text{atan} \left( \frac{b(t)}{a(t)} \right) = -\nless x(t)$
    \end{itemize}
    \item 如果有 $x(t) = -x^*(-t)$ 则为共轭奇对称信号
    \begin{itemize}
        \item 实部是奇函数、虚部是偶函数
        \item 模为偶函数、相位为奇函数
    \end{itemize}
\end{itemize}
\end{definition}

\section{常见的基本信号}

\begin{definition}
    指数信号
    \begin{itemize}
        \item 正(余)弦信号
        \[
        x(t) = A \cos(\omega_0 t + \phi)
        \]
        \begin{itemize}
            \item $A$:振幅
            \item $\omega_0$:角频率(弧度/秒,rad/s)
            \item $\phi$:初始相位
            \item (基波)周期为 $T = \frac{2\pi}{\omega_0}$,频率为 $f_0 = \frac{1}{T} = \frac{\omega_0}{2\pi}$,即 $\omega_0 = 2\pi f_0$
            \item 余弦信号也称为角频率为 $\omega_0$ 的单频信号
            \item 角频率 $\omega_0$ 是一个非常重要的概念,反映了信号的变换快慢
        \end{itemize}

        \item 连续时间复指数信号
        \[
        x(t) = Ce^{at}
        \]
        \begin{itemize}
            \item $C, a$ 一般为复常数
            \item 实指数信号:$C, a$ 为实常数
            \begin{itemize}
                \item 当 $a > 0$,呈现指数增长的形式
                \item 当 $a < 0$,呈现指数衰减的形式(很多系统的自然响应为指数衰减的形式)
                \item 当 $a = 0$,信号为常数
            \end{itemize}
        \end{itemize}

        \item 周期复指数和正弦信号:当 $C = 1, a = j\omega_0$ 为纯虚数
        \[
        x(t) = e^{j\omega_0 t} = \cos(\omega_0 t) + j \sin(\omega_0 t) \quad \text{(欧拉公式)}
        \]
        \begin{itemize}
            \item 显然,$x(t)$ 是周期信号,$\omega_0$ 为角频率(rad/s),也称为基波频率,基波周期为 $T_0 = \frac{2\pi}{|\omega_0|}$(s),频率为 $f_0 = \frac{1}{T_0} = \frac{|\omega_0|}{2\pi}$(Hz)
            \item $x(t)$ 是功率信号,其平均功率为
            \[
            P_\infty = \lim_{T \to \infty} \frac{1}{2T} \int_{-T}^{T} |e^{j\omega_0 t}|^2 dt = \lim_{T \to \infty} \frac{1}{2T} \int_{-T}^{T} 1 dt = 1
            \]
            \item 正弦信号可用周期复指数信号表示
            \[
            x(t) = A \cos(\omega_0 t + \phi) = \frac{A}{2} e^{j(\omega_0 t + \phi)} + \frac{A}{2} e^{-j(\omega_0 t + \phi)}
            \]
            出现负频率
        \end{itemize}

        \item 成谐波关系的复指数信号集合:$\phi_k(t)$,$\phi_k(t) = e^{jk\omega_0 t}$,$k = 0, \pm1, \pm2, \ldots$
        \begin{itemize}
            \item 集合中每个信号都是周期的,第 $k$ 个信号的(除 $k = 0$ 外)基波频率为 $k\omega_0$,是 $\omega_0$ 的整数倍,因此称信号集合为成谐波关系
            \item 第 $k$ 个信号的基波周期为 $\frac{2\pi}{|k\omega_0|}$,信号的公共周期是 $T_0 = \frac{2\pi}{|\omega_0|}$
            \item 利用 $\left\{\phi_k(t)\right\}$ 作为基本构造单元,可以构成几乎任意周期信号(傅里叶级数)
        \end{itemize}

        \item 一般复指数信号
        \[
        C, a \text{ 分别写成 } C = |C|e^{j\theta}, a = r + j\omega_0
        \]
        \[
        x(t) = Ce^{at} = |C| e^{j\theta} e^{(r + j\omega_0)t} = |C| e^{rt} e^{j(\omega_0 t + \theta)} = |C| e^{rt} \cos(\omega_0 t + \theta) + j |C| e^{rt} \sin(\omega_0 t + \theta)
        \]
        \begin{itemize}
            \item $|C| e^{rt}$ 提供包络线,显示变化趋势
            \item 具有衰减($r < 0$)指数信号常称为阻尼正弦振荡,例如汽车减震系统:由于摩擦、电阻等消耗能量
        \end{itemize}
    \end{itemize}
\end{definition}

\begin{definition}
    离散时间复指数信号
    \[
    x[n] = C\alpha^n
    \]
    其中 $C, \alpha$ 一般为复常数;也可以表述为 $x[n] = Ce^{\beta n}$,即 $\alpha = e^{\beta}$;一般前者更为方便
    \begin{itemize}
        \item 实指数信号(序列):$C, \alpha$ 均为实数,则如果
        \begin{itemize}
            \item $|\alpha| > 1$:信号指数增长
            \item $|\alpha| < 1$:信号指数衰减
            \item $\alpha < 0$:信号符号交替变化
        \end{itemize}
        \item 离散时间正弦信号(序列)
        \item $\beta$ 为纯虚数,也即:\par
        $\alpha = e^{j\omega_0}$,设 $C = 1$
        \[
        x[n] = e^{j\omega_0 n}
        \]
        与正弦信号 $x[n] = A \cos(\omega_0 n + \phi)$ 密切相关;$n$ 无量纲,$\omega_0$(弧度、数字角频率)、$\phi$ 量纲为弧度

        \item 一般复指数(序列):$C, \alpha$ 均为复数,表示为 $C = |C|e^{j\theta}$,$\alpha = |\alpha|e^{j\omega_0}$
        \[
        x[n] = C\alpha^n = |C| |\alpha|^n e^{j(\omega_0 n + \theta)} = |C| |\alpha|^n \cos(\omega_0 n + \theta) + j |C| |\alpha|^n \sin(\omega_0 n + \theta)
        \]

        \item 周期性质($C, \alpha$ 一般为复常数和连续时间复指数信号完全不同)
        \begin{itemize}
            \item 连续时间复指数信号 $x(t) = e^{j\omega_0 t}$ 是以 $T_0 = \frac{2\pi}{|\omega_0|}$(s)为基波周期的信号,$\omega_0$ 越大周期越短,振荡越快
            \item 在频率上,显然有 $e^{j\omega_0 n} = e^{j\omega_0 n} e^{j2k\pi n} = e^{j(\omega_0 + 2k\pi)n}$,$k = 0, \pm1, \pm2, \ldots$,因此频率是以 $2\pi$ 为周期的,低频部分在 $0$ 和 $2\pi$ 附近,高频部分在 $\pi$ 附近;只需在任意的一个 $2\pi$ 间隔内考虑 $\omega_0$,一般取 $[0, 2\pi)$ 或者 $[-\pi, \pi)$
            \item 在(时间)周期上,假设周期为 $N$,那么有
            \[
            e^{j\omega_0 n} = e^{j\omega_0 (n+N)} = e^{j\omega_0 n} e^{j\omega_0 N} \Rightarrow e^{j\omega_0 N} = 1 \Rightarrow \omega_0 N = 2\pi m \Rightarrow \frac{\omega_0}{2\pi} = \frac{m}{N}
            \]
            当 $\frac{\omega_0}{2\pi}$ 为有理数,$e^{j\omega_0 n}$ 才是周期信号,其基波频率为(假设 $N, m$ 无公因子):
            \[
            \frac{2\pi}{N} = \frac{\omega_0}{m} \quad \text{(不是 $\omega_0$)}
            \]
            基波周期为:$N = \frac{2\pi m}{|\omega_0|}$

            例:$x[n] = \cos\left(\frac{8\pi n}{31}\right)$,$\omega_0 = \frac{8\pi}{31}$,$m = 4$,$N = 31$;基波频率为:$\frac{2\pi}{31}$,样本每隔 31 个点才重复
        \end{itemize}

        \item 成谐波关系 DT 复指数信号集合:$\left\{\phi_k[n]\right\}$,$\phi_k[n] = e^{jk(2\pi/N)n}$,$k = 0, \pm1, \pm2, \ldots$
        \begin{itemize}
            \item 集合中每个信号都是周期的,公共周期是 $N$,称信号集合为成谐波关系
            \item 第 $k$ 个信号的(除 $k = 0$ 外)的基波频率是多少?(思考)
            \item $\phi_k[n] = \phi_{k+N}[n]$,集合中仅有 $N$ 个互不相同的周期复指数信号
            \item 利用成谐波关系的信号作为基本构造单元,可以构成几乎任意的周期信号(傅里叶级数)
        \end{itemize}

        \item 冲激和阶跃
        \begin{itemize}
            \item 离散时间单位脉冲序列和单位阶跃序列
            \begin{itemize}
                \item 单位脉冲
                \[
                \delta[n] = \begin{cases} 
                0 & n \neq 0 \\
                1 & n = 0 
                \end{cases}
                \]
                \item 单位阶跃
                \[
                u[n] = \begin{cases} 
                0 & n < 0 \\
                1 & n \geq 0 
                \end{cases}
                \]
            \end{itemize}
        \end{itemize}
    \end{itemize}
\end{definition}

\begin{definition}
    离散时间单位脉冲序列和单位阶跃序列
    \begin{itemize}
        \item 单位脉冲是单位阶跃的一次差分:
        \[
        \delta[n] = u[n] - u[n - 1]
        \]
        \item 单位阶跃是单位脉冲的求和:
        \[
        u[n] = \sum_{m=-\infty}^{n} \delta[m] = \sum_{k=0}^{+\infty} \delta[n - k] = \sum_{k=0}^{+\infty} u[k]\delta[n - k] = \sum_{k=-\infty}^{+\infty} u[k]\delta[n - k]
        \]
        (延迟脉冲的叠加)
    \end{itemize}

    单位脉冲序列的采样性质,用于一个信号在 $n = 0$ 时值的采样
    \[
    x[n] \delta[n] = x[0] \delta[n]
    \]
    更一般地,用于 $n = n_0$ 时值的采样
    \[
    x[n] \delta[n - n_0] = x[n_0] \delta[n - n_0]
    \]

    连续时间单位冲激函数和单位阶跃函数
    \begin{itemize}
        \item 连续时间单位阶跃函数 $u(t)$ 定义为
        \[
        u(t) = \begin{cases} 
        0 & t < 0 \\
        1 & t > 0 
        \end{cases}
        \]
        类似地可以定义为连续时间单位冲激函数 $\delta(\tau)$ 的积分
        \[
        u(t) = \int_{-\infty}^{t} \delta(\tau) d\tau = \int_{0}^{\infty} \delta(t - \sigma) d\sigma = \int_{-\infty}^{\infty} u(\sigma)\delta(t - \sigma) d\sigma
        \]
        反过来,$\delta(t)$ 可以定义为 $u(t)$ 的一次导数:
        \[
        \delta(t) = \frac{d u(t)}{d t}
        \]

        \item 考虑一个连续时间函数 $u_\Delta(t)$ 定义为:
        \[
        u_\Delta(t) = \begin{cases} 
        0 & t \leq 0 \\
        t/\Delta & 0 < t \leq \Delta \\
        1 & t > \Delta 
        \end{cases}
        \]
        $u_\Delta(t)$ 的导数为:
        \[
        \delta_\Delta(t) = \frac{d u_\Delta(t)}{d t} = \begin{cases} 
        0 & t \leq 0 \\
        1/\Delta & 0 < t \leq \Delta \\
        0 & t > \Delta 
        \end{cases}
        \]
        显然当 $0 < t \leq \Delta$,$\delta_\Delta(t) = 1$。因此,$\delta(t)$ 可以定义为
        \[
        \delta(t) = \lim_{\Delta \to 0} \delta_\Delta(t)
        \]
        $\delta(t)$ 的持续时间为 0,但面积为 1
    \end{itemize}

    冲激函数的采样性质:
    \[
    x(t) \delta_\Delta(t) \approx x(0) \delta_\Delta(t)
    \]
    \[
    x(t) \delta(t) = x(0) \delta(t)
    \]
    \[
    x(t) \delta(t - t_0) = x(t_0) \delta(t - t_0) \neq x(t_0)
    \]

    冲激函数的采样性质:
    例:求信号 $x(t) = e^{-2t}u(t)$ 的导数
    \[
    \frac{d x(t)}{d t} = \frac{d}{d t} \left( e^{-2t}u(t) \right) = e^{-2t} \delta(t) - 2e^{-2t} u(t) = \delta(t) - 2e^{-2t} u(t)
    \]
    冲激函数为偶函数
    \[
    \delta(t) = \delta(-t)
    \]
\end{definition}

\section{系统及基本性质}

\begin{definition}
    \begin{itemize}
        \item 连续时间系统:系统的输入和输出信号皆为连续时间信号,$x(t) \rightarrow y(t)$
        统一成一阶常系数线性微分方程的形式:
        \[
        \frac{d y(t)}{d t} + a y(t) = b x(t)
        \]

        \item 离散时间系统:系统的输入和输出信号皆为离散时间信号,$x[n] \rightarrow y[n]$
        统一成一阶常系数线性差分方程的形式:
        \[
        y[n] - a y[n - 1] = b x[n]
        \]

        \item 有记忆系统与无记忆系统
        一个系统的输出如果只取决于系统的当前输入,称为无记忆系统(又称即时系统)
        例如:
        \[
        y(t) = R x(t) \quad \text{(电阻电路)}
        \]
        \[
        y(t) = x(t)
        \]
        \[
        y[n] = x[n] \quad \text{(恒等系统)}
        \]
        否则称为有记忆系统(动态系统)
        例如:
        \[
        y(t) = \frac{1}{C} \int_{-\infty}^{t} x(\tau) d\tau \quad \text{(电容)}
        \]
        \[
        y[n] = \sum_{k=-\infty}^{n} x[k] = \sum_{k=-\infty}^{n-1} x[k] + x[n] \Rightarrow y[n] - y[n-1] = x[n] \quad \text{(累加器)}
        \]
        \[
        y[n] = x[n-1] \quad \text{(延时器)}
        \]
        \[
        y[n] = x[n] - x[n-1] \quad \text{(差分器)}
        \]

        \item 可逆性与可逆系统\par
        一个系统如果输入不同则输出不同,那么系统就是可逆的;
        一个系统如果是可逆的,那么存在一个逆系统,级联原系统后的输出就等于原系统的输入
        \begin{itemize}
            \item 乘法器 $y(t) = 2x(t)$ 的逆系统为 $w(t) = 0.5y(t)$
            \item 累加器 $y[n] = \sum_{k=-\infty}^{n} x[k]$ 逆系统为差分器 $w[n] = y[n] - y[n-1]$
            \item $y(t) = x^2(t)$ 不是可逆的;通信中的编码器是一个典型的可逆系统,解码器是其逆系统
        \end{itemize}

        \item 因果性\par
        如果一个系统在任何时刻的输出只取决于现在及过去的输入,就称为因果系统
        \begin{itemize}
            \item 因果系统例子:累加器、(后向)差分器($y[n] - y[n-1]$)、$y(t) = x(t) - \cos(t + 1)$
            \item 非因果系统例子:$y(t) = x(t) - x(t + 1)$、$y[n] = \frac{1}{2M+1} \sum_{k=-M}^{M} x[n - k]$
            \item 一切以时间为自变量的可物理实现的系统都是因果的
            \item 所有的无记忆系统都是因果的
        \end{itemize}

        \item 稳定性\par
        一个系统如果输入信号是有界的,输出信号也是有界的,则称系统是稳定的。
        \begin{itemize}
            \item 稳定系统例子:RC电路、汽车系统(输入为力$f(t)$,输出为速度$v(t)$,平衡点为$V = \frac{F}{\rho}$)、$y(t) = e^{x(t)}$
            \item 不稳定系统例子:$y[n] = \sum_{k=-\infty}^{n} u[k]=(n+1)u[n]$、$y(t) = t x(t)$
            \item 实际系统一般存在能量消耗,如电阻的耗能、摩擦的耗能;想象一下如果汽车的摩擦系数$\rho = 0$?
        \end{itemize}

        \item 时不变性\par
        直观来讲,系统的特性和行为不随时间而改变的系统为时不变系统。
        \begin{itemize}
            \item RC电路中的$R$和$C$不随时间而变化
            \item 汽车系统中的摩擦系数$\rho$和质量$m$不随时间而变化
            \item 数学上:当输入信号有一个时移,在输出信号产生同样的时移,则为时不变系统
            \begin{itemize}
                \item 如果$x(t) \rightarrow y(t)$,则有$x(t - t_0) \rightarrow y(t - t_0)$
                \item 如果$x[n] \rightarrow y[n]$,则有$x[n - n_0] \rightarrow y[n - n_0]$
            \end{itemize}
        \end{itemize}

        判断一个系统是否时不变方法:
        \begin{itemize}
            \item 直观判断方法: 若$y(\cdot)$前出现变系数、或有反转、展缩变换,则系统为时变系统
            \item 一般性步骤:
            \begin{enumerate}
                \item 令输入为$x_1(t) = x(t - t_0)$,得到输出为$y_1(t)$
                \item 如果$y(t - t_0)$与$y_1(t)$相同,则为时不变系统
            \end{enumerate}
        \end{itemize}

        \item 例1:$y(t) = \sin[x(t)]$
    \begin{itemize}
        \item 直观判断为时不变系统
        \item 证明:
        \begin{itemize}
            \item 步骤1:$x_1(t) = x(t - t_0) \Rightarrow y_1(t) = \sin[x(t - t_0)]$
            \item 步骤2:$y(t - t_0) = \sin[x(t - t_0)] = y_1(t)$
        \end{itemize}
        因此为时不变系统
    \end{itemize}

    \item 例2:$y[n] = nx[n]$
        \begin{itemize}
            \item 直观判断不是时不变系统
            \item 证明:
            \begin{itemize}
                \item 步骤1:$x_1[n] = x[n - n_0] \Rightarrow y_1[n] = nx[n - n_0]$
                \item 步骤2:$y[n - n_0] = (n - n_0)x[n - n_0] \neq y_1[n]$
            \end{itemize}
            因此不是时不变系统
        \end{itemize}

        \item 线性\par
        满足可加性和齐次性的系统是线性系统
        \begin{itemize}
            \item 可加性:$x_1 \rightarrow y_1, x_2 \rightarrow y_2$,则有$x_1 + x_2 \rightarrow y_1 + y_2$
            \item 齐次性(比例性):$x \rightarrow y$,则有$ax \rightarrow ay$,$a$为任意复常数
            \item 零输入产生零输出(齐次性)
        \end{itemize}
        两者组合得到线性系统的叠加性

        叠加性:如果系统对信号$x_k(t)$的响应为$y_k(t)$,$k=1,2,3,\ldots$,那么系统对信号$x(t) = \sum_{k} x_k(t)$的响应$y(t)$为
        \[
        y(t) = \sum_{k} y_k(t)
        \]
        离散时间线性系统同上

        判断一个系统“$\rightarrow$”是否线性系统的一般性步骤($a, b, x(t), y(t)$可为复数):
        \begin{enumerate}
            \item $x_1(t) \rightarrow y_1(t)$,$x_2(t) \rightarrow y_2(t)$
            \item 设$x_3(t) = a x_1(t) + b x_2(t) \rightarrow y_3(t)$
            \item 如果$y_3(t) = a y_1(t) + b y_2(t)$,则系统为线性的
        \end{enumerate}
        本质利用“线性”的定义

        \item 例1:证明系统$y(t) = t x(t)$是线性的
        \begin{itemize}
            \item 步骤1:$y_1(t) = t x_1(t)$,$y_2(t) = t x_2(t)$
            \item 步骤2:$x_3(t) = a x_1(t) + b x_2(t) \Rightarrow y_3(t) = t [a x_1(t) + b x_2(t)]$
            \item 步骤3:$a y_1(t) + b y_2(t) = a t x_1(t) + b t x_2(t) = y_3(t)$
            \item 因此系统是线性的
        \end{itemize}

        \item 例2:证明$y(t) = x^2(t)$是非线性的
        \begin{itemize}
            \item 步骤1:$y_1(t) = x_1^2(t)$,$y_2(t) = x_2^2(t)$
            \item 步骤2:$x_3(t) = a x_1(t) + b x_2(t) \Rightarrow y_3(t) = [a x_1(t) + b x_2(t)]^2$
            \item 步骤3:$a y_1(t) + b y_2(t) = a x_1^2(t) + b x_2^2(t) \neq y_3(t)$
            \item 因此系统不是线性的
        \end{itemize}

        \item 例3:判断$y(t) = \Re\{x(t)\}$是否为线性
        \begin{itemize}
            \item 步骤1:$y_1(t) = \Re\{x_1(t)\}$,$y_2(t) = \Re\{x_2(t)\}$
            \item 步骤2:$x_3(t) = j x_1(t) + j x_2(t) \Rightarrow y_3(t) = -\Im\{x_1(t) + x_2(t)\}$
            \item 步骤3:$j \Re\{x_1(t)\} + j \Re\{x_2(t)\} \neq y_3(t)$
            \item 因此系统不是线性的
        \end{itemize}

        \item 例4
        \begin{itemize}
            \item 例4:$y[n] = 2x[n] + 3$
            \item 容易验证系统不是线性的。但此系统输出信号的增量与输入信号的增量满足线性关系,此类系统称为增量线性系统
            \item 当 $y_0 = 0$,系统输出 $y$ 只取决于输入 $x$,称为系统的零状态响应(线性系统)
            \item 当 $x = 0$,系统输出 $y$ 只取决于 $y_0$,称为系统的零输入响应
            \item 完全响应 = 零状态响应 + 零输入响应
        \end{itemize}
    \end{itemize}
\end{definition}


\chapter{LTI系统时域分析}

\section{离散时间LTI系统:卷积和}

\begin{definition}
    线性时不变(LTI: Linear Time Invariant)系统
    \begin{itemize}
        \item 线性:叠加性质
        \item 时不变:激励延迟则响应相应延迟
    \end{itemize}
    \begin{itemize}
        \item 如果一个复杂信号能够分解为简单的基本信号的叠加,那么LTI系统对此信号的响应将可以分解为基本信号的响应的叠加。
        \item 为什么研究LTI系统?
        \begin{itemize}
            \item LTI系统可以表征很多物理过程,基于上述特点,发展了成熟而强大的LTI分析手段
        \end{itemize}
        \item 本章主要从时间域对信号和系统进行分析,因此为时域分析
    \end{itemize}

    用脉冲表示离散时间信号
    \begin{itemize}
        \item 回顾:单位脉冲信号的采样性质:$x[n] \delta[n - n_0] = x[n_0] \delta[n - n_0]$
        \item 离散时间信号可以表示为
        \[
        x[n] = \cdots + x[-1] \delta[n + 1] + x[0] \delta[n] + x[1] \delta[n - 1] + \cdots = \sum_{k=-\infty}^{\infty} x[k] \delta[n - k]
        \]
        \item 任意一个离散时间信号可以表示为时移单位脉冲序列的线性加权和
        \begin{itemize}
            \item 时移单位脉冲序列作为基本简单信号具有筛选性质
            \item 离散LTI系统时域分析的基础
        \end{itemize}
    \end{itemize}

    离散时间LTI系统的单位脉冲响应及卷积和表示
    \begin{itemize}
        \item 基本信号集合为单位脉冲及其时移信号的集合 $\left\{\delta[n - k] \right\}, k \in \mathbb{Z}$ ($\mathbb{Z}$ 表示整数集)
        \begin{itemize}
            \item 可以描述任意离散时间信号
            \item LTI系统对 $\delta[n]$ 的响应称为系统的单位脉冲响应,记为 $h[n]$,通常较易求得
            \item 单位脉冲及其时移信号的集合是一个标准正交信号集
            \[
            \sum_{n=-\infty}^{\infty} \delta[n - k] \delta[n - m] = \begin{cases} 
            1 & k = m \\
            0 & k \neq m 
            \end{cases}
            \]
            \item 想象一个二维坐标的标准正交集合 $[0,1]^T, [1,0]^T$,任意一个二维点很容易分解为集合的线性加权组合
        \end{itemize}

        \item 已知一个LTI系统的 $h[n]$,求取系统对信号 $x[n]$ 的响应 $y[n]$
        \begin{itemize}
            \item 充分利用LTI系统的特性:
            \[
            \delta[n] \rightarrow h[n]
            \]
            \item $\delta[n - k] \rightarrow h[n - k]$ (时不变)
            \item $x[k] \delta[n - k] \rightarrow x[k] h[n - k]$ (齐次性)
            \item $x[n] = \sum_{k=-\infty}^{\infty} x[k] \delta[n - k] \rightarrow \sum_{k=-\infty}^{\infty} x[k] h[n - k] = y[n]$ (叠加性)
        \end{itemize}

        \item 离散时间LTI系统对信号 $x[n]$ 的响应为
        \[
        y[n] = \sum_{k=-\infty}^{\infty} x[k] h[n - k]
        \]
        \begin{itemize}
            \item 一个LTI系统的任意激励的响应可用系统的单位脉冲响应来表示,也即一个LTI系统可以用其单位脉冲响应 $h[n]$ 来完全表征
            \item 上述公式称为 $x[n]$ 和 $h[n]$ 的卷积和,表示为
            \[
            y[n] = x[n] * h[n] = \sum_{k=-\infty}^{\infty} x[k] h[n - k]
            \]
        \end{itemize}

        \item 例1(原理示意):$x[n] = \begin{cases} 
        0.5 & n = 0 \\
        2 & n = 1 \\
        0 & \text{else} 
        \end{cases}$,
        \[
        h[n] = \begin{cases} 
        1 & 0 \leq n \leq 2 \\
        0 & \text{else} 
        \end{cases}
        \]
        \[
        y[n] = x[n] * h[n] = \sum_{k=-\infty}^{+\infty} x[k] h[n - k] = 0.5h[n] + 2h[n - 1]
        \]

        \item 解2 (一般化求解过程示意) :
        \[
        y[n] = x[n] * h[n] = \sum_{k=-\infty}^{\infty} x[k] h[n - k]
        \]
        $x[k] h[n - k]$ 可以看成是在时刻 $k$,信号 $x[k]$ 对系统在时刻 $n$ 的输出 $y[n]$ 的贡献,将所有时刻的贡献相加得到 $y[n]$
        \[
        y[n] = \begin{cases} 
        0 & n < 0 \\
        0.5 & n = 0 \\
        2.5 & n = 1, 2 \\
        2 & n = 3 \\
        0 & n > 3 
        \end{cases}
        \]

        \item $y[n] = x[n] * h[n]$ 卷积和的一般性求解过程(对某个 $n$)
        \begin{enumerate}
            \item 换元:将变量 $n$ 替换为 $k, x[k], h[k]$
            \item 反转(卷):将 $h[k]$ 反转得到 $h[-k]$
            \item 平移:$h[-k]$ 平移得到 $h[n - k]$
            \item 乘积(积):将 $x[k]$ 与 $h[n - k]$ 对应相乘
            \item 求和(和):将乘积后的结果相加得到 $y[n]$
        \end{enumerate}

        \item 例2:$x[n] = \alpha^n u[n], h[n] = u[n], 0 < \alpha < 1$
        \begin{itemize}
            \item 当 $n < 0, y[n] = 0$
            \item 当 $n \geq 0$,
            \[
            x[k] h[n - k] = \begin{cases} 
            \alpha^k & 0 \leq k \leq n \\
            0 & \text{else} 
            \end{cases}
            \]
            \[
            y[n] = \sum_{k=0}^{n} \alpha^k = \frac{1 - \alpha^{n+1}}{1 - \alpha}
            \]
            \[
            y[n] = \frac{1 - \alpha^{n+1}}{1 - \alpha} u[n]
            \]
        \end{itemize}

        \item 例3:$x[n] = \begin{cases} 
        1 & 0 \leq n \leq 4 \\
        0 & \text{else} 
        \end{cases}$,
        \[
        h[n] = \begin{cases} 
        \alpha^n & 0 \leq n \leq 6 \\
        0 & \text{else} 
        \end{cases}
        \]
        \begin{itemize}
            \item 当 $n < 0, y[n] = 0$
            \item 当 $0 \leq n \leq 4$,
            \[
            x[k] h[n - k] = \begin{cases} 
            \alpha^{n-k} & 0 \leq k \leq n \\
            0 & \text{else} 
            \end{cases}
            \]
            \[
            y[n] = \sum_{k=0}^{n} \alpha^{n-k} = \frac{1 - \alpha^{n+1}}{1 - \alpha}
            \]
            \item 当 $4 \leq n \leq 6$,
            \[
            y[n] = \sum_{k=0}^{4} \alpha^{n-k} = \alpha^{n-4} \frac{1 - \alpha^5}{1 - \alpha}
            \]
            \item 当 $6 < n \leq 10$,
            \[
            y[n] = \sum_{k=n-6}^{4} \alpha^{n-k} = \alpha^{n-4} \frac{1 - \alpha^{11}}{1 - \alpha}
            \]
            \item 当 $n > 10, y[n] = 0$
        \end{itemize}

        \item 例4:$x[n] = 2^n u[-n], h[n] = u[n]$
        \begin{itemize}
            \item 当 $n \geq 0, y[n] = \sum_{k=-\infty}^{0} 2^k = \sum_{k=0}^{\infty} \left(\frac{1}{2}\right)^k = \frac{1}{1 - \frac{1}{2}} = 2$
            \item 当 $n < 0, y[n] = \sum_{k=-\infty}^{n} 2^k = 2^{n+1}$
        \end{itemize}
    \end{itemize}
\end{definition}

\section{连续时间LTI系统:卷积积分}

\begin{definition}
    用冲激表示连续时间信号
    \begin{itemize}
        \item 回顾:单位冲激信号的采样性质:$x(t)\delta(t - t_0) = x(t_0)\delta(t - t_0)$
        \item 是否可以类似离散信号来构建连续时间信号的筛选性质:
        \[
        x(t) = \int_{-\infty}^{+\infty} x(\tau) \delta(t - \tau) d\tau
        \]
        也即:任意一个连续时间信号可以表示为时移冲激信号的线性加权积分(和)
        \item 时移冲激信号作为基本简单信号,作为连续LTI系统时域分析的基础
        \item 证明:
        \[
        \int_{-\infty}^{+\infty} x(\tau) \delta(t - \tau) d\tau = \int_{-\infty}^{+\infty} x(t) \delta(t - \tau) d\tau = x(t) \int_{-\infty}^{+\infty} \delta(t - \tau) d\tau = x(t)
        \]
    \end{itemize}

    从离散时间到连续时间
    \begin{itemize}
        \item 构建函数 $\delta_\Delta(t) = \begin{cases} 
        \frac{1}{\Delta} & 0 \leq t < \Delta \\
        0 & \text{otherwise} 
        \end{cases}$,基于 $x(t)$ 构建阶梯函数 $\hat{x}(t)$
        \[
        \hat{x}(t) = \sum_{k=-\infty}^{+\infty} x(k\Delta) \delta_\Delta(t - k\Delta) \Delta
        \]
        \item 设 $\Delta \rightarrow 0$:有 $\hat{x}(t) \rightarrow x(t)$,$\Delta \rightarrow d\tau$,$k\Delta \rightarrow \tau$,$\sum \rightarrow \int$:
        \[
        x(t) = \lim_{\Delta \rightarrow 0} \sum_{k=-\infty}^{+\infty} x(k\Delta) \delta_\Delta(t - k\Delta) \Delta = \int_{-\infty}^{+\infty} x(\tau) \delta(t - \tau) d\tau
        \]
    \end{itemize}

    连续时间LTI系统的冲激响应及卷积积分表示
    \begin{itemize}
        \item 将系统的冲激响应记为 $h(t)$ 即 $\delta(t) \rightarrow h(t)$
        \begin{itemize}
            \item $\delta_\Delta(t) \rightarrow h_\Delta(t)$
            \item $\delta_\Delta(t - k\Delta) \rightarrow h_\Delta(t - k\Delta)$ (时不变)
            \item $x(k\Delta)\delta_\Delta(t - k\Delta) \Delta \rightarrow x(k\Delta)h_\Delta(t - k\Delta) \Delta$ (齐次)
            \item $\sum_{k=-\infty}^{+\infty} x(k\Delta) \delta_\Delta(t - k\Delta) \Delta \rightarrow \sum_{k=-\infty}^{+\infty} x(k\Delta) h_\Delta(t - k\Delta) \Delta$ (叠加)
            \item $x(t) = \lim_{\Delta \rightarrow 0} \sum_{k=-\infty}^{+\infty} x(k\Delta) \delta_\Delta(t - k\Delta) \Delta \rightarrow \lim_{\Delta \rightarrow 0} \sum_{k=-\infty}^{+\infty} x(k\Delta) h_\Delta(t - k\Delta) \Delta$
        \end{itemize}
    \end{itemize}

    \item 连续时间LTI系统对信号 $x(t)$ 的响应为
    \[
    y(t) = \int_{-\infty}^{+\infty} x(\tau) h(t - \tau) d\tau
    \]
    \begin{itemize}
        \item 一个连续时间LTI系统的任意输入的响应可用系统的单位冲激响应来表示,也即一个LTI系统可以用其单位冲激响应 $h(t)$ 来完全表征
        \item 上述公式称为 $x(t)$ 和 $h(t)$ 的卷积积分,表示为
        \[
        y(t) = x(t) * h(t) = \int_{-\infty}^{+\infty} x(\tau) h(t - \tau) d\tau
        \]
    \end{itemize}

    \item 卷积积分的一般性求解过程(对某个 $t$)
    \begin{enumerate}
        \item 换元:将变量 $t$ 替换为 $\tau$,得到 $x(\tau)$,$h(\tau)$
        \item 反转(卷):将 $h(\tau)$ 反转得到 $h(-\tau)$
        \item 平移:$h(-\tau)$ 平移得到 $h(t - \tau)$
        \item 乘积(积):将 $x(\tau)$ 与 $h(t - \tau)$ 对应相乘
        \item 求和(积分):将乘积后的结果对 $\tau$ 积分得到 $y(t)$
    \end{enumerate}

    \item 例1:$x(t) = e^{-at}u(t), a > 0, h(t) = u(t)$,求 $y(t) = x(t) * h(t)$
    \begin{itemize}
        \item 当 $t < 0, x(\tau) h(t - \tau) = 0$,因此 $y(t) = 0$
        \item 当 $t \geq 0, x(\tau) h(t - \tau) = e^{-a\tau}$,$0 \leq \tau \leq t$
        \[
        y(t) = \int_{0}^{t} e^{-a\tau} d\tau = -\frac{1}{a} e^{-a\tau} \Big|_{0}^{t} = \frac{1}{a} (1 - e^{-at})
        \]
    \end{itemize}

    \item 例2:$x(t) = \begin{cases} 
    1 & 0 < t < T \\
    0 & \text{otherwise} 
    \end{cases}$, \;$h(t) = \begin{cases} 
    t & 0 < t < 2T \\
    0 & \text{otherwise} 
    \end{cases}$\par
    求 $y(t) = x(t) * h(t)$
    \[
    y(t) = \begin{cases} 
    0 & t < 0 \\
    0.5t^2 & 0 \leq t < T \\
    Tt - 0.5T^2 & T \leq t < 2T \\
    -0.5t^2 + Tt + 1.5T^2 & 2T \leq t < 3T \\
    0 & t \geq 3T 
    \end{cases}
    \]

    \item 例3:$x(t) = e^{2t}u(-t), h(t) = u(t - 3)$,求 $y(t) = x(t) * h(t)$
    \begin{itemize}
        \item 当 $t - 3 < 0, x(\tau) h(t - \tau)$ 在 $-\infty < \tau < t - 3$ 时非零
        \item 当 $t - 3 \geq 0, x(\tau) h(t - \tau)$ 在 $-\infty < \tau < 0$ 时非零
        \[
        y(t) = \begin{cases} 
        \int_{-\infty}^{t-3} e^{2\tau} d\tau = \frac{1}{2} e^{2(t-3)} & t < 3 \\
        \int_{-\infty}^{0} e^{2\tau} d\tau = \frac{1}{2} & t \geq 3 
        \end{cases}
        \]
    \end{itemize}
\end{definition}

\section{LTI系统的性质}

\begin{definition}
    LTI系统的表征
    \begin{itemize}
        \item 单位冲激(脉冲)响应可以完全表征一个LTI系统,因为任意一个信号的系统响应可以由单位冲激(脉冲)响应和输入信号的卷积来表示
        \item 但不能完全表征一个一般(非线性、时变)系统,例如
        \[
        h[n] = \begin{cases} 
        1 & n = 0, 1 \\
        0 & \text{otherwise} 
        \end{cases}
        \]
        如果是LTI系统,那么
        \[
        y[n] = \sum_{k=-\infty}^{\infty} x[k] h[n - k] = x[n] + x[n - 1]
        \]
        如果是非线性系统,那么 $y[n]$ 将会有很多可能,如
        \[
        y[n] = \max\{x[n], x[n - 1]\}, \quad y[n] = (x[n] + x[n - 1])^2 \text{等}
        \]
        \item 因此,LTI系统的性质可以由卷积入手,从而导出LTI系统的一些特殊性质
    \end{itemize}
\vspace{1em}
    \textbf{\textcolor{red}{交换律}}
    \begin{itemize}
        \item $x[n] * h[n] = h[n] * x[n] = \sum_{k=-\infty}^{\infty} h[k] x[n - k]$
        \item $x(t) * h(t) = h(t) * x(t) = \int_{-\infty}^{\infty} h(\tau) x(t - \tau) d\tau$
        \item 证明:
        \[
        x[n] * h[n] = \sum_{k=-\infty}^{\infty} x[k] h[n - k], \quad \text{令} \, r = n - k \, \text{或} \, k = n - r
        \]
        即
        \[
        \sum_{k=-\infty}^{\infty} x[k] h[n - k] = \sum_{r=+\infty}^{-\infty} x[n - r] h[r] = h[n] * x[n]
        \]
        \item 信号(函数)和系统(单位冲激函数)可以互换
    \end{itemize}
\vspace{1em}
    \textbf{\textcolor{red}{分配律}}
    \begin{itemize}
        \item $x[n] * (h_1[n] + h_2[n]) = x[n] * h_1[n] + x[n] * h_2[n]$
        \item $(x_1[n] + x_2[n]) * h[n] = x_1[n] * h[n] + x_2[n] * h[n]$
        \item $x(t) * [h_1(t) + h_2(t)] = x(t) * h_1(t) + x(t) * h_2(t)$
        \item $[x_1(t) + x_2(t)] * h(t) = x_1(t) * h(t) + x_2(t) * h(t)$
        \item LTI系统叠加性的体现
    \end{itemize}
\vspace{1em}
    \textbf{\textcolor{red}{结合律}}
    \begin{itemize}
        \item $x[n] * (h_1[n] * h_2[n]) = (x[n] * h_1[n]) * h_2[n]$
        \item $x(t) * [h_1(t) * h_2(t)] = [x(t) * h_1(t)] * h_2(t)$
        \item LTI系统的级联与级联的顺序无关,但非LTI系统却一般没有这个特性,例如
        \[
        y_1[n] = 2x[n], \quad y_2[n] = x^2[n]
        \]
        \item 先 $y_1$ 再 $y_2$,则 $y_3[n] = 4x^2[n]$
        \item 先 $y_2$ 再 $y_1$,则 $y_3[n] = 2x^2[n]$
    \end{itemize}
\vspace{1em}
    \textbf{\textcolor{red}{有记忆和无记忆系统}}
    \begin{itemize}
        \item 一个系统的输出如果只取决于系统的当前输入,称为无记忆系统(又称即时系统),否则称为有记忆系统
        \item 对于离散LTI系统,$y[n] = \sum_{k=-\infty}^{\infty} h[k] x[n - k]$,无记忆意味着对于 $k \neq 0$ 有 $h[k] = 0$,即 $y[n] = h[0] x[n] = K x[n]$,$K = h[0]$ 为常数;令 $x[n] = \delta[n]$,得到 $h[n] = K \delta[n]$
        \item 无记忆LTI系统的单位脉冲响应为脉冲函数
        \item 对于连续LTI系统,$y(t) = \int_{-\infty}^{\infty} x(\tau) h(t - \tau) d\tau$,无记忆意味着对于 $\tau \neq t$ 有 $h(t - \tau) = 0$,也即 $y(t) = \int_{-\infty}^{\infty} x(t) h(0) d\tau = K x(t)$,$K = \int_{-\infty}^{\infty} h(0) d\tau$ 为常数;令输入为 $\delta(t)$,得到 $h(t) = K \delta(t)$
        \item 系统的单位冲激响应为冲激函数
        \item 令 $K = 1$,得到 $y[n] = x[n] = x[n] * \delta[n]$,$y(t) = x(t) = x(t) * \delta(t)$
        \item 信号与单位冲激(脉冲)函数卷积得到自身,也即单位冲激(脉冲)函数的筛选性质
    \end{itemize}
\vspace{1em}
    \textbf{\textcolor{red}{可逆性}}
    \begin{itemize}
        \item 一个系统如果是可逆的,那么存在一个逆系统,级联原系统后的输出就等于原系统的输入,也即级联后的系统冲激响应为 $\delta(t)$,也即
    \[
    h(t) * h_1(t) = \delta(t) \quad \text{或} \quad h[n] * h_1[n] = \delta[n]
    \]
    
        \item 例1
    $y(t) = x(t - t_0)$,求其逆系统。
    
    令 $x(t) = \delta(t)$,得到 $h(t) = \delta(t - t_0)$,
    有 $y(t) = x(t - t_0) = x(t) * h(t) = x(t) * \delta(t - t_0)$。
    
    一个信号与时移冲激(脉冲)信号的卷积就是信号的时移:
    \[
    \delta(t - t_0) * \delta(t + t_0) = \delta(t) \Rightarrow h_1(t) = \delta(t + t_0)
    \]
    
        \item 例2
    $h[n] = u[n]$,求其逆系统。
    
    有 $y[n] = \sum_{k=-\infty}^{\infty} x[k] h[n - k] = \sum_{k=-\infty}^{n} x[k]$。
    
    一个信号与阶跃脉冲(冲激)函数的卷积是信号的累加(积分)。
    
    已知:$u[n] - u[n - 1] = \delta[n]$,
    \[
    h[n] * h_1[n] = u[n] - u[n - 1] = u[n] * \delta[n] - u[n] * \delta[n - 1] = u[n] * (\delta[n] - \delta[n - 1])
    \]
    因此,$h_1[n] = \delta[n] - \delta[n - 1]$。
    \end{itemize}
\vspace{1em}
    \textbf{\textcolor{red}{因果性}}\par
    一个因果系统的输出只取决于系统的过去和现在的输入。
    
    对于离散时间因果LTI系统,
    \[
    y[n] = \sum_{k=-\infty}^{\infty} x[k] h[n - k]
    \]
    意味着当 $k > n$,必有 $h[n - k] = 0$,因此当 $n < 0$,有 $h[n] = 0$(激励出现之前,响应为零),也即
    \[
    y[n] = \sum_{k=-\infty}^{n} x[k] h[n - k] \quad \text{或} \quad y[n] = \sum_{k=0}^{\infty} h[k] x[n - k]
    \]
    
    一个因果系统的输出只取决于系统的过去和现在的输入。
    
    对于一个连续时间因果LTI系统,类似也有当 $t < 0$,有 $h(t) = 0$,
    \[
    y(t) = \int_{\tau = -\infty}^{t} x(\tau) h(t - \tau) d\tau \quad \text{或} \quad y(t) = \int_{\tau = 0}^{\infty} h(\tau) x(t - \tau) d\tau
    \]
    
    一个因果系统的输入在某个时刻前为0,则系统的响应在那个时刻之前也必为0,这就是系统的初始松弛条件;对于线性系统,因果性等价初始松弛条件;也就是系统只有零状态响应。
    
    因果性是系统的性质,但一般也将 $n < 0$ 或 $t < 0$ 时为零的信号也称为因果信号,经常表示为 $x(t)u(t)$ 或者 $x[n]u[n]$:一个LTI系统的因果性等价于系统的冲激(脉冲)响应是一个因果信号。\par
\vspace{1em}
\raggedright
    \textbf{\textcolor{red}{稳定性}}
    \begin{itemize}

    \item 如果一个系统的输入是有界的,那么输出也是有界的,该系统就是稳定的。
    
    对于离散时间LTI系统,
    \[
    |y[n]| = |\sum_{k=-\infty}^{\infty} h[k] x[n - k]| \leq \sum_{k=-\infty}^{\infty} |h[k]| |x[n - k]| \leq B \sum_{k=-\infty}^{\infty} |h[k]|
    \]
    因此,$|y[n]| < \infty$ 意味着 $\sum_{k=-\infty}^{\infty} |h[k]| < \infty$,称系统的单位脉冲响应绝对可和,这个条件是离散时间LTI系统稳定的充要条件。
    
    对于连续时间LTI系统,类似可以得到
    \[
    \int_{t = -\infty}^{\infty} |h(t)| dt < \infty
    \]
    称系统的单位冲激响应绝对可积,这个条件是连续时间LTI系统稳定的充要条件。
 
        \item 例1\\
    纯时移系统,$h[n] = \delta[n - n_0]$ 或 $h(t) = \delta(t - t_0)$
    \[
    \sum_{n=-\infty}^{\infty} |h[n]| = \sum_{n=-\infty}^{\infty} |\delta[n - n_0]| = 1
    \]
    \[
    \int_{t = -\infty}^{\infty} |h(t)| dt = \int_{t = -\infty}^{\infty} |\delta(t - t_0)| dt = 1
    \]
    因此,纯时移系统是稳定系统。
    
    \item 例2\\
    累加器或积分器,$h[n] = u[n]$ 或 $h(t) = u(t)$
    \[
    \sum_{n=-\infty}^{\infty} |h[n]| = \sum_{n=-\infty}^{\infty} |u[n]| = \sum_{n=0}^{\infty} 1 = \infty
    \]
    \[
    \int_{t = -\infty}^{\infty} |h(t)| dt = \int_{t = -\infty}^{\infty} |u(t)| dt = \int_{t = 0}^{\infty} u(t) dt = \infty
    \]
    因此,累加器或积分器不是稳定系统。
    \end{itemize}
    
    \vspace{1em}
    \raggedright
    \makebox[0pt][l]{\textbf{\textcolor{red}{单位阶跃响应}}}
    \begin{itemize}
    \item 阶跃函数与冲激函数之间存在积分(导数)的关系,因此可望利用阶跃响应来描述LTI系统
    \[
    s[n] = u[n] * h[n] = h[n] * u[n] = \sum_{k=-\infty}^{n} h[k]
    \]
    \[
    s(t) = u(t) * h(t) = \int_{\tau = -\infty}^{t} h(\tau) d\tau
    \]
    显然,单位阶跃响应和单位冲激响应之间存在如下关系:
    \[
    h[n] = s[n] - s[n - 1] \quad h(t) = \frac{ds(t)}{dt}  = s'(t)
    \]
    单位阶跃响应也可以表征一个LTI系统。
    \end{itemize}
\end{definition}

\section{用微/差分方程描述的因果LTI系统}

\begin{definition}
    现实中相当部分系统可用线性常系数微分/差分方程来描述和建模,
    对应的是LTI系统,例如RC电路:
    \[
    \frac{dV_c(t)}{dt} + \frac{1}{RC} V_c(t) = \frac{1}{RC} V_s(t)
    \]
    \begin{itemize}
        \item 如果参数如R与时间相关,则不是时不变系统
        \[
        \frac{dV_c(t)}{dt} + \frac{1}{R\left(t\right)C} V_c(t) = \frac{1}{R\left(t\right)C}  V_s(t)
        \]
        \item 如果R是非线性电阻,如 $v = Ri^2$,则不是线性系统
        \[
        [\frac{dV_c(t)}{dt}]^2 + \frac{1}{RC^2} V_c(t) = \frac{1}{RC^2} V_s(t)
        \]
        \item 如果系统是瞬时系统(无记忆系统),则可以利用代数方程描述
    \end{itemize}
\end{definition}
    \subsection{\textbf{线性常系数微分方程}}
\begin{definition}
    LCCDE:Linear Constant Coefficient Differential Equation\\
    一个N阶LCCDE:
    \[
    \sum_{k=0}^{N} a_k \frac{d^k y(t)}{dt^k} = \sum_{k=0}^{M} b_k \frac{d^k x(t)}{dt^k}
    \]
    \begin{itemize}
        \item N阶指的是y(t)的最高阶导数,微分方程给出的是系统输入输出间的一种约束关系,并不直接给出明确的表达式或者说没有完全表征系统;通常N > M
        \item LCCDE的完全解 $y(t) = \text{特解}y_p(t) + \text{齐次解}y_h(t)$
        \item $y_p(t)$ 是与输入 $x(t)$ 同类型函数
        \item $y_h(t)$ 是令 $\sum_{k=0}^{N} a_k \frac{d^k y(t)}{dt^k} = 0$ 的解,当齐次方程 $\sum_{k=0}^{N} a_k \lambda^k = 0$ 的特征根 $\lambda_i$ 均为单根时,$y_h(t) = \sum_{i=1}^{N} C_k e^{\lambda_i t}$
        \item $C_k$ 一般还需要给定附加条件才能求解,我们一般考虑因果LTI系统,也即初始松弛条件,意味着如果 $t \leq t_0, x(t) = 0$,那么有 $t \leq t_0, y(t) = 0$,那么可以利用初始松弛条件:
        \[
        y(t_0) = \frac{dy(t_0)}{dt} = \cdots = \frac{d^{N-1} y(t_0)}{dt^{N-1}} = 0
        \]
    \end{itemize}

    \textbf{例1:}考虑LCCDE:
    \[
    \frac{dy(t)}{dt} + 2y(t) = x(t), \quad \text{当} x(t) = K e^{3t} u(t) \quad \text{求} y(t)
    \]
    \begin{itemize}
        \item 求特解 $y_p(t)$,$y_p(t)$ 是与输入 $x(t)$ 同类型函数,当 $t > 0$,令 $y_p(t) = Y e^{3t}$,代入方程得到
        \[
        3Y e^{3t} + 2Y e^{3t} = K e^{3t} \Rightarrow Y = \frac{K}{5} \Rightarrow y_p(t) = \frac{K}{5} e^{3t}
        \]
        \item 求齐次解 $y_h(t)$,当 $t > 0$,$\sum_{k=0}^{N} a_k \lambda^k = \lambda + 2 = 0 \Rightarrow \lambda = -2$,设 $y_h(t) = C e^{-2t}$,$C$ 为待定常数
        \item 完全解 $y(t) = C e^{-2t} + \frac{K}{5} e^{3t}$
        \item 如果系统是因果的,因为有 $t < 0, x(t) = 0$,那么有 $t < 0, y(t) = 0$,即 $y(0) = 0 = C + \frac{K}{5} \Rightarrow C = -\frac{K}{5}$,当 $t > 0$,$y(t) = \frac{K}{5} [e^{3t} - e^{-2t}]$,得到 $y(t) = \frac{K}{5} [e^{3t} - e^{-2t}] u(t)$
    \end{itemize}

    \textbf{几个概念:}
    \begin{itemize}
        \item 特解是由输入信号和系统结构决定的,具有和输入信号相同的函数形式,所以又称为受迫响应;而齐次解(通解)和输入信号无关,所以也称为自然响应
        \item 一个LCCDE如果具有初始松弛条件,则LCCDE描述的系统是线性、非时变和因果的
        \item 如果一个LCCDE初始条件不为零,那么系统是增量线性的,系统响应=零输入响应+零状态响应(线性的)
        \begin{itemize}
            \item 零输入响应属于自然响应(和输入无关)
            \item 输入为因果信号时,初始松弛条件意味着零输入响应为零
        \end{itemize}
    \end{itemize}

    \textbf{例2:}考虑LCCDE:
    \[
    \frac{dy(t)}{dt} + 3y(t) = 3x(t), \quad \text{当} x(t) = u(t) \quad , y(0) = \frac{3}{2},\quad \text{求} y(t)
    \]
    \begin{itemize}
        \item 解1(特解+齐次解):求特解 $y_p(t)$,当 $t > 0$,令 $y_p(t) = Y$,代入方程得到
        \[
        3Y = 3 \Rightarrow Y = 1 \Rightarrow y_p(t) = 1
        \]
        求齐次解 $y_h(t)$,当 $t > 0$,$\lambda = -3$,设 $y_h(t) = C e^{-3t}$。得到完全解 $y(t) = C e^{-3t} + 1$。$y(0) = C + 1 = \frac{3}{2} \Rightarrow C = \frac{1}{2}$,因此 $y(t) = \frac{1}{2} e^{-3t} + 1$
        \item 解2(零输入响应+零状态响应):
        \begin{itemize}
            \item 求零输入响应 $y_{zi}(t)$:特解为0(零输入),特征根 $\lambda = -3$,设 $y_{zi}(t) = C e^{-3t}$,由初始条件得到 $y(0) = C = \frac{3}{2}$
            \item 求零状态响应 $y_{zs}(t)$:特解 $y_{zsp}(t) = 1$。设齐次解 $y_{zsh}(t) = C e^{-3t}$,$y_{zs}(t) = C e^{-3t} + 1$。由零状态初始条件 $y(0) = C + 1 = 0$ 得到 $C = -1$,因此 $y_{zs}(t) = -e^{-3t} + 1$
        \end{itemize}
        \[
        y(t) = y_{zi}(t) + y_{zs}(t) = \frac{3}{2} e^{-3t} - e^{-3t} + 1 = \frac{1}{2} e^{-3t} + 1
        \]
    \end{itemize}
\end{definition}


\subsection{\textbf{线性常系数差分方程}}

\begin{definition}
    LCCDE:Linear Constant Coefficient Difference Equation
    
    一个N阶LCCDE:
    \[
    \sum_{k=0}^{N} a_k y[n - k] = \sum_{k=0}^{M} b_k x[n - k]
    \]
    
    \begin{itemize}
        \item LCCDE不能完全表征输出,必须给出附加条件;一般情况下我们还是利用初始松弛条件:如果 $n < n_0, x[n] = 0$,那么 $n < n_0, y[n] = 0$,此时系统是因果的。
        \item 同微分方程一样,差分方程的求解也可以分为特解和齐次解来完成,但还可以表示成如下形式
        \[
        y[n] = \frac{1}{a_0} \left( \sum_{k=0}^{M} b_k x[n - k] - \sum_{k=1}^{N} a_k y[n - k] \right)
        \]
        要求出 $y[n]$,就需要知道 $y[n - 1], y[n - 2], \ldots, y[n - N]$
        \item 如果知道 $y[-1], y[-2], \ldots, y[-N]$,那么就可以通过递归的方式求得任意的 $y[n]$,方程也称为递归方程
    \end{itemize}
    
    \begin{itemize}
        \item 当 $N = 0$ 时,或者,当 $k > 0$ 时,$a_k = 0$,有 $y[n] = \sum_{k=0}^{M} \frac{b_k}{a_0} x[n - k]$
        \item 方程也称为非递归方程,是输入的显函数,无需递归求解,其单位脉冲响应为
        \[
        h[n] = 
        \begin{cases} 
        \frac{b_k}{a_0} & 0 \leq n \leq M \\
        0 & \text{otherwise}
        \end{cases}
        \]
        系统称为有限脉冲响应(FIR, Finite Impulse Response)系统;FIR系统是稳定系统
        \item 当 $N > 0$,或者当 $k > 0$ 时,$a_k$ 不全为0,此时系统称为无限脉冲响应(IIR, Infinite Impulse Response)系统
    \end{itemize}
    
    \textbf{例1:}求解差分方程 $y[n] - \frac{1}{2} y[n - 1] = x[n]$,$x[n] = \delta[n]$,系统满足初始松弛条件
    
    当 $n \leq -1$ 时,$x[n] = 0$,由初始松弛条件得到当 $n \leq -1$ 时,$y[n] = 0$,得到初始条件为 $y[-1] = 0$
    
    递归求解 $y[n]$:
    \[
    y[0] = x[0] + \frac{1}{2} y[-1] = 1
    \]
    \[
    y[1] = x[1] + \frac{1}{2} y[0] = \frac{1}{2}
    \]
    \[
    y[2] = x[2] + \frac{1}{2} y[1] = \frac{1}{2^2}
    \]
    \[
    \ldots
    \]
    \[
    y[n] = x[n] + \frac{1}{2} y[n - 1] = \frac{1}{2^n}
    \]
    得到系统的单位脉冲响应为 $h[n] = \frac{1}{2^n}$,具有无限长的脉冲响应
    
    \end{definition}

\begin{definition}

\begin{figure}[ht]
    \centering
    \includegraphics[width=1\textwidth]{fkt1.png}
    \end{figure}

\begin{figure}[ht]
    \centering
    \includegraphics[width=1\textwidth]{fkt2.png}
    \end{figure}

\begin{figure}[ht]
    \centering
    \includegraphics[width=1\textwidth]{fkt3.png}
    \end{figure}

\begin{figure}[ht]
    \centering
    \includegraphics[width=1\textwidth]{fkt4.png}
    \end{figure}

\end{definition}

\cleardoublepage
\section{奇异函数}

\begin{definition}
    \textbf{单位冲激函数}
    \begin{itemize}
        \item 单位冲激函数 $\delta(t)$
        \[
        x(t) = x(t) * \delta(t)\quad \delta(t) = \delta(t) * \delta(t)
        \]
        把 $\delta(t)$ 看成是 $\delta_\Delta(\tau)$ 的极限,并定义
        \[
        r_\Delta(t) = \delta_\Delta(\tau) * \delta_\Delta(\tau)
        \]
        当 $\Delta \to 0$,$\delta_\Delta(\tau) * \delta_\Delta(\tau) \to \delta(t) * \delta(t) = \delta(t)$
        因此,
        \[
        \lim_{\Delta \to 0} r_\Delta(t) = \delta(t)
        \]

        考虑一个LTI系统,其单位冲激响应为
        \[
        h(t) = e^{-2t}u(t)
        \]
        不同的激励得到的响应对比:
        \begin{itemize}
            \item 当 $\Delta$ 足够小,几个激励信号的响应都像是单位冲激响应
        \end{itemize}

        \item 通过卷积定义单位冲激函数:
        \[
        x(t) = x(t) * \delta(t)
        \]
        满足此等式的函数为单位冲激函数
        \item 等价定义:
        \[
        g(0) = \int_{-\infty}^{\infty} g(\tau) \delta(\tau) d\tau
        \]
    \end{itemize}

    \textbf{单位冲激偶}
    \begin{itemize}
        \item 单位冲激偶 $u_1(t)$ 或 $\delta'(t)$\par
        定义一个LTI系统:
        \[
        y(t) = \frac{dx(t)}{dt}
        \]
        其单位冲激响应为:
        \[
        h(t) = \frac{d\delta(t)}{dt} \triangleq u_1(t)
        \]
        对信号 $x(t)$ 的响应为:
        \[
        \frac{dx(t)}{dt} = u_1(t) * x(t)
        \]

        \item 一个信号与单位冲激偶的卷积是信号的导数,$k$ 次卷积则为 $k$ 阶导数,记为
        \[
        u_k(t) = u_1(t) * u_1(t) * \cdots * u_1(t)
        \]
        \[
        \frac{d^k x(t)}{dt^k} = u_k(t) * x(t)
        \]
        \item 单位冲激偶的面积为零(图形理解)
        \item 冲激偶的筛选特性:
        \[
        \int_{-\infty}^{\infty} \delta'(t) f(t) dt = -f'(0)
        \]
    \end{itemize}

    \textbf{单位阶跃函数}
    \begin{itemize}
        \item 单位阶跃函数 $u(t)$ 和单位斜坡函数 $u_{-2}(t)$
        定义一个LTI系统:
        \[
        y(t) = \int_{-\infty}^{t} x(\tau) d\tau
        \]
        其单位冲激响应为:
        \[
        h(t) = \int_{-\infty}^{t} \delta(\tau) d\tau \triangleq  u(t)
        \]
        对信号 $x(t)$ 的响应为:
        \[
        u(t) * x(t) = \int_{-\infty}^{t} x(\tau) d\tau
        \]
        是一阶积分

        \item 进一步,设输入 $x(t) = u(t)$,定义 $u_{-2}(t) \triangleq u(t) * u(t) = \int_{-\infty}^{t} u(\tau) d\tau = tu(t)$
        \[
        u_{-2}(t) * x(t) = u(t) * u(t) * x(t) = \left(\int_{-\infty}^{t} x(\sigma)d\sigma \right) * u(t) = \int_{-\infty}^{t} \left( \int_{-\infty}^{\tau} x(\sigma) d\sigma \right) d\tau
        \]
        :二阶积分

        \item $\delta(\tau)$ 的 $k$ 阶积分可以定义为多个积分器的级联的单位冲激响应
        \[
        u_{-k}(t) = u(t) * u(t) * \cdots * u(t) = \int_{-\infty}^{t} u_{-(k-1)}(\tau) d\tau
        \]
        \[
        x(t) * u^{-k}(t) = x(t) \text{的 k 阶积分}
        \]
    \end{itemize}
\end{definition}


\chapter{周期信号的傅里叶级数表示}

\begin{definition}
    \textbf{\fontsize{14}{16}\selectfont 引言}

    从时域分析到频域分析
    \begin{itemize}
        \item 时域分析:把信号表示成一组时移冲激信号的加权和,将系统响应表述为卷积和与卷积积分
        \item 频域分析:把信号表示成一组复指数信号的加权和,而LTI系统对复指数信号的响应形式一般相对简单,且同时可以将时域的卷积转换为频域的代数运算
    \end{itemize}

    \textbf{在频域对信号进行处理的好处:}
    \begin{itemize}
        \item 直接对频率进行操作,有时候比较直观(例如过滤掉图像的高频噪声部分等)
        \item 代数运算比卷积计算要高效直接,并且有快速的实现算法(快速离散傅里叶变换等)
    \end{itemize}
\end{definition}


\section{LTI系统对复指数信号的响应}

\begin{definition}
    将信号表示成基本信号的线性组合是信号与LTI系统分析的核心
    \begin{itemize}
        \item 表述称为时移的冲激(脉冲)信号组合,导出系统响应的卷积和与卷积积分
        \[
        x[n] = \sum_{k=-\infty}^{\infty} x[k] \delta[n - k], \quad x(t) = \int_{-\infty}^{\infty} x(\tau) \delta(t - \tau) d\tau
        \]
        \item 基本信号选取的基本原则:
        \begin{itemize}
            \item 由这些基本信号能够构成相当广泛的一类有用信号
            \item LTI系统对基本信号的响应应该简单明了
        \end{itemize}
        \item 傅里叶分析采用复指数信号 $e^{st}$($s$ 为复数,连续时间)和 $z^n$($z$ 为复数,离散时间)作为基本信号
    \end{itemize}

    \textbf{\textcolor{blue}{一个LTI系统对一个复指数信号的响应也是此复指数信号,不同的只是幅度变化}}
    \begin{itemize}
        \item 连续时间系统:当系统输入 $x(t) = e^{st}$,系统响应 $y(t) = \int_{-\infty}^{\infty} e^{s\tau} h(t - \tau) d\tau = e^{st} \int_{-\infty}^{\infty} h(\tau) e^{-s\tau} d\tau$
        \[
        y(t) = H(s) e^{st}
        \]
        其中:$H(s) = \int_{-\infty}^{\infty} h(\tau) e^{-s\tau} d\tau$
        \item $e^{st}$ 是连续时间LTI系统的特征函数,$H(s)$ 是对应的特征值(常数,与 $t$ 无关)
        \item 一个信号,如果系统对它的响应仅是一个常数(可能是复数)乘以它,那么这个信号称为系统的特征函数,对应的常数称为系统的特征值
    \end{itemize}

    离散时间系统:当系统输入 $x[n] = z^n$,有 $y[n] = \sum_{k=-\infty}^{\infty} h[k] z^{n-k} = z^n \sum_{k=-\infty}^{\infty} h[k] z^{-k}$
    \[
    y[n] = z^n H(z)
    \]
    其中:$H(z) = \sum_{k=-\infty}^{\infty} h[k] z^{-k}$
    \item $z^n$ 是离散时间LTI系统的特征函数,$H(z)$ 是对应的特征值(常数,与 $n$ 无关)\par
\vspace{2em}
    如果一个连续时间LTI系统的输入 $x(t)$ 可以表示为复指数信号的线性组合,即 
    \[
    x(t) = \sum_{k} a_k e^{s_k t}
    \]
    那么系统的输出响应可以表示为
    \[
    y(t) = \sum_{k} a_k H(s_k) e^{s_k t}
    \]

    如果一个离散时间LTI系统的输入 $x[n]$ 可以表示为复指数信号的线性组合,即 $x[n] = \sum_{k} a_k z_k^n$
    \[
    \text{那么系统的输出响应可以表示为} y[n] = \sum_{k} a_k H(z_k) z_k^n
    \]

    究竟多大范围的信号可以表示为复指数信号的线性组合?
    \begin{itemize}
        \item $s, z$ 可以为任意复数,但连续时间傅里叶分析仅限于 $s = j\omega$,即信号为 $e^{j\omega t}$ 的形式,离散时间傅里叶分析仅考虑单位振幅的 $z = e^{j\omega}$,即信号为 $e^{j\omega n}$ 的形式;后期将考虑一般的复数形式。
    \end{itemize}

    例1.1(简单例子说明LTI系统的特征值与特征函数):一个LTI系统:$y(t) = x(t - 3)$,求系统对输入 $x(t) = e^{j2t}$ 的响应
    \begin{itemize}
        \item 直接代入得到:$y(t) = e^{j2(t-3)} = e^{-j6} e^{j2t}$
        \item 利用特征函数性质:$h(t) = \delta(t - 3) \Rightarrow H(s) = \int_{-\infty}^{\infty} \delta(\tau - 3) e^{-s\tau} d\tau = e^{-3s}$
        \item $y(t) = H\left(s\right)e^{st} = H(j2)e^{j2t} = e^{-j6} e^{j2t}$
    \end{itemize}

    例1.2(简单例子说明LTI系统的特征值与特征函数):一个LTI系统:$y(t) = x(t - 3)$,求系统对输入 $x(t) = \cos(4t) + \cos(7t)$ 的响应
    \begin{itemize}
        \item 显然,$y(t) = \cos(4(t - 3)) + \cos(7(t - 3))$
        \item 利用特征函数性质,$x(t) = \frac{1}{2} (e^{j4t} + e^{-j4t} + e^{j7t} + e^{-j7t})$
        \item $y(t) = \frac{1}{2} (H(j4)e^{j4t} + H(-j4)e^{-j4t} + H(j7)e^{j7t} + H(-j7)e^{-j7t}) = \frac{1}{2} (e^{-j12}e^{j4t} + e^{j12}e^{-j4t} + e^{-j21}e^{j7t} + e^{j21}e^{-j7t}) = \cos\left(4\left(t-3\right)\right) + \cos\left(7\left(t-3\right)\right)$
    \end{itemize}

    究竟多大范围的信号可以表示为复指数信号的线性组合?
    \begin{itemize}
        \item $s, z$ 可以为任意复数,但连续时间傅里叶分析仅限于 $s = j\omega$,即信号为 $e^{j\omega t}$ 的形式,离散时间傅里叶分析仅考虑单位振幅的 $z = e^{j\omega}$,即信号为 $e^{j\omega n}$ 的形式。
    \end{itemize}
\end{definition}


\section{连续时间周期信号傅里叶级数表示}

\begin{definition}
    成谐波关系的复指数信号的线性组合
    成谐波关系的复指数信号集合
    \[
    \phi_k(t) = e^{jk\omega_0 t}, \quad k = 0, \pm1, \pm2, \ldots
    \]
    集合中每个信号都是周期的,第 $k$ 个信号的(除 $k = 0$ 外)基波频率为 $k\omega_0$,都是 $\omega_0$ 的整数倍,第 $k$ 个信号的基波周期为 $T = \frac{2\pi}{k|\omega_0|}$,称为 $k$ 次谐波分量,信号的公共周期是 $T = \frac{2\pi}{|\omega_0|}$。

    由成谐波关系复指数信号集合线性组合而成如下的周期信号
    \[
    x(t) = \sum_{k=-\infty}^{\infty} a_k e^{jk\omega_0 t} = \sum_{k=-\infty}^{\infty} a_k e^{jk\left(\frac{2\pi}{T}\right)t}
    \]
    是以 $T$ 为周期的周期信号。反过来说,如果一个信号 $x(t)$ 能够表述成上述形式,则称为 $x(t)$ 的傅里叶级数表示,显然,我们还需要确定 $a_k$。

    例1:周期信号 $x(t)$ 的基波频率为 $\omega_0 = 2\pi$,表示成
    \[
    x(t) = \sum_{k=-3}^{3} a_k e^{jk2\pi t}
    \]
    其中 $a_0 = 1$,$a_1 = a_{-1} = \frac{1}{4}$,$a_2 = a_{-2} = \frac{1}{2}$,$a_3 = a_{-3} = \frac{1}{3}$。
    \[
    x(t) = 1 + \frac{1}{4} e^{j2\pi t} + \frac{1}{4} e^{-j2\pi t} + \frac{1}{2} e^{j4\pi t} + \frac{1}{2} e^{-j4\pi t} + \frac{1}{3} e^{j6\pi t} + \frac{1}{3} e^{-j6\pi t}
    \]
    \[
    = 1 + \frac{1}{2} \cos 2\pi t + \cos 4\pi t + \frac{2}{3} \cos 6\pi t
    \]

    类似于例1中的信号 $x(t)$ 为实函数,有 $x(t) = x^*(t)$,也就是
    \[
    \sum_{k=-\infty}^{\infty} a_k e^{jk\omega_0 t} = \sum_{k=-\infty}^{\infty} a_k^* e^{-jk\omega_0 t} = \sum_{k=-\infty}^{\infty} a_{-k}^* e^{jk\omega_0 t}
    \]
    因此,对于实函数,必有 $a_k = a_{-k}^*$ 或 $a_k^* = a_{-k}$,共轭对称。

    实周期函数的傅里叶级数的三角函数表达形式
    \[
    x(t) = \sum_{k=-\infty}^{\infty} a_k e^{jk\omega_0 t} = a_0 + \sum_{k=1}^{\infty} (a_k e^{jk\omega_0 t} + a_{-k} e^{-jk\omega_0 t})
    \]
    \[
    = a_0 + \sum_{k=1}^{\infty} (a_k e^{jk\omega_0 t} + a_k^* e^{-jk\omega_0 t}) = a_0 + \sum_{k=1}^{\infty} 2\Re\{a_k e^{jk\omega_0 t}\}
    \]
    将 $a_k$ 表述为 $a_k = A_k e^{j\theta_k}$,
    \[
    x(t) = a_0 + \sum_{k=1}^{\infty} 2A_k \Re\{e^{j(k\omega_0 t + \theta_k)}\} = a_0 + 2 \sum_{k=1}^{\infty} A_k \cos(k\omega_0 t + \theta_k)
    \]

    系数 $a_k$ 的确定:\par
    在 $x(t) = \sum_{k=-\infty}^{\infty} a_k e^{jk\omega_0 t}$ 左右两边同时乘以 $e^{-jn\omega_0 t}$,
    \[
    x(t) e^{-jn\omega_0 t} = \sum_{k=-\infty}^{\infty} a_k e^{jk\omega_0 t} e^{-jn\omega_0 t}
    \]
    在 $x(t)$ 的一个基波周期 $T = \frac{2\pi}{\omega_0}$ 内积分
    \[
    \int_{0}^{T} x(t) e^{-jn\omega_0 t} dt = \int_{0}^{T} \sum_{k=-\infty}^{\infty} a_k e^{j(k-n)\omega_0 t} dt = \sum_{k=-\infty}^{\infty} a_k \int_{0}^{T} e^{j(k-n)\omega_0 t} dt
    \]
    \[
    \int_{0}^{T} e^{j(k-n)\omega_0 t} dt = \int_{0}^{T} \cos[(k-n)\omega_0 t] dt + j \int_{0}^{T} \sin[(k-n)\omega_0 t] dt
    \]
    其中 $\cos[(k-n)\omega_0 t]$ 和 $\sin[(k-n)\omega_0 t]$ 的基波周期为 $T/|k-n|$,因此 $T$ 也是其周期。

    连续时间周期信号傅里叶级数表示的确定
    \[
    \int_{0}^{T} e^{j(k-n)\omega_0 t} dt = \begin{cases}
    T, & k = n \\
    0, & \text{otherwise}
    \end{cases}
    \]
    \[
    \sum_{k=-\infty}^{\infty} a_k \int_{0}^{T} e^{j(k-n)\omega_0 t} dt = Ta_n = \int_{0}^{T} x(t) e^{-jn\omega_0 t} dt
    \]
    因此得到:
    \[
    a_n = \frac{1}{T} \int_{0}^{T} x(t) e^{-jn\omega_0 t} dt = \frac{1}{T} \int_{T} x(t) e^{-jn\omega_0 t} dt
    \]

    综合公式:
    \[
    x(t) = \sum_{k=-\infty}^{\infty} a_k e^{jk\omega_0 t} = \sum_{k=-\infty}^{\infty} a_k e^{jk\left(\frac{2\pi}{T}\right)t}
    \]
    分析公式:
    \[
    a_k = \frac{1}{T} \int_{T} x(t) e^{-jk\omega_0 t} dt = \frac{1}{T} \int_{T} x(t) e^{-jk\left(\frac{2\pi}{T}\right)t} dt
    \]
    \begin{itemize}
    \item $a_k$:傅里叶级数系数、频谱系数,$x(t)$ 中每一个谐波分量大小的度量。\par
    \item $a_0$:$a_0 = \frac{1}{T} \int_{T} x(t) dt$ 代表了信号 $x(t)$ 的直流分量,也就是一个周期的均值。
    \end{itemize}
\end{definition}

\begin{definition}
    例1:求信号 $x(t) = \sin(\omega_0 t)$ 的傅里叶级数系数
    \[
    \sin(\omega_0 t) = \frac{1}{2j} e^{j\omega_0 t} - \frac{1}{2j} e^{-j\omega_0 t}
    \]
    \[
    a_1 = \frac{1}{2j}, \quad a_{-1} = -\frac{1}{2j}, \quad a_k = 0 \quad \text{当} \quad k \neq 1 \quad \text{或} \quad -1
    \]

    例2:求信号 $x(t) = 1 + \sin(\omega_0 t) + 2 \cos(\omega_0 t) + \cos(2\omega_0 t + \frac{\pi}{4})$ 的傅里叶级数系数
    \[
    x(t) = 1 + \frac{1}{2j} e^{j\omega_0 t} - \frac{1}{2j} e^{-j\omega_0 t} + e^{j\omega_0 t} + e^{-j\omega_0 t} + \frac{1}{2} e^{j(2\omega_0 t + \frac{\pi}{4})} + \frac{1}{2} e^{-j(2\omega_0 t + \frac{\pi}{4})}
    \]
    \[
    = 1 + \left(1 + \frac{1}{2j}\right) e^{j\omega_0 t} + \left(1 - \frac{1}{2j}\right) e^{-j\omega_0 t} + \frac{1}{2} e^{j\frac{\pi}{4}} e^{j2\omega_0 t} + \frac{1}{2} e^{-j\frac{\pi}{4}} e^{-j2\omega_0 t}
    \]

    例3:信号 $x(t + NT) = 
    \begin{cases} 
    1, & |t| < T_1 \\
    0, & T_1 < |t| < T/2 
    \end{cases}$,$N = 0,1,2, \ldots$ 表示周期为 $T$ 的方波,基波频率 $\omega_0 = \frac{2\pi}{T}$,求傅里叶级数系数
    \[
    a_0 = \frac{1}{T} \int_{-\frac{T}{2}}^{\frac{T}{2}} x(t) dt = \frac{1}{T} \int_{-T_1}^{T_1} 1 dt = \frac{2T_1}{T}
    \]
    \[
    a_k = \frac{1}{T} \int_{-T_1}^{T_1} e^{-jk\omega_0 t} dt = -\frac{1}{jk\omega_0 T} e^{-jk\omega_0 t} \Bigg|_{-T_1}^{T_1} = \frac{2 \sin(k\omega_0 T_1)}{k\omega_0 T} = \frac{ \sin(k\omega_0 T_1)}{k\pi}
    \]
    \[
    a_0 = \frac{2T_1}{T}, \quad a_k = \frac{\sin(k\omega_0 T_1)}{k\pi} = \frac{\sin(k\frac{2\pi}{T} T_1)}{k\pi} = \frac{2T_1}{T} \text{Sa}(k\omega_0 T_1)
    \]
    \[
    a_0 = \frac{2T_1}{T} \quad \text{又称占空比;} \quad \text{Sa}(x) \triangleq  \frac{\sin x}{x} \quad \text{又名抽样函数,是一个偶函数,} \quad x = k\pi \quad (k \neq 0) \quad \text{时过零点}
    \]
    频谱的离散性,谱线的间隔为 $\omega_0$,且具有包络线 $\frac{2T_1}{T} \text{Sa}(\omega T_1)$;有时也讨论 $T a_k = {2T_1 \text{Sa}(\omega T_1)}\Bigg|_{\omega = k\omega_0}$;频谱具有收敛性,总趋势减小。
\end{definition}


\begin{definition}
    傅里叶级数的收敛
    究竟什么样的函数能够利用傅里叶级数来表示?欧拉和拉格朗日对综合公式和分析公式的适用范围持保留态度,而傅里叶认为任何周期信号都可以。\par
    综合公式: 
    \[
    x(t) = \sum_{k=-\infty}^{\infty} a_k e^{jk\omega_0 t}
    \]\par
    分析公式:
    \[
    a_k = \frac{1}{T} \int_{T} x(t) e^{-jk\omega_0 t} dt
    \]
    收敛的两层意义:
    \begin{itemize}
    \item 分析公式:$a_k$ 是否收敛
    \item 综合公式:当 $a_k$ 收敛的时候,能否收敛于真值
    \end{itemize}

    \textbf{\textcolor{blue}{傅里叶级数的两类(充分)收敛条件}}\par
    \vspace{1em}
    1.平方可积条件\par
    即 $x(t)$ 在一个周期内的能量有限,则保证 $a_k$ 为有限值
    \[
    \int_{T} |x(t)|^2 dt < \infty
    \]\par
    且保证了当综合得到的 
    \[
    \hat{x}(t)  = \sum_{k=-\infty}^{\infty} a_k e^{jk\omega_0 t}
    \]
    与 $x(t)$ 之间的能量差别为零,但并不是说 $\hat{x}(t)$ 和 $x(t)$ 在每个点的值都相等\par
\vspace{1em}
    2.狄利赫里条件:\par
    \begin{itemize}
    \item 任何周期内,$x(t)$ 绝对可积,即 
    \[
    \int_{T} |x(t)| dt < \infty
    \]
    保证 $a_k$ 为有限值
    \item 任何周期内,$x(t)$ 的最大值和最小值的数目有限
    \item $x(t)$ 的任何有限区间,只有有限个不连续点,而且在不连续点上,函数是有限值
    \end{itemize}
    满足狄利赫里条件意味着:\par
    1. $a_k$ 是有限值;\par
    2. 一个不存在间断点的周期信号,傅里叶级数收敛且逐点相等\par
    3. 存在有限不连续点的周期信号,除了孤立的不连续点,逐点相等;且在不连续点上,级数收敛于不连续点处的均值;两者在任意区间的积分是一样的,因此,在卷积的意义下,或者说从LTI系统分析角度来看,两者是完全一致的\par
\vspace{1em}
    吉伯斯(Gibbs)现象\par
    \begin{itemize}
    \item 一个不连续信号的傅里叶级数,在不连续点附近将呈现出高频起伏和超量,且其幅度不会随着级数的项数增加而减小
    \item 不连续点处级数的值等于不连续点的均值\par
    \item 在极限 $N \to \infty$ 情况下,误差的能量为零
    \end{itemize}
    \vspace{1em}
    吉伯斯(Gibbs)现象\par
    例如,对于周期方波信号:
    \[
    x_N(t) = \sum_{k=-N}^{N} a_k e^{jk\omega_0 t} = \sum_{k=-N}^{N} \frac{2T_1}{T} \text{Sa}(k\omega_0 T_1) e^{jk\omega_0 t} = \frac{2T_1}{T} \left(1 + \sum_{k=1}^{N} \text{Sa}(k\omega_0 T_1) \cos(k\omega_0 t)\right)
    \]
\end{definition}



\section{连续时间周期信号的傅里叶级数性质}

\subsection{线性性质}
令 $x(t)$ 和 $y(t)$ 为两个同为周期 $T$ 的周期信号,它们的傅里叶级数系数分别为 $a_k$ 和 $b_k$,即
\[
x(t) \xleftrightarrow{FS} a_k, \quad y(t) \xleftrightarrow{FS} b_k
\]
令 $z(t) = A x(t) + B y(t)$,则有:
\[
z(t) \xleftrightarrow{FS} A a_k + B b_k
\]
适用于任意多个具有同周期的信号。

\subsection{时移性质}
$x(t)$ 为以 $T$ 为周期的周期信号,记 $x(t) \xleftrightarrow{FS} a_k$,则有:
\[
x(t - t_0) \xleftrightarrow{FS} e^{-jk\omega_0 t_0} a_k
\]
时间上的位移不影响傅里叶级数系数的幅度,但影响系数的相位,且与 $k$ 成线性关系。

\subsection{时间反转}
$x(t)$ 为以 $T$ 为周期的周期信号,记 $x(t) \xleftrightarrow{FS} a_k$,则有:
\[
x(-t) \xleftrightarrow{FS} a_{-k}
\]
时间上的反转导致傅里叶级数系数的反转。
如果 $x(-t) = x(t)$,则有 $a_{-k} = a_k$;如果 $x(-t) = -x(t)$,那么 $a_{-k} = -a_k$。

\subsection{时间尺度变换}
$x(t)$ 为以 $T$ 为周期的周期信号,记 $x(t) \xleftrightarrow{FS} a_k$,则有:
\[
x_1(t) = x(\alpha t) \xleftrightarrow{FS} a_k
\]
傅里叶级数系数保持不变,但注意此时新的基波周期 $T_1 = T/\alpha$ 和频率 $\omega_1 = \alpha \omega_0$。
\[
b_k = \frac{1}{T_1} \int_{T_1} x(\alpha t) e^{-jk(\alpha \omega_0)t} dt = \frac{\alpha}{T} \int_{T} x(\tau) e^{-jk\omega_0 \tau} d\tau/\alpha = \frac{1}{T} \int_{T} x(\tau) e^{-jk\omega_0 \tau} d\tau = a_k
\]

\subsection{相乘性质}
令 $x(t)$ 和 $y(t)$ 为两个同为周期 $T$ 周期信号,它们的傅里叶级数系数分别为 $a_k$ 和 $b_k$,则有:
\[
z(t) = x(t) y(t) \xleftrightarrow{FS} z_k = \sum_{l=-\infty}^{\infty} a_l b_{k-l}
\]
时间域的相乘对应于傅里叶级数系数的卷积。

\subsection{周期卷积}
令 $x(t)$ 和 $y(t)$ 为两个同为周期 $T$ 周期信号,它们的傅里叶级数系数分别为 $a_k$ 和 $b_k$,则有:
\[
z(t) = \int_{T} x(\tau) y(t - \tau) d\tau \xleftrightarrow{FS} z_k = T a_k b_k
\]
时间域的卷积对应于傅里叶级数系数的相乘。

\subsection{微分性质}
$x(t)$ 为以 $T$ 为周期的周期信号,记 $x(t) \xleftrightarrow{FS} a_k$,则有:
\[
\frac{d x(t)}{dt} \xleftrightarrow{FS} jk\omega_0 a_k
\]
证明:
\[
\frac{d x(t)}{dt} = \frac{d}{dt} \sum_{k=-\infty}^{\infty} a_k e^{jk\omega_0 t} = \sum_{k=-\infty}^{\infty} jk\omega_0 a_k e^{jk\omega_0 t}
\]
微分增强了频率高的部分。

\subsection{共轭与共轭对称性质}
$x(t)$ 为以 $T$ 为周期的周期信号,记 $x(t) \xleftrightarrow{FS} a_k$,则有:
\[
x^*(t) \xleftrightarrow{FS} a_{-k}^*
\]
证明:
\[
x^*(t) = \sum_{k=-\infty}^{\infty} a_k^* e^{-jk\omega_0 t} = \sum_{k=-\infty}^{\infty} a_{-k}^* e^{jk\omega_0 t}
\]\par
当 $x(t)$ 为实函数,$a_k = a_{-k}^* \Rightarrow a_{-k} = a_k^*$,系数共轭对称。\par
当 $x(t)$ 为实偶函数,$a_{-k} = a_k = a_k^*$,系数为实偶函数。\par
当 $x(t)$ 为实奇函数,$a_{-k} = -a_k = a_k^*$,系数为纯虚奇函数,可以推出 $a_0 = 0$。

\subsection{帕塞瓦尔定理}
\[
\frac{1}{T} \int_{T} |x(t)|^2 dt = \sum_{k=-\infty}^{\infty} |a_k|^2
\]
周期信号在一个周期内的平均功率等于傅里叶级数的系数功率之和。

\[
\sum_{k=-\infty}^{\infty} |a_k|^2 = \sum_{k=-\infty}^{\infty} a_k a_k^* = \sum_{k=-\infty}^{\infty} a_k \frac{1}{T} \int_{T} x^*(t) e^{jk\omega_0 t} dt
\]
\[
= \frac{1}{T} \int_{T} x^*(t) \sum_{k=-\infty}^{\infty} a_k e^{jk\omega_0 t} dt = \frac{1}{T} \int_{T} x^*(t) x(t) dt = \frac{1}{T} \int_{T} |x(t)|^2 dt
\]

\subsection{性质的利用}
\textbf{例1.1:求信号 $g(t)$ 的傅里叶级数表示,信号基波周期 $T = 4$,基波频率 $\omega_0 = \frac{\pi}{2}$。}

\[
g(t) = x(t - 1) - \frac{1}{2}
\]
\[
x(t) \xleftrightarrow{FS} \frac{\sin(k\pi/2)}{k\pi} = \frac{1}{2} \text{Sa}(k\pi/2)
\]
为方波周期信号,利用时移性质得到
\[
x(t - 1) \xleftrightarrow{FS} \frac{1}{2} \text{Sa}(k\pi/2) e^{-jk\pi/2}
\]
得到 $g(t)$ 的傅里叶系数
\[
d_k = \begin{cases}
\frac{1}{2} \text{Sa}(k\pi/2) e^{-jk\pi/2}, \quad k \neq 0\\
d_0 = 0, \quad k = 0
\end{cases}
\]
$d_0$可以直接观察得到。

\textbf{例1.2:进一步求如图信号 $x(t)$ 的傅里叶级数}
\begin{figure}[h]
    \centering
    \includegraphics[width=0.4\textwidth]{li2.png} % 替换为你的图片文件名
\end{figure}

\[
\frac{d x(t)}{dt} = g(t)
\]
因此有
\[
d_k = jk\omega_0 a_k
\]
也即
\[
a_k = \frac{d_k}{jk\omega_0} = \frac{2d_k}{jk\pi} = \frac{1}{jk\pi} \text{Sa}(k\pi/2) e^{-jk\pi/2}, \quad k \neq 0
\]
\[a_k = \begin{cases}
\frac{1}{jk\pi} \text{Sa}(k\pi/2) e^{-jk\pi/2}, \quad k \neq 0\\
\frac{1}{2}, \quad k = 0
\end{cases}
\]


\textbf{例1.3:(周期冲激串)求 $x(t) = \sum_{k=-\infty}^{\infty} \delta(t - kT)$ 的傅里叶级数}

\[
a_k = \frac{1}{T} \int_{-\frac{T}{2}}^{\frac{T}{2}} \delta(t) e^{-jk\omega_0 t} dt = \frac{1}{T}
\]
也即:
\[
\sum_{k=-\infty}^{\infty} \delta(t - kT) = a_0 + 2 \sum_{k=1}^{\infty} a_k \cos(k\omega_0 t)
\]
\[
\hat{x}(t) = \frac{1}{T} + \frac{2}{T} \sum_{k=1}^{N} \cos(k\omega_0 t)
\]

\textbf{例1.4:(周期冲激串的应用):利用周期冲激串 $x(t)$ 求周期方波 $y(t)$ 的傅里叶级数}
\[
q(t) \triangleq  \frac{d y(t)}{dt} = \delta(t + T_1) - \delta(t - T_1) \xleftrightarrow{FS} \frac{e^{jk\omega_0 T_1} - e^{-jk\omega_0 T_1}}{T}
\]
所以:
\[
y(t) \xleftrightarrow{FS} b_k = \frac{e^{jk\omega_0 T_1} - e^{-jk\omega_0 T_1}}{T jk\omega_0} = \frac{2j \sin(k\omega_0 T_1)}{T jk\omega_0} = \frac{2T_1}{T} \text{Sa}(k\omega_0 T_1)
\]
\[
b_0 = \frac{2T_1}{T}
\]

\section{离散时间周期信号的傅里叶级数表示}

\subsection{成谐波关系的复指数信号的线性组合}
成谐波关系的复指数信号集合
\[
\phi_k[n] = e^{jk\omega_0 n}, \quad k = 0, \pm 1, \pm 2, \ldots
\]
集合中每个信号都是周期的,给定公共周期为 $N$,$\omega_0 = \frac{2\pi}{N}$,基波频率都是 $\omega_0$ 的整数倍。
\[
\left\{\phi_k[n]\right\} \text{ 只有 } N \text{ 个信号是不同的,也就是 } \phi_k[n] = \phi_{k+rN}[n]
\]

成谐波关系的复指数信号的线性组合
\[
x[n] = \sum_{k=<N>} a_k \phi_k[n] = \sum_{k=<N>} a_k e^{jk\omega_0 n}
\]
称为离散时间周期信号 $x[n]$(周期为 $N$)的离散时间傅里叶级数,$a_k$ 称为离散时间傅里叶级数系数。

\subsection{周期信号傅里叶级数系数的确定}
\[
x[n] = \sum_{k=<N>} a_k e^{jk\omega_0 n}
\]
可以写成含有 $N$ 个参数的 $N$ 个方程组,通过求解线性方程组可以计算得到 $a_k$。
可以通过类似连续时间傅里叶级数求解的方式得到解析解:
\[
\sum_{n=<N>} x[n] e^{-jr\omega_0 n} = \sum_{k=<N>} a_k \sum_{n=<N>} e^{j(k-r)\omega_0 n} = N a_r \Rightarrow a_r = \frac{1}{N} \sum_{n=<N>} x[n] e^{-jr\omega_0 n}
\]

\subsection{傅里叶级数的综合公式和分析公式}
综合公式:
\[
x[n] = \sum_{k=<N>} a_k e^{jk\omega_0 n} = \sum_{k=<N>} a_k e^{jk \frac{2\pi}{N} n}
\]
分析公式:
\[
a_k = \frac{1}{N} \sum_{n=<N>} x[n] e^{-jk\omega_0 n} = \frac{1}{N} \sum_{n=<N>} x[n] e^{-jk \frac{2\pi}{N} n}
\]
$a_k$:傅里叶级数系数、频谱系数,$x[n]$ 中每一个谐波分量大小的度量。
与连续时间信号不同的是,离散时间信号的 $a_k$ 是以 $N$ 为周期的级数,也即 $a_k = a_{k+N}$。

\subsection{例子}
例1:求信号 $x[n] = \sin(\omega_0 n)$ 的傅里叶级数系数。该信号当且仅当 $\omega_0 = \frac{2\pi}{N}$ 或 $\frac{2\pi M}{N}$ 才是周期的,取 $\omega_0 = \frac{2\pi}{N}$。
\[
\sin(\omega_0 n) = \frac{1}{2j} e^{j \frac{2\pi}{N} n} - \frac{1}{2j} e^{-j \frac{2\pi}{N} n}
\]
\[
a_1 = \frac{1}{2j}, \quad a_{-1} = -\frac{1}{2j}, \quad a_k = 0 \text{ 当 } k \neq 1 \text{ 或 } -1
\]
取 $N = 5$,$a_k$ 以 5 为周期。

取 $\omega_0 = \frac{2\pi M}{N}$,$N = 5$,$M = 3$,信号的基波周期为 5,
\[
\sin(\omega_0 n) = \frac{1}{2j} e^{j3 \frac{2\pi}{5} n} - \frac{1}{2j} e^{-j3 \frac{2\pi}{5} n}
\]
\[
a_3 = \frac{1}{2j}, \quad a_{-3} = -\frac{1}{2j}
\]
在一个周期内其余系数为零。如果取周期 $-2 \leq k \leq 2$,那么,
\[
a_2 = a_{-3} = -\frac{1}{2j}, \quad a_{-2} = a_3 = \frac{1}{2j}
\]

例2:求信号 $x[n] = 1 + \sin \frac{2\pi}{N} n + 3 \cos \frac{2\pi}{N} n + \cos \left( \frac{4\pi}{N} n + \frac{\pi}{2} \right)$ 的傅里叶级数系数。
\[
x[n] = 1 + (\frac{3}{2} + \frac{1}{2j}) e^{j \frac{2\pi}{N} n} + (\frac{3}{2} - \frac{1}{2j}) e^{-j \frac{2\pi}{N} n} + \frac{1}{2} e^{j \frac{\pi}{2}} e^{j \frac{4\pi}{N} n} + \frac{1}{2} e^{-j\frac{\pi}{2}}e^{-j \frac{4\pi}{N} n}
\]
\[
a_0 = 1, \quad a_1 = \frac{3}{2} + \frac{1}{2j}, \quad a_{-1} = \frac{3}{2} - \frac{1}{2j}, \quad a_2 = \frac{1}{2j}, \quad a_{-2} = -\frac{1}{2j}
\]
实序列:$a_k^* = a_{-k}$。
取 $N = 10$,即 $\omega_0 = \frac{\pi}{5}$。

例3:求周期为 $N$ 方波序列的傅里叶级数系数
\[
x[n] = \begin{cases}
1, & -N_1 + rN \leq n \leq N_1 + rN \\
0, & \text{其他}
\end{cases}, \quad N_1 = 5, \quad r = 0,1,2, \ldots
\]
\[
a_k = \frac{1}{N} \sum_{n=-N_1}^{N_1} x[n] e^{-jk\omega_0 n} = \frac{1}{N} \frac{\sin \left( \frac{\omega_0 k (2N_1 + 1)}{2} \right)}{\sin \left( \frac{\omega_0 k}{2} \right)}, \quad \omega_0 = \frac{2\pi}{N}
\]
当 $k \neq 0, \pm rN$ 时,
\[
a_k = \frac{2N_1+1}{N}, \quad \text{当 } k = 0, \pm rN \text{ 时}
\]
取 $N = 9$,
\[
\hat{x}[n] = \sum_{n=-M}^{M} a_n e^{jk \frac{2\pi}{N} n}, \quad \text{取} M = 1,2,3,4
\]

离散傅里叶级数不存在收敛问题,也不存在吉伯斯现象。
本质上就是一个 $N$ 个(时域)参数利用 $N$ 个(频域)傅里叶系数的等价描述;而连续时间信号因为有连续取值,理论上要求有无穷多个傅里叶系数来表示。

\subsection{离散时间傅里叶级数性质}
相乘性质\par
如果 $x[n] \xleftrightarrow{FS} a_k, \quad y[n] \xleftrightarrow{FS} b_k$ 同为周期为 $N$ 的序列,那么 $x[n] y[n]$ 也是周期为 $N$ 的序列,并有
\[
x[n] y[n] \xleftrightarrow{FS} d_k = \sum_{l=<N>} a_l b_{k-l}
\]
称为周期卷积,类似连续时间信号,时域相乘对应于频域卷积。\par

一次差分\par
如果 $x[n] \xleftrightarrow{FS} a_k$,那么 $x[n] - x[n-1]$ 也是周期为 $N$ 的序列,并有
\[
x[n] - x[n-1] \xleftrightarrow{FS} b_k = (1 - e^{-jk \frac{2\pi}{N}}) a_k
\]
$x[n-1] \xleftrightarrow{FS} e^{-jk \frac{2\pi}{N}} a_k$:时移不影响系数的幅度,但影响相位。
注意和连续时间的微分性质结果的差别。

\subsection{离散时间周期信号的帕塞瓦尔定理}
\[
\frac{1}{N} \sum_{n=<N>} |x[n]|^2 = \sum_{k=<N>} |a_k|^2
\]
一个周期信号的平均功率等于它的所有谐波分量的平均功率之和。

\subsection{例子}
例1:求如图所示周期 $N = 5$ 的序列 $x[n]$ 的傅里叶级数。观察 $x[n]$ 可以分解为 $x[n] = x_1[n] + x_2[n]$,$x_1[n]$ 为方波,$x_2[n]$ 为直流序列。
\begin{figure}[h]
    \centering
    \includegraphics[width=0.4\textwidth]{li1.png} % 替换为你的图片文件名
\end{figure}

\[
x_1[n] \xleftrightarrow{FS} b_k = \begin{cases}
\frac{1}{5} \frac{\sin \left( \frac{3\pi k}{5} \right)}{\sin \left( \frac{\pi k}{5} \right)}, & k \neq 0, \pm 5, \pm 10 \\
\frac{3}{5}, & k = 0, \pm 5, \pm 10
\end{cases}
\]
\[
x_2[n] \xleftrightarrow{FS} c_k = \begin{cases}
0, & k \neq 0, \pm 5, \pm 10 \\
1, & k = 0, \pm 5, \pm 10
\end{cases}
\]
\[
x[n] \xleftrightarrow{FS} a_k = b_k + c_k = \begin{cases}
\frac{1}{5} \frac{\sin \left( \frac{3\pi k}{5} \right)}{\sin \left( \frac{\pi k}{5} \right)}, & k \neq 0, \pm 5, \pm 10 \\
\frac{8}{5}, & k = 0, \pm 5, \pm 10
\end{cases}
\]

例2:序列 $x[n]$ 满足条件:\par
1. $x[n]$ 是周期的,$N = 6$;\par
2. $\sum_{n=0}^{5} x[n] = 2$;\par
3. $\sum_{n=2}^{7} -n x[n] = 1$;\par
4. 在满足上述3个条件的所有信号中,$x[n]$ 具有最小的功率。\par
求 $x[n]$。

由条件2):得到 $a_0 = \frac{1}{3}$;\par
由条件3):因为 $-n = e^{j(-\pi n)} = e^{j3(-2\pi/6)n}$,所以
\[
a_3 = \frac{1}{6} \sum_{n=2}^{7} e^{-j3 \frac{2\pi}{6} n} x[n] = 1 \Rightarrow a_3 = \frac{1}{6}
\]
根据帕塞瓦尔定理,平均功率 $P = \sum_{k=<N>} |a_k|^2$,因此基于条件4,$a_1 = a_2 = a_4 = a_5 = 0$。因此
\[
x[n] = a_0 + a_3 e^{j3 \frac{2\pi}{6} n} = \frac{1}{3} + \frac{1}{6} e^{-j3 \frac{2\pi}{6} n}
\]




\section{傅里叶级数与LTI系统的频域分析和频域滤波}

\subsection{傅里叶级数与LTI系统}

从系统函数到频率响应,对LTI系统
\[
H(s) = \int_{-\infty}^{+\infty} h(t) e^{-st} dt, \quad H(z) = \sum_{k=-\infty}^{+\infty} h[k] z^{-k}
\]
称为连续时间或离散时间系统的系统函数。特别地,当限制 $s = j\omega$(纯虚数)或 $z = e^{j\omega}$(长度为1),那么系统函数变成频率响应
\[
H(j\omega) = \int_{-\infty}^{+\infty} h(t) e^{-j\omega t} dt, \quad H(e^{j\omega}) = \sum_{n=-\infty}^{+\infty} h[n] e^{-j\omega n}
\]

几乎所有的连续时间和离散时间周期信号都可以利用傅里叶级数表示,
\[
x(t) = \sum_{k=-\infty}^{+\infty} a_k e^{jk\omega_0 t}, \quad x[n] = \sum_{k=0}^{N-1} a_k e^{jk\omega_0 n}
\]
结合系统的频率响应,可以得到系统的输出响应为
\[
y(t) = \sum_{k=-\infty}^{+\infty} a_k H(jk\omega_0) e^{jk\omega_0 t}, \quad y[n] = \sum_{k=0}^{N-1} a_k H(e^{jk\omega_0}) e^{jk\omega_0 n}
\]
上述输出响应也是傅里叶级数的形式,其系数为 $\{a_k H(jk\omega_0)\}$ 或 $\{a_k H(e^{jk\omega_0})\}$,LTI系统的作用就是通过乘以相应频率点上的频率响应值来逐个改变输入信号的每一个傅里叶系数;显然,输出也是以 $\omega_0$ 为基波频率的周期信号。

例1:LTI系统的单位冲激响应是 $h(t) = e^{-t}u(t)$,输入为周期信号 $x(t)$,其基波频率为 $2\pi$,表示成
\[
x(t) = \sum_{k=-3}^{3} a_k e^{jk2\pi t}, \quad \text{其中} a_0 = 1, \quad a_1 = a_{-1} = \frac{1}{4}, \quad a_2 = a_{-2} = \frac{1}{2}, \quad a_3 = a_{-3} = \frac{1}{3}
\]
求输出 $y(t)$。
求系统的频率响应:
\[
H(j\omega) = \int_{0}^{\infty} e^{-\tau} e^{-j\omega \tau} d\tau = \frac{1}{1 + j\omega}
\]
\[
y(t) = \sum_{k=-3}^{3} b_k e^{jk2\pi t} = \sum_{k=-3}^{3} a_k H(jk2\pi) e^{jk2\pi t}
\]
\[
b_k = a_k H(jk2\pi)
\]
如:
\[
b_1 = \frac{1}{4} \cdot \frac{1}{1 + j2\pi}, \quad b_{-1} = \frac{1}{4} \cdot \frac{1}{1 - j2\pi}
\]
$y(t)$ 为实信号,必有 $b_1^* = b_{-1}$。

例2:LTI系统的单位脉冲响应是 $h[n] = \alpha^n u[n]$,$-1 < \alpha < 1$,输入为周期信号 $x[n] = \cos \left( \frac{2\pi n}{N} \right) = \frac{1}{2} e^{j \frac{2\pi}{N} n} + \frac{1}{2} e^{-j \frac{2\pi}{N} n}$,求输出 $y[n]$。
系统的频率响应
\[
H(e^{j\omega}) = \sum_{n=0}^{\infty} \alpha^n e^{-j\omega n} = \frac{1}{1 - \alpha e^{-j\omega}}
\]
\[
y[n] = \frac{1}{2} \cdot \frac{1}{1 - \alpha e^{-j \frac{2\pi}{N}}} e^{j \frac{2\pi}{N} n} + \frac{1}{2} \cdot \frac{1}{1 - \alpha e^{j \frac{2\pi}{N}}} e^{-j \frac{2\pi}{N} n}
\]
令
\[
\frac{1}{1 - \alpha e^{-j \frac{2\pi}{N}}} = r e^{j\theta}
\]
\[
y[n] = r \cos \left( \frac{2\pi}{N} n + \theta \right)
\]

\subsection{滤波(Filter)}

改变一个信号中各频率分量的相对大小、将信号中特定波段频率滤除的操作称为滤波。
\begin{itemize}
    \item 频率成形滤波器:用于改变频谱的形状,如调解音质
    \item 频率选择性滤波器:用于通过某些频率,同时显著地消除或者衰减另一些频率,如选择信道
\end{itemize}

\subsubsection{频率成形滤波器}
例1:微分滤波器 $y(t) = \frac{d x(t)}{dt}$,其冲激响应为 $h(t) = \delta'(t)$
\[
\text{频率响应为} H(j\omega) = j\omega = \int_{-\infty}^{+\infty} \delta'(t) e^{-j\omega t} dt = - \frac{d e^{-j\omega t}}{dt} \Bigg|_{t=0}
\]
(冲激偶的抽样特性 $\int_{-\infty}^{+\infty} \delta'(t) f(t) dt = -f'(0)$)
或取 $x(t) = e^{j\omega t}$,那么 $y(t) = j\omega e^{j\omega t}$,因此 $H(j\omega) = j\omega$。
频率越高,放大部分越大,也即突出信号的快速变化部分(图像的边缘检测)。

例2:均值滤波器 $y[n] = \frac{1}{2} \left( x[n] + x[n-1] \right)$
\[
\text{当} x[n] = e^{j\omega n}, \quad y[n] = \frac{1}{2} \left( e^{j\omega n} + e^{j\omega (n-1)} \right) = \frac{1}{2} e^{j\omega n} \left( 1 + e^{-j\omega} \right)
\]
因此
\[
H(e^{j\omega}) = \frac{1}{2} \left( 1 + e^{-j\omega} \right) = e^{-j\frac{\omega}{2}} \cos \left( \frac{\omega}{2} \right)
\]
或者:
\[
h[n] = \frac{1}{2} \left( \delta[n] + \delta[n-1] \right)
\]
结合 $\delta[n]$ 的频响(为1)和(傅里叶变换)时移特性得到 $H(e^{j\omega})$。\par
离散复指数在0附近为低频,$\pm \pi$ 附近为高频,因此均值滤波器抑制高频部分。
\begin{itemize}
    \item 如输入 $x[n] = K = K e^{-j0n}$ 为常数,$y[n] = H(e^{j0}) K e^{-j0n} = K$
    \item 如输入 $x[n] = K(-1)^n = K e^{-j\pi n}$ 为高频振荡序列,$y[n] = H(e^{j\pi}) K e^{-j\pi n} = 0$
\end{itemize}

\begin{figure}[H]
    \centering
    \includegraphics[width=0.4\textwidth]{lv2.png} % 替换为你的图片文件名
\end{figure}

\subsubsection{频率选择性滤波器}
用于选取某些频带范围的信号,同时去除其它频带范围的信号的滤波器。
\begin{itemize}
    \item 低通滤波器:通过低频,阻止高频
    \item 高通滤波器:通过高频,阻止低频
    \item 带通滤波器:通过某一频带范围
\end{itemize}
理想连续时间和离散时间的低通、高通、带通滤波器示意图。注意频谱的对称性及离散时间滤波器的周期性。

\begin{figure}[H]
    \centering
    \includegraphics[width=0.6\textwidth]{lv1.png} % 替换为你的图片文件名
\end{figure}

\subsection{用微分方程描述的连续时间滤波器举例}
简单 $RC$ 低通滤波器
\[
RC \frac{dV_c(t)}{dt} + V_c(t) = V_s(t)
\]
假定系统是初始松弛的(LTI系统),设 $V_s(t) = e^{j\omega t}$,$V_c(t) = H(j\omega) e^{j\omega t}$,代入微分方程得到
\[
H(j\omega) = \frac{1}{1 + j\omega RC}
\]
是一个非理想的低通滤波器。

\begin{figure}[H]
    \centering
    \includegraphics[width=0.4\textwidth]{lv3.png} % 替换为你的图片文件名
\end{figure}

简单 $RC$ 高通滤波器,将电阻的电压 $V_r(t)$ 作为系统输出
\[
RC \frac{dV_r(t)}{dt} + V_r(t) = RC \frac{dV_s(t)}{dt}
\]
得到
\[
G(j\omega) = \frac{j\omega RC}{1 + j\omega RC}
\]
是一个非理想的高通滤波器。

\begin{figure}[H]
    \centering
    \includegraphics[width=0.4\textwidth]{lv4.png} % 替换为你的图片文件名
\end{figure}

\subsection{用差分方程描述的离散时间滤波器举例}
一阶递归离散时间滤波器(IIR)
\[
y[n] - ay[n-1] = x[n]
\]
设 $x[n] = e^{j\omega n}$,$y[n] = H(e^{j\omega}) e^{j\omega n}$ 代入求得
\[
H(e^{j\omega}) = \frac{1}{1 - ae^{-j\omega}}
\]
当 $-1 < a < 1$ 时,系统稳定。
\begin{itemize}
    \item $0 < a < 1$:非理想低通滤波器
    \item $-1 < a < 0$:非理想高通滤波器
\end{itemize}

\begin{figure}[H]
    \centering
    \includegraphics[width=0.4\textwidth]{lv5.png} % 替换为你的图片文件名
\end{figure}

非递归离散时间滤波器:一个有限脉冲响应非递归差分方程一般形式(FIR)
\[
y[n] = \sum_{k=-N}^{M} b_k x[n-k]
\]
输出为输入的加权平均,起到平滑作用(权重为正)
\[
h[n] = \sum_{k=-N}^{M} b_k \delta[n-k] \Rightarrow H(e^{j\omega}) = \sum_{k=-N}^{M} b_k e^{-j\omega k}
\]
取 $b_k = \frac{1}{M+N+1}$,
\begin{itemize}
    \item $N = 0, M = 1$
    \item $N = M = 1$
    \item $N = M = 16$
    \item $N = M = 32$
\end{itemize}
项数越多,则低频通带相对越为陡峭。当 $N > 0$,系统为非因果的。

\begin{figure}[H]
    \centering
    \includegraphics[width=0.4\textwidth]{lv6.png} % 替换为你的图片文件名
\end{figure}

非递归离散时间滤波器:当权重为负,则可能为高通滤波器
\[
y[n] = \frac{1}{2} \left( x[n] - x[n-1] \right)
\]
\[
h[n] = \frac{1}{2} \left( \delta[n] - \delta[n-1] \right) \Rightarrow H(e^{j\omega}) = \frac{1}{2} \left( 1 - e^{-j\omega} \right) = je^{j\frac{\omega}{2}} \sin \left( \frac{\omega}{2} \right)
\]

\begin{figure}[H]
    \centering
    \includegraphics[width=0.4\textwidth]{lv7.png} % 替换为你的图片文件名
\end{figure}














\end{document}