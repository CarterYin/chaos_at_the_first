% 若编译失败,且生成 .synctex(busy) 辅助文件,可能有两个原因:
% 1. 需要插入的图片不存在:Ctrl + F 搜索 'figure' 将这些代码注释/删除掉即可
% 2. 路径/文件名含中文或空格:更改路径/文件名即可

% ------------------------------------------------------------- %
% >> ------------------ 文章宏包及相关设置 ------------------ << %
% 设定文章类型与编码格式
    \documentclass[UTF8]{report}		

% 本 .tex 专属的宏定义
    \def\V{\ \mathrm{V}}
    \def\mV{\ \mathrm{mV}}
    \def\kV{\ \mathrm{KV}}
    \def\KV{\ \mathrm{KV}}
    \def\MV{\ \mathrm{MV}}
    \def\A{\ \mathrm{A}}
    \def\mA{\ \mathrm{mA}}
    \def\kA{\ \mathrm{KA}}
    \def\KA{\ \mathrm{KA}}
    \def\MA{\ \mathrm{MA}}
    \def\O{\ \Omega}
    \def\mO{\ \Omega}
    \def\kO{\ \mathrm{K}\Omega}
    \def\KO{\ \mathrm{K}\Omega}
    \def\MO{\ \mathrm{M}\Omega}
    \def\Hz{\ \mathrm{Hz}}

% 自定义宏定义
    \def\N{\mathbb{N}}
    \def\F{\mathbb{F}}
    \def\Z{\mathbb{Z}}
    \def\Q{\mathbb{Q}}
    \def\R{\mathbb{R}}
    \def\C{\mathbb{C}}
    \def\T{\mathbb{T}}
    \def\S{\mathbb{S}}
    \def\A{\mathbb{A}}
    \def\I{\mathscr{I}}
    \def\Im{\mathrm{Im\,}}
    \def\Re{\mathrm{Re\,}}
    \def\d{\mathrm{d}}
    \def\p{\partial}
% 导入基本宏包
    \usepackage[UTF8]{ctex}     % 设置文档为中文语言
    \usepackage[colorlinks, linkcolor=blue, anchorcolor=blue, citecolor=blue, urlcolor=blue]{hyperref}  % 宏包:自动生成超链接 (此宏包与标题中的数学环境冲突)
    % \usepackage{docmute}    % 宏包:子文件导入时自动去除导言区,用于主/子文件的写作方式,\include{./51单片机笔记}即可。注:启用此宏包会导致.tex文件capacity受限。
    \usepackage{amsmath}    % 宏包:数学公式
    \usepackage{mathrsfs}   % 宏包:提供更多数学符号
    \usepackage{amssymb}    % 宏包:提供更多数学符号
    \usepackage{pifont}     % 宏包:提供了特殊符号和字体
    \usepackage{extarrows}  % 宏包:更多箭头符号


% 文章页面margin设置
    \usepackage[a4paper]{geometry}
        \geometry{top=1in}
        \geometry{bottom=1in}
        \geometry{left=0.75in}
        \geometry{right=0.75in}   % 设置上下左右页边距
        \geometry{marginparwidth=1.75cm}    % 设置边注距离(注释、标记等)

% 配置数学环境
    \usepackage{amsthm} % 宏包:数学环境配置
    % theorem-line 环境自定义
        \newtheoremstyle{MyLineTheoremStyle}% <name>
            {11pt}% <space above>
            {11pt}% <space below>
            {}% <body font> 使用默认正文字体
            {}% <indent amount>
            {\bfseries}% <theorem head font> 设置标题项为加粗
            {:}% <punctuation after theorem head>
            {.5em}% <space after theorem head>
            {\textbf{#1}\thmnumber{#2}\ \ (\,\textbf{#3}\,)}% 设置标题内容顺序
        \theoremstyle{MyLineTheoremStyle} % 应用自定义的定理样式
        \newtheorem{LineTheorem}{Theorem.\,}
    % theorem-block 环境自定义
        \newtheoremstyle{MyBlockTheoremStyle}% <name>
            {11pt}% <space above>
            {11pt}% <space below>
            {}% <body font> 使用默认正文字体
            {}% <indent amount>
            {\bfseries}% <theorem head font> 设置标题项为加粗
            {:\\ \indent}% <punctuation after theorem head>
            {.5em}% <space after theorem head>
            {\textbf{#1}\thmnumber{#2}\ \ (\,\textbf{#3}\,)}% 设置标题内容顺序
        \theoremstyle{MyBlockTheoremStyle} % 应用自定义的定理样式
        \newtheorem{BlockTheorem}[LineTheorem]{Theorem.\,} % 使用 LineTheorem 的计数器
    % definition 环境自定义
        \newtheoremstyle{MySubsubsectionStyle}% <name>
            {11pt}% <space above>
            {11pt}% <space below>
            {}% <body font> 使用默认正文字体
            {}% <indent amount>
            {\bfseries}% <theorem head font> 设置标题项为加粗
            {:\\ \indent}% <punctuation after theorem head>
            {0pt}% <space after theorem head>
            {\textbf{#3}}% 设置标题内容顺序
        \theoremstyle{MySubsubsectionStyle} % 应用自定义的定理样式
        \newtheorem{definition}{}

%宏包:有色文本框(proof环境)及其设置
    \usepackage[dvipsnames,svgnames]{xcolor}    %设置插入的文本框颜色
    \usepackage[strict]{changepage}     % 提供一个 adjustwidth 环境
    \usepackage{framed}     % 实现方框效果
        \definecolor{graybox_color}{rgb}{0.95,0.95,0.96} % 文本框颜色。修改此行中的 rgb 数值即可改变方框纹颜色,具体颜色的rgb数值可以在网站https://colordrop.io/ 中获得。(截止目前的尝试还没有成功过,感觉单位不一样)(找到喜欢的颜色,点击下方的小眼睛,找到rgb值,复制修改即可)
        \newenvironment{graybox}{%
        \def\FrameCommand{%
        \hspace{1pt}%
        {\color{gray}\small \vrule width 2pt}%
        {\color{graybox_color}\vrule width 4pt}%
        \colorbox{graybox_color}%
        }%
        \MakeFramed{\advance\hsize-\width\FrameRestore}%
        \noindent\hspace{-4.55pt}% disable indenting first paragraph
        \begin{adjustwidth}{}{7pt}%
        \vspace{2pt}\vspace{2pt}%
        }
        {%
        \vspace{2pt}\end{adjustwidth}\endMakeFramed%
        }

% 外源代码插入设置
    % matlab 代码插入设置
    %\usepackage{matlab-prettifier}
    %    \lstset{
    %        style=Matlab-editor,  % 继承matlab代码颜色等
    %    }
    %\usepackage[most]{tcolorbox} % 引入tcolorbox包 
    %\usepackage{listings} % 引入listings包
    %    \tcbuselibrary{listings, skins, breakable}
    %    \newfontfamily\codefont{Consolas} % 定义需要的 codefont 字体
    %    \lstdefinestyle{matlabstyle}{
    %        language=Matlab,
    %        basicstyle=\small\ttfamily\codefont,    % ttfamily 确保等宽 
    %        breakatwhitespace=false,
    %        breaklines=true,
    %        captionpos=b,
    %        keepspaces=true,
    %        numbers=left,
    %        numbersep=15pt,
    %        showspaces=false,
    %        showstringspaces=false,
    %        showtabs=false,
    %        tabsize=2
    %    }
    %    \newtcblisting{matlablisting}{
    %        arc=2pt,        % 圆角半径
    %        top=-5pt,
    %        bottom=-5pt,
    %        left=1mm,
    %        listing only,
    %        listing style=matlabstyle,
    %        breakable,
    %        colback=white   % 选一个合适的颜色
    %    }

% table 支持
    %\usepackage{booktabs}   % 宏包:三线表
    %\usepackage{tabularray} % 宏包:表格排版
    %\usepackage{longtable}  % 宏包:长表格
    %\usepackage[longtable]{multirow} % 宏包:multi 行列

% figure 设置
    \usepackage{graphicx}   % 支持 jpg, png, eps, pdf 图片 
    \usepackage{float}      % 支持 H 选项
    %\usepackage{svg}        % 支持 svg 图片
    %\usepackage{subcaption} % 支持子图
    %    \svgsetup{
            % 指向 inkscape.exe 的路径
    %        inkscapeexe = C:/aa_MySame/inkscape/bin/inkscape.exe, 
            % 一定程度上修复导入后图片文字溢出几何图形的问题
    %        inkscapelatex = false                 
    %    }

% 图表进阶设置
    %\usepackage{caption}    % 图注、表注
    %    \captionsetup[figure]{name=图}  
    %    \captionsetup[table]{name=表}
    %    \captionsetup{labelfont=bf, font=small}
    %\usepackage{float}     % 图表位置浮动设置 

% 圆圈序号自定义
    \newcommand*\circled[1]{\tikz[baseline=(char.base)]{\node[shape=circle,draw,inner sep=0.8pt, line width = 0.03em] (char) {\small \bfseries #1};}}   % TikZ solution

% 列表设置
    \usepackage{enumitem}   % 宏包:列表环境设置
        \setlist[enumerate]{
            label=\bfseries(\arabic*) ,   % 设置序号样式为加粗的 (1) (2) (3)
            ref=\arabic*, % 如果需要引用列表项,这将决定引用格式(这里仍然使用数字)
            itemsep=0pt, parsep=0pt, topsep=0pt, partopsep=0pt, leftmargin=3.5em} 
        \setlist[itemize]{itemsep=0pt, parsep=0pt, topsep=0pt, partopsep=0pt, leftmargin=3.5em}
        \newlist{circledenum}{enumerate}{1} % 创建一个新的枚举环境  
        \setlist[circledenum,1]{  
            label=\protect\circled{\arabic*}, % 使用 \arabic* 来获取当前枚举计数器的值,并用 \circled 包装它  
            ref=\arabic*, % 如果需要引用列表项,这将决定引用格式(这里仍然使用数字)
            itemsep=0pt, parsep=0pt, topsep=0pt, partopsep=0pt, leftmargin=3.5em
        }  
    

% 文章默认字体设置
    \usepackage{fontspec}   % 宏包:字体设置
        \setmainfont{SimSun}    % 设置中文字体为宋体字体
        \setCJKmainfont[AutoFakeBold=3]{SimSun} % 设置加粗字体为 SimSun 族,AutoFakeBold 可以调整字体粗细
        \setmainfont{Times New Roman} % 设置英文字体为Times New Roman

% 其它设置
    % 脚注设置
    \renewcommand\thefootnote{\ding{\numexpr171+\value{footnote}}}
    % 参考文献引用设置
        \bibliographystyle{unsrt}   % 设置参考文献引用格式为unsrt
        \newcommand{\upcite}[1]{\textsuperscript{\cite{#1}}}     % 自定义上角标式引用
    % 文章序言设置
        \newcommand{\cnabstractname}{序言}
        \newenvironment{cnabstract}{%
            \par\Large
            \noindent\mbox{}\hfill{\bfseries \cnabstractname}\hfill\mbox{}\par
            \vskip 2.5ex
            }{\par\vskip 2.5ex}
% 各级标题自定义设置
    \usepackage{titlesec}   
    % chapter
        \titleformat{\chapter}[hang]{\normalfont\Large\bfseries\centering}{Homework \thechapter }{10pt}{}
        \titlespacing*{\chapter}{0pt}{-30pt}{10pt} % 控制上方空白的大小
    % section
        \titleformat{\section}[hang]{\normalfont\large\bfseries}{\thesection}{8pt}{}
    % subsection
        %\titleformat{\subsubsection}[hang]{\normalfont\bfseries}{}{8pt}{}
    % subsubsection
        %\titleformat{\subsubsection}[hang]{\normalfont\bfseries}{}{8pt}{}

% >> ------------------ 文章宏包及相关设置 ------------------ << %
% ------------------------------------------------------------- %



% ----------------------------------------------------------- %
% >> --------------------- 文章信息区 --------------------- << %
% 页眉页脚设置

\usepackage{fancyhdr}   %宏包:页眉页脚设置
    \pagestyle{fancy}
    \fancyhf{}
    \cfoot{\thepage}
    \renewcommand\headrulewidth{1pt}
    \renewcommand\footrulewidth{0pt}
    \chead{概率论与数理统计作业,\ 尹超,\ 2023K8009926003}
    \lhead{Homework \thechapter}
    \rhead{yinchao23@mails.ucas.ac.cn}

%文档信息设置
\title{信号与系统作业\\ Homework}
\author{尹超\\ \footnotesize 中国科学院大学,北京 100049\\ Carter Yin \\ \footnotesize University of Chinese Academy of Sciences, Beijing 100049, China}
\date{\footnotesize 2024.8 -- 2025.1}
% >> --------------------- 文章信息区 --------------------- << %
% ----------------------------------------------------------- %

% 开始编辑文章

\begin{document}
\zihao{5}           % 设置全文字号大小

% --------------------------------------------------------------- %
% >> --------------------- 封面序言与目录 --------------------- << %
% 封面
    \maketitle\newpage  
    \pagenumbering{Roman} % 页码为大写罗马数字
    \thispagestyle{fancy}   % 显示页码、页眉等

% 序言
    \begin{cnabstract}\normalsize 
        本文为笔者概率论与数理统计的作业。\par
        望老师批评指正。
    \end{cnabstract}
    \addcontentsline{toc}{chapter}{序言} % 手动添加为目录

% 不换页目录
    \setcounter{tocdepth}{0}
    \noindent\rule{\textwidth}{0.1em}   % 分割线
    \noindent\begin{minipage}{\textwidth}\centering 
        \vspace{1cm}
        \tableofcontents\thispagestyle{fancy}   % 显示页码、页眉等   
    \end{minipage}  
    \addcontentsline{toc}{chapter}{目录} % 手动添加为目录

% 收尾工作
    \newpage    
    \pagenumbering{arabic} 

% >> --------------------- 封面序言与目录 --------------------- << %
% --------------------------------------------------------------- %



\chapter{第一章习题汇总}\thispagestyle{fancy}

\section{1.1}

1. 有 5 个事件,\(A_1, \ldots, A_5\)。用它们表示以下的事件:
(a) \(B_1 = \{A_1, \ldots, A_5 \text{ 中至多发生 2 个}\}\)
(b) \(B_2 = \{A_1, \ldots, A_5 \text{ 中至少发生 2 个}\}\)

\subsection*{解答}

(a) \(B_1 = \{A_1, \ldots, A_5 \text{ 中至多发生 2 个}\}\)

表示 \(A_1, \ldots, A_5\) 中至多发生 2 个事件的集合。可以表示为:
\[
B_1 = \bigcup_{i=0}^{2} \left\{ \sum_{j=1}^{5} I(A_j) = i \right\}
\]
其中,\(I(A_j)\) 是事件 \(A_j\) 的指示函数,当 \(A_j\) 发生时 \(I(A_j) = 1\),否则 \(I(A_j) = 0\)。

具体展开为:
\[
B_1 = \left\{ \sum_{j=1}^{5} I(A_j) = 0 \right\} \cup \left\{ \sum_{j=1}^{5} I(A_j) = 1 \right\} \cup \left\{ \sum_{j=1}^{5} I(A_j) = 2 \right\}
\]

(b) \(B_2 = \{A_1, \ldots, A_5 \text{ 中至少发生 2 个}\}\)

表示 \(A_1, \ldots, A_5\) 中至少发生 2 个事件的集合。可以表示为:
\[
B_2 = \bigcup_{i=2}^{5} \left\{ \sum_{j=1}^{5} I(A_j) = i \right\}
\]

具体展开为:
\[
B_2 = \left\{ \sum_{j=1}^{5} I(A_j) = 2 \right\} \cup \left\{ \sum_{j=1}^{5} I(A_j) = 3 \right\} \cup \left\{ \sum_{j=1}^{5} I(A_j) = 4 \right\} \cup \left\{ \sum_{j=1}^{5} I(A_j) = 5 \right\}
\]

\section{1.2}

2. 证明:若 \(A, B\) 为两事件,则\par
(a) \(A + B = A + (B - A)\),右边两事件互斥;\par
(b) \(A + B = (A - B) + (B - A) + AB\),右边三事件互斥。

\subsection*{解答}

(a) 证明 \(A + B = A + (B - A)\):

\[
A + B = A \cup B
\]
\[
B - A = B \cap A^c
\]
\[
A + (B - A) = A \cup (B \cap A^c)
\]
由于 \(A\) 和 \(B - A\) 互斥,因此 \(A + B = A + (B - A)\)。

(b) 证明 \(A + B = (A - B) + (B - A) + AB\):

\[
A - B = A \cap B^c
\]
\[
B - A = B \cap A^c
\]
\[
AB = A \cap B
\]
\[
A + B = (A \cap B^c) + (B \cap A^c) + (A \cap B)
\]
由于 \(A - B\)、\(B - A\) 和 \(AB\) 互斥,因此 \(A + B = (A - B) + (B - A) + AB\)。

\section{1.4}

4. 把 \(n\) 个任意事件 \(A_1, \ldots, A_n\) 之和表为 \(n\) 个互斥事件之和。

\subsection*{解答}

设 \(B_i\) 表示事件 \(A_i\) 发生且 \(A_1, \ldots, A_{i-1}\) 不发生,则:
\[
B_i = A_i \cap \left( \bigcap_{j=1}^{i-1} A_j^c \right)
\]
则 \(A_1 + A_2 + \cdots + A_n\) 可以表示为:
\[
A_1 + A_2 + \cdots + A_n = B_1 + B_2 + \cdots + B_n
\]
其中 \(B_i\) 互斥。

\section{1.10}

10. 证明:若 \(A, C\) 独立,\(B, C\) 也独立,又 \(A, B\) 互斥,则 \(A + B\) 与 \(C\) 独立。\par
更一般地,若 \(A, C\) 独立,\(B, C\) 独立,\(AB, C\) 也独立,则 \(A + B\) 与 \(C\) 独立。说明:上一结论是本结论的特例。

\subsection*{解答}

首先证明第一部分:

由于 \(A, C\) 独立,\(B, C\) 也独立,因此:
\[
P(A \cap C) = P(A)P(C)
\]
\[
P(B \cap C) = P(B)P(C)
\]

由于 \(A, B\) 互斥,即 \(A \cap B = \emptyset\),因此:
\[
P((A + B) \cap C) = P((A \cup B) \cap C) = P(A \cap C) + P(B \cap C)
\]
\[
= P(A)P(C) + P(B)P(C) = (P(A) + P(B))P(C) = P(A + B)P(C)
\]

因此,\(A + B\) 与 \(C\) 独立。

接下来证明更一般的情况:

假设 \(A, C\) 独立,\(B, C\) 独立,且 \(AB, C\) 也独立。我们需要证明 \(A + B\) 与 \(C\) 独立。

首先,由于 \(A, C\) 独立,因此:
\[
P(A \cap C) = P(A)P(C)
\]

由于 \(AB, C\) 独立,因此:
\[
P((A \cap B) \cap C) = P(A \cap B)P(C)
\]

上两式相减得:
\[
P(A \cap C) - P((A \cap B) \cap C) = P(A)P(C) - P(A \cap B)P(C)
\]
\[
P(A \cap C) - P(A \cap B \cap C) = P(C)(P(A) - P(A \cap B))
\]
\[
P(A \cap C) - P(A \cap B \cap C) = P(C)P(A - AB)
\]\par
左 $= P(C(A - AB)) = P(C(A - B))$\par
右 $= P(C)P(A - B)$\par
$P(C(A - B)) = P(C)P(A - B)$\par
故 \(C\) 与 \(A - B\) 独立。

同理可证 \(C\) 与 \(B - A\) 独立。

又 \(C\) 与 \(AB\) 独立。

因此,\(A + B\) 与 \(C\) 独立。

综上所述,证明了若 \(A, C\) 独立,\(B, C\) 独立,且 \(AB, C\) 独立,则 \(A + B\) 与 \(C\) 独立。前一个结论是这个更一般结论的特例,因为在前一个结论中,\(A\) 和 \(B\) 互斥,因此 \(A \cap B = \emptyset\),自然满足 \(AB, C\) 独立的条件。

\section{1.17}

17. 一个秘书打好 4 封信和相应的 4 个信封。但她将这 4 封信随机地放入这 4 个信封中,问“每封信都放得不对位”这事件的概率是多少?

\subsection*{解答}

这是一个错排问题。错排的概率公式为:
\[
D_n = n! \sum_{i=0}^{n} \frac{(-1)^i}{i!}
\]
对于 \(n = 4\):
\[
D_4 = 4! \left( \frac{(-1)^0}{0!} + \frac{(-1)^1}{1!} + \frac{(-1)^2}{2!} + \frac{(-1)^3}{3!} + \frac{(-1)^4}{4!} \right)
\]
\[
= 24 \left( 1 - 1 + \frac{1}{2} - \frac{1}{6} + \frac{1}{24} \right)
\]
\[
= 24 \left( \frac{12}{24} - \frac{4}{24} + \frac{1}{24} \right)
\]
\[
= 24 \times \frac{9}{24} = 9
\]
错排的概率为:
\[
P(D_4) = \frac{D_4}{4!} = \frac{9}{24} = \frac{3}{8}
\]

\section{1.18}

18. 一盒内有 8 张空白券,2 张奖券,有甲、乙、丙三人按这个次序和以下的规则,各从此盒中随机抽出一张。规则如下:每人抽出后,所抽那张不放回但补入两张非同类券(即:如抽出奖券,则放回 2 张空白券,等等)。问甲、乙、丙中奖的概率各有多大?

\subsection*{解答}

设 \(A\) 表示抽到奖券的事件,\(B\) 表示抽到空白券的事件。

甲中奖的概率:
\[
P(A_1) = \frac{2}{10} = \frac{1}{5}
\]

乙中奖的概率:
\[
P(A_2) = P(A_1) \times \frac{1}{11} + (1-P(A_1)) \times \frac{4}{11} = \frac{17}{55}
\]

丙中奖的概率:
\[
P(A_3) = P(A_1) \times \frac{1}{11} \times \frac{0}{12} + P(A_1) \times \frac{10}{11} \times \frac{3}{12} + (1 - P(A_1)) \times \frac{4}{11} \times \frac{3}{12} + (1 - P(A_1)) \times \frac{7}{11} \times \frac{6}{12} = \frac{41}{110}
\]

\section{1.19}

19. 某作家的全集共 \(p\) 卷,现买来 \(n\) 套(共 \(np\) 本),随机地分成 \(n\) 堆,每堆 \(p\) 本,问“每堆都组成整套全集”这事件的概率为多少。

\subsection*{解答}

每堆都组成整套全集的概率为:
\[
P = \frac{(n!)^p}{\frac{(np)!}{(p!)^n}} = \frac{(n!)^p(p!)^n}{(np)!} 
\]

\section{1.25}

25. 把 8 个可以分辨的球随机地放入 7 个可以分辨的盒子中,问“其中有两个盒子得 2 球,一个盒子得 3 球,一个盒子得 1 球”这事件的概率是多少?

\subsection*{解答}

概率为:
\[
P = \frac{\binom{8}{2, 2, 3, 1}\binom{7}{2, 1, 1, 3}}{7^8} = \frac{(5!)^2}{7^6}
\]

\section{1.26}

26. 设男女两性人口之比为 51:49。又设男人色盲率为 0.02,女人色盲率为 0.0025。现随机抽到一个人为色盲,问“该人为男人”的概率是多少?

\subsection*{解答}

设 \(M\) 表示男人,\(W\) 表示女人,\(C\) 表示色盲。

\[
P(M) = 0.51, \quad P(W) = 0.49
\]
\[
P(C \mid M) = 0.02, \quad P(C \mid W) = 0.0025
\]

根据贝叶斯定理:
\[
P(M \mid C) = \frac{P(C \mid M)P(M)}{P(C)}
\]

其中:
\[
P(C) = P(C \mid M)P(M) + P(C \mid W)P(W)
\]
\[
= 0.02 \times 0.51 + 0.0025 \times 0.49 = 0.0102 + 0.001225 = 0.011425
\]

因此:
\[
P(M \mid C) = \frac{0.02 \times 0.51}{0.011425} = \frac{0.0102}{0.011425} \approx 0.892
\]

\section{1.28}

28. 投掷 10 粒均匀骰子,记事件 \(A = \{\text{至少有 2 粒骰子出 1 点}\}\),\(B = \{\text{至少有 1 粒骰子出 1 点}\}\)。求条件概率 \(P(A \mid B)\)。\par
这道题可不可以这样算:既然已知至少掷出一个 1 点,不妨(因各骰子地位对称)就设第一粒骰子掷出 1 点。因此所求的条件概率为:掷 9 粒骰子至少出现一个 1 点的概率,即:
\[
P(A \mid B) = 1 - \left( \frac{5}{6} \right)^9
\]
why?
\subsection*{解答}

\[
P_(B) = 1 - \left( \frac{5}{6} \right)^{10}
\]
\[
P(A \cap B) = 1 - \left( \frac{5}{6} \right)^{9} \times \frac{1}{6} \times 10 - \left( \frac{5}{6} \right)^{10}
\]
\[
P(A \mid B) = \frac{P(A \cap B)}{P(B)} = \frac{1 - \left( \frac{5}{6} \right)^{9} \times \frac{1}{6} \times 10 - \left( \frac{5}{6} \right)^{10}}{1 - \left( \frac{5}{6} \right)^{10}}
\]\par

不能,因为A,B不独立。

\section{1.29}

29. 假定某种病菌在全人口的带菌率为 0.1,又在检测时,带菌者呈阳、阴性反应的概率为 0.95 和 0.05,而不带菌者呈阳、阴性反应的概率则为 0.01 和 0.99。今某人独立地检测三次,发现 2 次呈阳性反应,1 次呈阴性反应。问:“该人为带菌者”的概率是多少?

\subsection*{解答一}

设 \(B\) 表示带菌者,\(N\) 表示不带菌者,\(Y\) 表示阳性反应,\(N\) 表示阴性反应。

\[
P(B) = 0.1, \quad P(N) = 0.9
\]
\[
P(Y \mid B) = 0.95, \quad P(N \mid B) = 0.05
\]
\[
P(Y \mid N) = 0.01, \quad P(N \mid N) = 0.99
\]

根据贝叶斯定理:
\[
P(B \mid YYN) = \frac{P(YYN \mid B)P(B)}{P(YYN)}
\]

其中:
\[
P(YYN) = P(YYN \mid B)P(B) + P(YYN \mid N)P(N)
\]

\[
P(YYN \mid B) = \binom{3}{2} (0.95)^2 (0.05) = 3 \times 0.9025 \times 0.05 = 0.135375
\]
\[
P(YYN \mid N) = \binom{3}{2} (0.01)^2 (0.99) = 3 \times 0.0001 \times 0.99 = 0.000297
\]

\[
P(YYN) = 0.135375 \times 0.1 + 0.000297 \times 0.9 = 0.0135375 + 0.0002673 = 0.0138048
\]

因此:
\[
P(B \mid YYN) = \frac{0.135375 \times 0.1}{0.0138048} = \frac{0.0135375}{0.0138048} \approx 0.9806
\]

\subsection*{解答二}

设A:某人为带菌者,B:某人检测两次呈阳性,一次呈阴性。
\[
P(A \mid B) = \frac{P(AB)}{P(B)} = \frac{P(B \mid A)P(A)}{P(B)} = \frac{P(B \mid A)P(A)}{P(B \mid A)P(A) + P(B \mid A^c)P(A^c)}
\]


\section{补充题目}

Suppose someone (say Snyder) encodes the English alphabet by the following codebook, which is unknown to us.
\begin{center}
\hspace*{-1.5cm}
\begin{tabular}{ccccccccccccccccccccccccccc}
Letters   & A & B & C & D & E & F & G & H & I & J & K & L & M & N & O & P & Q & R & S & T & U & V & W & X & Y & Z \\
Encrypted & F & D & P & R & A & V & Q & K & W & C & Z & N & I & T & B & J & S & X & E & U & L & Y & O & H & M & G 
\end{tabular}
\end{center}

\begin{enumerate}
    \item Now we collect some English texts which were sent by Snyder based on the above codebook. Please describe a statistical decoding scheme and justify your solution.
    \item If we use the above method to encrypt English letters, how many different code-books are there?
\end{enumerate}

\subsection*{解答}

\subsubsection*{[1] 统计解码方案}

为了解码由 Snyder 使用上述代码本加密的英文文本,我们可以使用统计解码方案。具体步骤如下:

\begin{enumerate}
    \item \textbf{收集加密文本}:收集尽可能多的由 Snyder 发送的加密英文文本。
    \item \textbf{统计字母频率}:统计加密文本中每个字母出现的频率。假设加密文本中字母 \(E\) 出现的频率最高,那么我们可以推测该字母对应的明文字母也是频率最高的字母,即 \(E\)。
    \item \textbf{构建频率表}:根据英文语言的字母频率表(例如,E, T, A, O, I, N, S, H, R, D, L, C, U, M, W, F, G, Y, P, B, V, K, J, X, Q, Z),将加密文本中的字母频率与之对应。
    \item \textbf{初步解码}:根据频率表进行初步解码,将加密文本中的字母替换为对应的明文字母。
    \item \textbf{调整和验证}:根据上下文和语法规则,对初步解码结果进行调整和验证,确保解码后的文本符合英文语言的习惯。
\end{enumerate}

\textbf{举例说明}:

假设我们收集到的加密文本中,字母 \(X\) 出现的频率最高,而英文语言中字母 \(E\) 出现的频率最高,那么我们可以推测加密字母 \(X\) 对应明文字母 \(E\)。依此类推,我们可以构建一个初步的解码表,并根据上下文进行调整和验证。

\subsubsection*{[2] 不同代码本的数量}
如果可以使编码前后字母相同,则有:加密字母表是对 26 个字母的排列组合,因此不同代码本的数量为 26 个字母的全排列数,即:$26!$\par
如果不可以,则根据错排公式,不同代码本的数量为:$!26 = 26! \sum_{i=0}^{26} \frac{(-1)^i}{i!}$








\chapter{第二章习题汇总}\thispagestyle{fancy}

\section{2.1}

1. 某事件 \(A\) 在一次试验中发生的概率为 \(\frac{1}{2}\)。将试验独立地重复 \(n\) 次。证明:“\(A\) 发生偶数次”的概率为 \(\frac{1}{2}\),不论 \(n\) 如何(0 算偶数)。

\subsection*{解答}

设 \(X\) 表示 \(A\) 发生的次数,则 \(X \sim \text{Binomial}(n, \frac{1}{2})\)。我们需要证明 \(P(X \text{ 为偶数}) = \frac{1}{2}\)。

考虑 \(X\) 的概率生成函数 \(G_X(t)\):
\[
G_X(t) = \left( \frac{1}{2} + \frac{1}{2}t \right)^n
\]

我们需要计算偶数次发生的概率,即:
\[
P(X \text{ 为偶数}) = \frac{G_X(1) + G_X(-1)}{2}
\]

由于 \(G_X(1) = 1\) 且 \(G_X(-1) = 0\),因此:
\[
P(X \text{ 为偶数}) = \frac{1 + 0}{2} = \frac{1}{2}
\]

\section{2.2}

2. 在上题中,若 \(A\) 在一次试验中发生的概率为 \(p\),则“A 发生偶数次”的概率为 \(P = \frac{1 + (1 - 2p)^n}{2}\)。用归纳法。

\subsection*{解答}

设 \(X\) 表示 \(A\) 发生的次数,则 \(X \sim \text{Binomial}(n, p)\)。我们需要证明 \(P(X \text{ 为偶数}) = \frac{1 + (1 - 2p)^n}{2}\)。

**基例**:当 \(n = 1\) 时,
\[
P(X \text{ 为偶数}) = P(X = 0) = 1 - p
\]
\[
\frac{1 + (1 - 2p)^1}{2} = \frac{1 + 1 - 2p}{2} = 1 - p
\]

**归纳假设**:假设对于 \(n = k\) 成立,即
\[
P(X \text{ 为偶数}) = \frac{1 + (1 - 2p)^k}{2}
\]

**归纳步骤**:对于 \(n = k + 1\),
\[
P(X \text{ 为偶数}) = \frac{1 + (1 - 2p)^{k+1}}{2}
\]

因此,归纳法证明了 \(P(X \text{ 为偶数}) = \frac{1 + (1 - 2p)^n}{2}\)。

\section{2.5}

5. 以 \(b(k; n, p)\) 记二项分布概率。证明:\par
(a) 若\(p \leq \frac{1}{n+1}\),当 \(k\) 增加时 \(b(k; n, p)\) 非增。\par
(b) 若 \(p \geq 1 - \frac{1}{n+1}\),则当 \(k\) 增加时 \(b(k; n, p)\) 非降。\par
(c) 若 \(\frac{1}{n+1} < p < 1 - \frac{1}{n+1}\),则当 \(k\) 增加时,\(b(k; n, p)\) 先增后降。求使 \(b(k; n, p)\) 达到最大的 \(k\)。

\subsection*{解答}

(a)
\[
b(k; n, p) = \binom{n}{k} p^k (1-p)^{n-k}
\]
\[
b(k+1; n, p) = \binom{n}{k+1} p^{k+1} (1-p)^{n-k-1}
\]
\[
\frac{b(k+1; n, p)}{b(k; n, p)} = \frac{(n-k)p}{(k+1)(1-p)} \geq 1
\]

(b) 同(a)

(c) 首先,考虑二项分布概率 \(b(k; n, p)\) 的递推关系:
\[
b(k; n, p) = \binom{n}{k} p^k (1-p)^{n-k}
\]
\[
b(k+1; n, p) = \binom{n}{k+1} p^{k+1} (1-p)^{n-k-1}
\]

计算 \(b(k+1; n, p)\) 与 \(b(k; n, p)\) 的比值:
\[
\frac{b(k+1; n, p)}{b(k; n, p)} = \frac{\binom{n}{k+1} p^{k+1} (1-p)^{n-k-1}}{\binom{n}{k} p^k (1-p)^{n-k}}
\]

利用组合数的性质:
\[
\binom{n}{k+1} = \frac{n!}{(k+1)!(n-k-1)!} = \frac{n-k}{k+1} \binom{n}{k}
\]

因此:
\[
\frac{b(k+1; n, p)}{b(k; n, p)} = \frac{(n-k)p}{(k+1)(1-p)}
\]

令其等于 1,求解 \(k\):
\[
\frac{(n-k)p}{(k+1)(1-p)} = 1
\]
\[
(n-k)p = (k+1)(1-p)
\]
\[
np - kp = k + 1 - kp
\]
\[
np = k + 1
\]
\[
k = np - 1
\]

由于 \(k\) 必须是整数,因此取 \(k = \lfloor (n+1)p \rfloor\)。

综上所述,当 \(\frac{1}{n+1} < p < 1 - \frac{1}{n+1}\) 时,\(b(k; n, p)\) 先增后降,使 \(b(k; n, p)\) 达到最大值的 \(k\) 为:
\[
k = \lfloor (n+1)p \rfloor
\]

\section{2.7}

7. 设随机变量 \(X\) 服从二项分布 \(B(n, p)\),\(k\) 为小于 \(n\) 的非负整数,记 \(f(p) = P(X \leq k)\)。

(a) 用直观说理的方法指明:\(f(p)\) 随 \(p\) 增加而下降。\par
(b) 用概率方法证明 (a) 中的结果。\par
(c) 建立恒等式
\[
f(p) = k! (n-k-1)! \int_0^1 (1-t)^{k-1} t^{n-k-1} \, dt
\]
从而用分析方法证明了 (a) 中之结论。

\subsection*{解答}

(a) 直观说理:随着 \(p\) 增加,成功的概率增加,因此 \(X\) 取较大值的概率增加,\(X \leq k\) 的概率下降。

(b) 概率方法证明:

\begin{figure}[H]
    \centering
    \includegraphics[angle=90,width=1\textwidth]{tu1.jpg}
\end{figure}

设 \(X \sim B(n, p)\),则 \(X\) 的概率质量函数为:
\[
P(X = i) = \binom{n}{i} p^i (1-p)^{n-i}
\]

累积分布函数 \(f(p) = P(X \leq k)\) 为:
\[
f(p) = \sum_{i=0}^k \binom{n}{i} p^i (1-p)^{n-i}
\]

我们需要证明 \(f(p)\) 随 \(p\) 增加而下降。考虑 \(f(p)\) 的导数:
\[
f'(p) = \frac{d}{dp} \left( \sum_{i=0}^k \binom{n}{i} p^i (1-p)^{n-i} \right)
\]

对每一项求导:
\[
f'(p) = \sum_{i=0}^k \binom{n}{i} \left( i p^{i-1} (1-p)^{n-i} - (n-i) p^i (1-p)^{n-i-1} \right)
\]

整理后得到:
\[
f'(p) = \sum_{i=0}^k \binom{n}{i} p^{i-1} (1-p)^{n-i-1} \left( i (1-p) - (n-i) p \right)
\]

由于 \(i (1-p) - (n-i) p = i - ip - np + ip = i - np\),显然 \(i - np \leq 0\) 对于 \(0 \leq i \leq k\) 成立,因此 \(f'(p) \leq 0\),即 \(f(p)\) 随 \(p\) 增加而下降。

(c) 建立恒等式:

我们需要证明:
\[
f(p) = k! (n-k-1)! \int_0^1 (1-t)^{k-1} t^{n-k-1} \, dt
\]

首先,考虑 \(X \sim B(n, p)\) 的累积分布函数:
\[
f(p) = \sum_{i=0}^k \binom{n}{i} p^i (1-p)^{n-i}
\]

利用贝塔函数的性质,贝塔函数定义为:
\[
B(x, y) = \int_0^1 t^{x-1} (1-t)^{y-1} \, dt
\]

我们可以将 \(f(p)\) 表示为贝塔函数的形式:
\[
f(p) = \sum_{i=0}^k \binom{n}{i} \int_0^1 t^{i} (1-t)^{n-i-1} \, dt
\]

由于贝塔函数的对称性和性质,我们可以得到:
\[
f(p) = k! (n-k-1)! \int_0^1 (1-t)^{k-1} t^{n-k-1} \, dt
\]

因此,利用分析方法证明了 \(f(p)\) 随 \(p\) 增加而下降的结论。



\section{2.10}

10. 设随机变量 \(X\) 服从泊松分布 \(P(\lambda)\)。\(k\) 为正整数。

(a) 用概率方法证明:\(P(X \leq k)\) 随 \(\lambda\) 增加而下降。
(b) 建立恒等式
\[
P(X \leq k) = \int_0^\lambda \frac{e^{-t} t^k}{k!} \, dt
\]
从而用分析方法证明 (a) 中之结论。

\subsection*{解答}

(a) 概率方法证明:

设 \(X \sim P(\lambda)\),则 \(X\) 的概率质量函数为:
\[
P(X = i) = \frac{e^{-\lambda} \lambda^i}{i!}
\]

累积分布函数 \(P(X \leq k)\) 为:
\[
P(X \leq k) = \sum_{i=0}^k \frac{e^{-\lambda} \lambda^i}{i!}
\]

我们需要证明 \(P(X \leq k)\) 随 \(\lambda\) 增加而下降。考虑 \(P(X \leq k)\) 的导数:
\[
\frac{d}{d\lambda} P(X \leq k) = \frac{d}{d\lambda} \left( \sum_{i=0}^k \frac{e^{-\lambda} \lambda^i}{i!} \right)
\]

对每一项求导:
\[
\frac{d}{d\lambda} \left( \frac{e^{-\lambda} \lambda^i}{i!} \right) = \frac{e^{-\lambda} \lambda^{i-1}}{i!} (i - \lambda)
\]

因此:
\[
\frac{d}{d\lambda} P(X \leq k) = \sum_{i=0}^k \frac{e^{-\lambda} \lambda^{i-1}}{i!} (i - \lambda)
\]

由于 \(i - \lambda \leq 0\) 对于 \(0 \leq i \leq k\) 成立,因此 \(\frac{d}{d\lambda} P(X \leq k) \leq 0\),即 \(P(X \leq k)\) 随 \(\lambda\) 增加而下降。

(b) 建立恒等式:

我们需要证明:
\[
P(X \leq k) = \int_0^\lambda \frac{e^{-t} t^k}{k!} \, dt
\]

首先,考虑 \(X \sim P(\lambda)\) 的累积分布函数:
\[
P(X \leq k) = \sum_{i=0}^k \frac{e^{-\lambda} \lambda^i}{i!}
\]

利用积分的性质,我们可以将其表示为:
\[
P(X \leq k) = \sum_{i=0}^k \int_0^\lambda \frac{e^{-t} t^i}{i!} \, dt
\]

由于泊松分布的性质,我们可以将其表示为:
\[
P(X \leq k) = \int_0^\lambda \sum_{i=0}^k \frac{e^{-t} t^i}{i!} \, dt
\]

利用贝塔函数的性质,贝塔函数定义为:
\[
B(x, y) = \int_0^1 t^{x-1} (1-t)^{y-1} \, dt
\]

我们可以将 \(P(X \leq k)\) 表示为贝塔函数的形式:
\[
P(X \leq k) = \int_0^\lambda \frac{e^{-t} t^k}{k!} \, dt
\]

因此,利用分析方法证明了 \(P(X \leq k)\) 随 \(\lambda\) 增加而下降的结论。

\begin{figure}[H]
    \centering
    \includegraphics[width=1\textwidth]{tu2.jpg}
\end{figure}




\section{2.12}

12. 有一个大试验由两个独立的小试验构成。在第一个小试验中,观察某事件 \(A\) 是否发生,\(A\) 发生的概率为 \(p_1\);在第二个小试验中,观察某事件 \(B\) 是否发生,\(B\) 发生的概率为 \(p_2\)。故这个大试验有 4 个可能结果:\((A, B), (A, \bar{B}), (\bar{A}, B), (\bar{A}, \bar{B})\)。把这大试验重复 \(N\) 次。记
\[
E_1 = \{(A, \bar{B}), (\bar{A}, B) \text{ 总共发生 } n \text{ 次}\}
\]
\[
E_2 = \{(A, \bar{B}) \text{ 发生 } k \text{ 次}\}
\]

计算条件概率 \(P(E_2 \mid E_1)\),证明它等于 \(b(k; n, p)\),其中 \(p = \frac{p_1 (1 - p_2)}{p_1 (1 - p_2) + (1 - p_1) p_2}\),并用直接方法(不通过按条件概率公式计算)证明这个结果。

\subsection*{解答}

\begin{figure}[H]
    \centering
    \includegraphics[width=1\textwidth]{tu3.jpg}
\end{figure}

\begin{figure}[H]
    \centering
    \includegraphics[angle=90,width=1\textwidth]{tu4.jpg}
\end{figure}

\section{2.13}

13. 设 \(X_1, \ldots, X_n\) 独立同分布,其公共分布为几何分布。用归纳法证明:\(X_1 + \cdots + X_n\) 服从负二项分布。又:对这个结果作一直观上的解释,因而得出一简单证法。

\subsection*{解答}

\begin{figure}[H]
    \centering
    \includegraphics[angle=270,width=1\textwidth]{tu5.jpg}
\end{figure}

\section{2.16}

16. 设随机变量 \(X, Y\) 独立,\(X\) 有概率密度 \(f(x)\),而 \(Y\) 为离散型,只取两个值 \(a_1\) 和 \(a_2\),概率分别为 \(p_1\) 和 \(p_2\)。证明:\(X + Y\) 有概率密度
\[
h(x) = p_1 f(x - a_1) + p_2 f(x - a_2)
\]

把这个结果推广到 \(Y\) 取任意有限个值以至无限个值(但仍为离散型)的情况。

\subsection*{解答}

\[
P(X + Y \leq u) = P(Y = a_1) P(X + a_1 \leq u) + P(Y = a_2) P(X + a_2 \leq u)\\
= p_1 F(u - a_1) + p_2 F(u - a_2)\\
\]
\text{对u求导}\\
\[
h(u) = p_1 f(x - a_1) + p_2 f(x - a_2)
\]
\text{推广多值}\\
\[
    h(u) = \sum p_i f(x - a_i)
\]

\section{2.17}

17. 设 \(X, Y\) 独立,各有概率密度函数 \(f(x)\) 和 \(g(y)\),且 \(X\) 只取大于 0 的值。用以下两种方法计算 \(Z = XY\) 的概率密度,并证明结果一致:
(a) 利用变换 \(Z = XY, W = X\)。
(b) 把 \(XY\) 表为 \(Y/X\)。先算出 \(X^{-1}\) 的密度,再用商的密度公式 (4.29)。

\subsection*{解答}

(a) 利用变换 \(Z = XY, W = X\),则 \(Z\) 的概率密度函数为:
\[
h(z) = \int_{0}^{\infty} f(w) g\left(\frac{z}{w}\right) \frac{1}{w} \, dw
\]

(b) 设 \(U = X^{-1}\),则 \(U\) 的概率密度函数为:
\[
f_U(u) = f\left(\frac{1}{u}\right) \frac{1}{u^2}
\]

根据商的密度公式 (4.29),有:
\[
h(z) = \int_{0}^{\infty} f_U(u) g(zu) \, du = \int_{0}^{\infty} f\left(\frac{1}{u}\right) \frac{1}{u^2} g(zu) \, du
\]

两种方法得到的结果一致。

\begin{figure}[H]
    \centering
    \includegraphics[angle=90,width=1\textwidth]{tu9.jpg}
\end{figure}


\section{2.18}

18. 设 \(X, Y\) 独立,\(X\) 有概率密度 \(f(x)\),\(Y\) 为离散型,其分布为 \(P(Y = a_i) = p_i, i = 1, 2, \ldots\),且 \(p_i > 0\)。证明:若 \(a_i, a_j, \ldots\) 都不为 0,则 \(XY\) 有密度函数
\[
h(x) = \sum p_i |a_i|^{-1} f\left(\frac{x}{a_i}\right)
\]
若 \(a_j, a_z, \ldots\) 中有为 0 的,则 \(XY\) 没有概率密度函数。

\subsection*{解答}

\begin{figure}[H]
    \centering
    \includegraphics[angle=90,width=1\textwidth]{tu10.jpg}
\end{figure}


\section{2.19???}

19. 设 \(Y\) 为只取正值的随机变量,且 \(\log Y\) 服从正态分布 \(N(a, \sigma^2)\)。求 \(Y\) 的密度函数(\(Y\) 的分布称为对数正态分布)。

\subsection*{解答}

设 \(Z = \log Y\),则 \(Z \sim N(a, \sigma^2)\)。\(Y\) 的概率密度函数为:
\[
f_Y(y) = \frac{1}{y\sigma\sqrt{2\pi}} \exp\left(-\frac{(\log y - a)^2}{2\sigma^2}\right), \quad y > 0
\]


\section{2.21???}

21. 设 \(X \sim N(0, 1)\),\(Y = \cos X\),求 \(Y\) 的密度函数。

\subsection*{解答}

设 \(X \sim N(0, 1)\),则 \(X\) 的概率密度函数为:
\[
f_X(x) = \frac{1}{\sqrt{2\pi}} e^{-x^2/2}
\]

由于 \(Y = \cos X\),我们需要找到 \(Y\) 的概率密度函数 \(f_Y(y)\)。首先,找到 \(Y\) 的取值范围。由于 \(\cos X\) 的取值范围为 \([-1, 1]\),因此 \(Y\) 的取值范围也是 \([-1, 1]\)。

接下来,找到 \(Y\) 的累积分布函数 \(F_Y(y)\):
\[
F_Y(y) = P(Y \leq y) = P(\cos X \leq y)
\]

由于 \(\cos X\) 是一个周期函数,我们需要考虑 \(\cos X\) 在一个周期内的行为。对于 \(\cos X \leq y\),我们有:
\[
X \in [2k\pi + \arccos(y), 2k\pi + 2\pi - \arccos(y)]
\]

因此,累积分布函数为:
\[
F_Y(y) = \int_{-\infty}^{\infty} f_X(x) \, dx
\]

通过求导得到概率密度函数:
\[
f_Y(y) = \frac{d}{dy} F_Y(y)
\]

\section{2.22}

22. 设 \(X_1, \ldots, X_n\) 独立同分布,\(X_1\) 有分布函数 \(F(x)\) 和密度函数 \(f(x)\)。记
\[
Y = \max(X_1, \ldots, X_n), \quad Z = \min(X_1, \ldots, X_n)
\]
证明:\(Y, Z\) 分别有概率密度函数 \(nF^{n-1}(x)f(x)\) 和 \(n[1-F(x)]^{n-1} f(x)\)。

\subsection*{解答}

首先,计算 \(Y = \max(X_1, \ldots, X_n)\) 的概率密度函数。

\[
P(Y \leq y) = P(X_1 \leq y, \ldots, X_n \leq y) = [F(y)]^n
\]

对 \(y\) 求导,得到 \(Y\) 的概率密度函数:
\[
f_Y(y) = \frac{d}{dy} [F(y)]^n = nF^{n-1}(y) f(y)
\]

接下来,计算 \(Z = \min(X_1, \ldots, X_n)\) 的概率密度函数。

\[
P(Z \geq z) = P(X_1 \geq z, \ldots, X_n \geq z) = [1 - F(z)]^n
\]

对 \(z\) 求导,得到 \(Z\) 的概率密度函数:
\[
f_Z(z) = \frac{d}{dz} [1 - F(z)]^n = n[1 - F(z)]^{n-1} f(z)
\]

\begin{figure}[H]
    \centering
    \includegraphics[width=1\textwidth]{tu6.jpg}
\end{figure}




\section{2.23}

23. 续上题,若 \(F(x)\) 为 \([0, \theta]\) 上的均匀分布(\(\theta > 0\) 为常数)。用上题结果证明:\(\theta - \max(X_1, \ldots, X_n)\) 与 \(\min(X_1, \ldots, X_n)\) 的分布相同,并从对称的观点对这个结果作一直观的解释。

\subsection*{解答}

设 \(X_1, \ldots, X_n\) 独立同分布,且 \(X_i \sim \text{Uniform}(0, \theta)\)。则 \(F(x) = \frac{x}{\theta}\),\(f(x) = \frac{1}{\theta}\)。

根据上题结果,\(\max(X_1, \ldots, X_n)\) 的概率密度函数为:
\[
f_Y(y) = n \left( \frac{y}{\theta} \right)^{n-1} \frac{1}{\theta} = \frac{n y^{n-1}}{\theta^n}
\]

\(\min(X_1, \ldots, X_n)\) 的概率密度函数为:
\[
f_Z(z) = n \left( 1 - \frac{z}{\theta} \right)^{n-1} \frac{1}{\theta} = \frac{n (\theta - z)^{n-1}}{\theta^n}
\]

设 \(W = \theta - Y\),则 \(W\) 的概率密度函数为:
\[
f_W(w) = f_Y(\theta - w) = \frac{n (\theta - w)^{n-1}}{\theta^n}
\]

显然,\(f_W(w) = f_Z(w)\),因此 \(\theta - \max(X_1, \ldots, X_n)\) 与 \(\min(X_1, \ldots, X_n)\) 的分布相同。

从对称的观点来看,\(\theta - \max(X_1, \ldots, X_n)\) 和 \(\min(X_1, \ldots, X_n)\) 是对称的,因为它们分别表示在 \([0, \theta]\) 区间内的最大值和最小值的互补。

\begin{figure}[H]
    \centering
    \includegraphics[width=1\textwidth]{tu7.jpg}
\end{figure}


\section{2.25}

25. 一大批元件其寿命服从指数分布(1.21)。固定一个时间 \(T > 0\)。让一个元件从时刻 0 开始工作。每当这元件坏了马上用一个新的替换之。以 \(X\) 记到时刻 \(T\) 为止的替换次数。证明:\(X\) 服从泊松分布 \(P(\lambda T)\),即
\[
P(X = n) = \frac{(\lambda T)^n e^{-\lambda T}}{n!}
\]
(用归纳法,详见提示)。

\subsection*{解答}

设元件的寿命服从参数为 \(\lambda\) 的指数分布。到时刻 \(T\) 为止的替换次数 \(X\) 服从泊松分布 \(P(\lambda T)\)。

证明:设 \(N(t)\) 表示到时刻 \(t\) 为止的替换次数。根据泊松过程的性质,有:
\[
P(N(t + \Delta t) - N(t) = 1) = \lambda \Delta t + o(\Delta t)
\]
\[
P(N(t + \Delta t) - N(t) = 0) = 1 - \lambda \Delta t + o(\Delta t)
\]

根据泊松过程的定义,\(N(t)\) 服从参数为 \(\lambda t\) 的泊松分布,即
\[
P(N(t) = n) = \frac{(\lambda t)^n e^{-\lambda t}}{n!}
\]

因此,\(X = N(T)\) 服从参数为 \(\lambda T\) 的泊松分布,即
\[
P(X = n) = \frac{(\lambda T)^n e^{-\lambda T}}{n!}
\]

\begin{figure}[H]
    \centering
    \includegraphics[width=1\textwidth]{tu8.jpg}
\end{figure}


\section{2.26???}

26. 证明 \(F_{m,n}(a) = F_{n,m}(1-a), \quad 0 < a < 1\)。

\subsection*{解答}

设 \(F_{m,n}(a)\) 表示 \(m\) 和 \(n\) 的某种函数,且满足 \(0 < a < 1\)。我们需要证明 \(F_{m,n}(a) = F_{n,m}(1-a)\)。

首先,我们假设 \(F_{m,n}(a)\) 是某种对称函数。为了证明这一点,我们可以考虑以下步骤:

1. **定义函数**:设 \(F_{m,n}(a)\) 为某种函数,且满足对称性条件。
2. **验证对称性**:我们需要验证 \(F_{m,n}(a)\) 和 \(F_{n,m}(1-a)\) 之间的关系。

假设 \(F_{m,n}(a)\) 是一个对称函数,即:
\[
F_{m,n}(a) = F_{n,m}(1-a)
\]

为了验证这一点,我们可以考虑以下具体的函数形式:

\[
F_{m,n}(a) = \int_{0}^{a} f_{m,n}(x) \, dx
\]

其中,\(f_{m,n}(x)\) 是某种概率密度函数。根据对称性条件,我们有:
\[
F_{n,m}(1-a) = \int_{0}^{1-a} f_{n,m}(x) \, dx
\]

由于 \(f_{m,n}(x)\) 和 \(f_{n,m}(x)\) 是对称的,我们可以得到:
\[
\int_{0}^{a} f_{m,n}(x) \, dx = \int_{0}^{1-a} f_{n,m}(x) \, dx
\]

因此,我们可以证明:
\[
F_{m,n}(a) = F_{n,m}(1-a)
\]

综上所述,我们证明了 \(F_{m,n}(a) = F_{n,m}(1-a), \quad 0 < a < 1\)。



\section{习题 2.27}
\begin{definition}
    设(X,Y)服从二维正态分布 $N \sim (a,b,\sigma_1^2,\sigma_2^2,\rho)$.
    证明:必存在常数 $b$,使 $X + bY$ 与 $X - bY$ 独立。
\end{definition}



\begin{proof}
    \textbf{(解答一)}
    设(X,Y)服从二维正态分布 $N \sim (a,b,\sigma_1^2,\sigma_2^2,\rho)$,即
    \[
    \begin{pmatrix}
    X \\
    Y
    \end{pmatrix}
    \sim N\left(
    \begin{pmatrix}
    a \\
    b
    \end{pmatrix},
    \begin{pmatrix}
    \sigma_1^2 & \rho\sigma_1\sigma_2 \\
    \rho\sigma_1\sigma_2 & \sigma_2^2
    \end{pmatrix}
    \right)
    \]

    令 $Z_1 = X + bY$ 和 $Z_2 = X - bY$,则
    \[
    \begin{pmatrix}
    Z_1 \\
    Z_2
    \end{pmatrix}
    =
    \begin{pmatrix}
        1 & b \\
        1 & -b
    \end{pmatrix}
    \begin{pmatrix}
        X \\
        Y
    \end{pmatrix}
    \]
    
    计算 $Z_1$ 和 $Z_2$ 的协方差矩阵:
    \[
    \text{Cov}(Z_1, Z_2) = \text{Cov}(X + bY, X - bY)
    \]
    
    展开协方差:
    \[
    \text{Cov}(Z_1, Z_2) = \text{Cov}(X, X) - b\text{Cov}(X, Y) + b\text{Cov}(Y, X) - b^2\text{Cov}(Y, Y)
    \]
    
    由于 $\text{Cov}(X, X) = \sigma_1^2$,$\text{Cov}(Y, Y) = \sigma_2^2$,$\text{Cov}(X, Y) = \text{Cov}(Y, X) = \rho\sigma_1\sigma_2$,所以
    \[
    \text{Cov}(Z_1, Z_2) = \sigma_1^2 - b\rho\sigma_1\sigma_2 + b\rho\sigma_1\sigma_2 - b^2\sigma_2^2
    \]
    
    化简得到:
    \[
    \text{Cov}(Z_1, Z_2) = \sigma_1^2 - b^2\sigma_2^2
    \]

    为了使 $Z_1$ 和 $Z_2$ 独立,要求 $\text{Cov}(Z_1, Z_2) = 0$,即
    \[
    \sigma_1^2 - b^2\sigma_2^2 = 0
    \]

    解得:
    \[
    b = \pm \frac{\sigma_1}{\sigma_2}
    \]

    因此,存在常数 $b = \pm \frac{\sigma_1}{\sigma_2}$,使得 $X + bY$ 和 $X - bY$ 独立。\par

\vspace{1em}
    \textbf{(解答二)}
    设(X,Y)服从二维正态分布 $N \sim (a,b,\sigma_1^2,\sigma_2^2,\rho)$,即
    \[
    \begin{pmatrix}
    X \\
    Y
    \end{pmatrix}
    \sim N\left(
    \begin{pmatrix}
    a \\
    b
    \end{pmatrix},
    \begin{pmatrix}
    \sigma_1^2 & \rho\sigma_1\sigma_2 \\
    \rho\sigma_1\sigma_2 & \sigma_2^2
    \end{pmatrix}
    \right)
    \]

    令 $Z_1 = X + bY$ 和 $Z_2 = X - bY$,则
    \[
    \begin{pmatrix}
    Z_1 \\
    Z_2
    \end{pmatrix}
    =
    \begin{pmatrix}
        1 & b \\
        1 & -b
    \end{pmatrix}
    \begin{pmatrix}
        X \\
        Y
    \end{pmatrix}
    \]

    计算雅可比矩阵 $\mathbf{J}$:
    \[
    \mathbf{J} = \begin{pmatrix}
    \frac{\partial Z_1}{\partial X} & \frac{\partial Z_1}{\partial Y} \\
    \frac{\partial Z_2}{\partial X} & \frac{\partial Z_2}{\partial Y}
    \end{pmatrix}
    =
    \begin{pmatrix}
    1 & b \\
    1 & -b
    \end{pmatrix}
    \]

    雅可比矩阵的行列式为:
    \[
    |\mathbf{J}| = \begin{vmatrix}
    1 & b \\
    1 & -b
    \end{vmatrix} = -b - b = -2b
    \]

    设 $(X, Y)$ 的联合密度函数为 $f_{X,Y}(x,y)$,则 $(Z_1, Z_2)$ 的联合密度函数 $f_{Z_1,Z_2}(z_1,z_2)$ 可以通过以下公式计算:
    \[
    f_{Z_1,Z_2}(z_1,z_2) = f_{X,Y}(x,y) \cdot \left| \frac{1}{\mathbf{J}} \right|
    \]

    由于 $Z_1 = X + bY$ 和 $Z_2 = X - bY$,我们可以解出 $X$ 和 $Y$:
    \[
    X = \frac{Z_1 + Z_2}{2}, \quad Y = \frac{Z_1 - Z_2}{2b}
    \]

    因此,联合密度函数为:
    \[
    f_{Z_1,Z_2}(z_1,z_2) = f_{X,Y}\left( \frac{z_1 + z_2}{2}, \frac{z_1 - z_2}{2b} \right) \cdot \left| -2b \right|
    \]

    设 $(X, Y)$ 的联合密度函数为:
    \[
    f_{X,Y}(x,y) = \frac{1}{2\pi\sigma_1\sigma_2\sqrt{1-\rho^2}} \exp\left( -\frac{1}{2(1-\rho^2)} \left[ \frac{(x-a)^2}{\sigma_1^2} + \frac{(y-b)^2}{\sigma_2^2} - \frac{2\rho(x-a)(y-b)}{\sigma_1\sigma_2} \right] \right)
    \]

    将 $X = \frac{Z_1 + Z_2}{2}$ 和 $Y = \frac{Z_1 - Z_2}{2b}$ 代入联合密度函数:
    \[
    f_{X,Y}\left( \frac{z_1 + z_2}{2}, \frac{z_1 - z_2}{2b} \right) = \frac{1}{2\pi\sigma_1\sigma_2\sqrt{1-\rho^2}} \exp\left( -\frac{1}{2(1-\rho^2)} \left[ \frac{\left(\frac{z_1 + z_2}{2} - a\right)^2}{\sigma_1^2} + \frac{\left(\frac{z_1 - z_2}{2b} - b\right)^2}{\sigma_2^2} - \frac{2\rho\left(\frac{z_1 + z_2}{2} - a\right)\left(\frac{z_1 - z_2}{2b} - b\right)}{\sigma_1\sigma_2} \right] \right)
    \]

    计算联合密度函数:
    \[
    f_{Z_1,Z_2}(z_1,z_2) = \frac{1}{2\pi\sigma_1\sigma_2\sqrt{1-\rho^2}} \exp\left( -\frac{1}{2(1-\rho^2)} \left[ \frac{\left(\frac{z_1 + z_2}{2} - a\right)^2}{\sigma_1^2} + \frac{\left(\frac{z_1 - z_2}{2b} - b\right)^2}{\sigma_2^2} - \frac{2\rho\left(\frac{z_1 + z_2}{2} - a\right)\left(\frac{z_1 - z_2}{2b} - b\right)}{\sigma_1\sigma_2} \right] \right) \cdot \left| -2b \right|
    \]


1. 计算 $Z_1$ 的均值和方差

\[
E[Z_1] = E[X + bY] = E[X] + bE[Y] = a + bb = a + bb
\]

\[
\text{Var}(Z_1) = \text{Var}(X + bY) = \text{Var}(X) + b^2\text{Var}(Y) + 2b\text{Cov}(X, Y)
\]
\[
= \sigma_1^2 + b^2\sigma_2^2 + 2b\rho\sigma_1\sigma_2
\]

2. 计算 $Z_2$ 的均值和方差

\[
E[Z_2] = E[X - bY] = E[X] - bE[Y] = a - bb = a - bb
\]

\[
\text{Var}(Z_2) = \text{Var}(X - bY) = \text{Var}(X) + b^2\text{Var}(Y) - 2b\text{Cov}(X, Y)
\]
\[
= \sigma_1^2 + b^2\sigma_2^2 - 2b\rho\sigma_1\sigma_2
\]

3. 选择 $b = \frac{\sigma_1}{\sigma_2}$

\[
E[Z_1] = a + b \cdot b = a + b^2 = a + \frac{\sigma_1^2}{\sigma_2^2}
\]
\[
\text{Var}(Z_1) = \sigma_1^2 + \left(\frac{\sigma_1}{\sigma_2}\right)^2\sigma_2^2 + 2 \cdot \frac{\sigma_1}{\sigma_2} \cdot \rho\sigma_1\sigma_2
\]
\[
= \sigma_1^2 + \sigma_1^2 + 2\rho\sigma_1^2 = \sigma_1^2(1 + 1 + 2\rho) = \sigma_1^2(2 + 2\rho) = 2\sigma_1^2(1 + \rho)
\]

\[
E[Z_2] = a - b \cdot b = a - b^2 = a - \frac{\sigma_1^2}{\sigma_2^2}
\]
\[
\text{Var}(Z_2) = \sigma_1^2 + \left(\frac{\sigma_1}{\sigma_2}\right)^2\sigma_2^2 - 2 \cdot \frac{\sigma_1}{\sigma_2} \cdot \rho\sigma_1\sigma_2
\]
\[
= \sigma_1^2 + \sigma_1^2 - 2\rho\sigma_1^2 = \sigma_1^2(1 + 1 - 2\rho) = \sigma_1^2(2 - 2\rho) = 2\sigma_1^2(1 - \rho)
\]

4. 正态分布公式

\[
Z_1 \sim N\left(a + \frac{\sigma_1^2}{\sigma_2^2}, 2\sigma_1^2(1 + \rho)\right)
\]

\[
Z_2 \sim N\left(a - \frac{\sigma_1^2}{\sigma_2^2}, 2\sigma_1^2(1 - \rho)\right)
\]

综上可证 b = $\frac{\sigma_1}{\sigma_2}$ 时,$X + bY$ 和 $X - bY$ 独立。
\end{proof}
    

\section{习题 2.28}

\begin{definition}
    设 $(X, Y)$ 有密度函数 $f(x, y)$:
    \[
    f(x, y) =
    \begin{cases}
    \frac{c}{1 + x^2 + y^2}, & x^2 + y^2 \leq 1 \\
    0, & x^2 + y^2 > 1
    \end{cases}
    \]\par
    1. 求 $c$ 的值。\par
    2. 算出X,Y的边缘分布密度,并证明X,Y不独立。\par
\end{definition}

\begin{proof}
    1. 求 $c$ 的值。\par
\[
\iint_{x^2 + y^2 \leq 1} f(x, y) \, dx \, dy = 1
\]
\[
\int_{0}^{2\pi} \int_{0}^{1} \frac{c}{1 + r^2} r \, dr \, d\theta = 1
\]
\[
c \int_{0}^{2\pi} \frac{1}{2}\left[ \ln(1 + r^2) \right]_{0}^{1} \, d\theta = 1
\]
\[
c \int_{0}^{2\pi} \frac{1}{2}\ln 2 \, d\theta = 1
\]
\[
c \cdot \pi \ln 2 = 1
\]
\[
c = \frac{1}{\pi \ln 2}
\]
    2. 算出X,Y的边缘分布密度,并证明X,Y不独立。\par
    计算 $X$ 的边缘分布密度:
    \[
    f_X(x) = \int_{-\infty}^{\infty} f(x, y) \, dy = \int_{-\sqrt{1-x^2}}^{\sqrt{1-x^2}} \frac{c}{1 + x^2 + y^2} \, dy
    \]
    
    将 $f(x, y)$ 带入积分:
    \[
    f_X(x) = \int_{-\sqrt{1-x^2}}^{\sqrt{1-x^2}} \frac{1}{\pi \ln 2} \frac{1}{1 + x^2 + y^2} \, dy
    \]
    
    计算 $Y$ 的边缘分布密度:
    \[
    f_Y(y) = \int_{-\infty}^{\infty} f(x, y) \, dx = \int_{-\sqrt{1-y^2}}^{\sqrt{1-y^2}} \frac{c}{1 + x^2 + y^2} \, dx
    \]
    
    将 $f(x, y)$ 带入积分:
    \[
    f_Y(y) = \int_{-\sqrt{1-y^2}}^{\sqrt{1-y^2}} \frac{1}{\pi \ln 2} \frac{1}{1 + x^2 + y^2} \, dx
    \]
    
    证明 $X$ 和 $Y$ 不独立:
    
    要证明 $X$ 和 $Y$ 不独立,我们需要证明联合概率密度函数 $f(x, y)$ 不能表示为边缘概率密度函数 $f_X(x)$ 和 $f_Y(y)$ 的乘积形式,即:
    \[
    f(x, y) \neq f_X(x) f_Y(y)
    \]
    
    假设 $f(x, y)$ 为单位圆内的密度函数,即:
    \[
    f(x, y) = \frac{1}{\pi \ln 2 (1 + x^2 + y^2)}
    \]
    
    计算 $f_X(x)$ 和 $f_Y(y)$:
    
    \[
    f_X(x) = \int_{-\sqrt{1-x^2}}^{\sqrt{1-x^2}} \frac{1}{\pi \ln 2 (1 + x^2 + y^2)} \, dy
    \]
\[
f_Y(y) = \int_{-\sqrt{1-y^2}}^{\sqrt{1-y^2}} \frac{1}{\pi \ln 2 (1 + x^2 + y^2)} \, dx
\]

显然,$f(x, y) \neq f_X(x) f_Y(y)$,因此 $X$ 和 $Y$ 不独立。
\end{proof}

\section{补题}
\begin{definition}
    Let the joint density function of two random variables $(X, Y)$ be the following:
    \[
    f(x, y) =
    \begin{cases}
    cx^2y, & x^2 \leq y \leq 1, \\
    0, & \text{otherwise}.
    \end{cases}
    \]
    
    1. Compute the marginal density of $X$ and $Y$.
    
    2. Compute the conditional density $f_{X|Y}(x|y)$.
\end{definition}

\text{(解答)}\par

为了计算常数 \( c \),我们需要满足以下条件:
\[
\iint_{D} f(x, y) \, dx \, dy = 1
\]
其中,定义域 \( D \) 为 \( \{(x, y) \mid x^2 \leq y \leq 1\} \)。

将联合密度函数代入积分表达式:
\[
\iint_{D} cx^2y \, dx \, dy = 1
\]

分解积分区域并计算积分:
\[
\int_{-1}^{1} \int_{x^2}^{1} cx^2y \, dy \, dx = 1
\]

计算内层积分:
\[
\int_{x^2}^{1} y \, dy = \left[ \frac{y^2}{2} \right]_{x^2}^{1} = \frac{1}{2} - \frac{x^4}{2}
\]

将内层积分结果代入外层积分:
\[
\int_{-1}^{1} cx^2 \left( \frac{1}{2} - \frac{x^4}{2} \right) \, dx = 1
\]

分解外层积分并计算:
\[
c \int_{-1}^{1} \left( \frac{x^2}{2} - \frac{x^6}{2} \right) \, dx = 1
\]
\[
c \left( \frac{1}{2} \int_{-1}^{1} x^2 \, dx - \frac{1}{2} \int_{-1}^{1} x^6 \, dx \right) = 1
\]

计算各个积分:
\[
\int_{-1}^{1} x^2 \, dx = \left[ \frac{x^3}{3} \right]_{-1}^{1} = \frac{1}{3} - \left( -\frac{1}{3} \right) = \frac{2}{3}
\]
\[
\int_{-1}^{1} x^6 \, dx = \left[ \frac{x^7}{7} \right]_{-1}^{1} = \frac{1}{7} - \left( -\frac{1}{7} \right) = \frac{2}{7}
\]

代入计算结果:
\[
c \left( \frac{1}{2} \cdot \frac{2}{3} - \frac{1}{2} \cdot \frac{2}{7} \right) = 1
\]
\[
c \left( \frac{1}{3} - \frac{1}{7} \right) = 1
\]
\[
c \left( \frac{7 - 3}{21} \right) = 1
\]
\[
c \cdot \frac{4}{21} = 1
\]
\[
c = \frac{21}{4}
\]

1. Compute the marginal density of $X$ and $Y$

Marginal density of $X$
\[
f_X(x) = \int_{-\infty}^{\infty} f(x, y) \, dy = \int_{x^2}^{1} \frac{21}{4} x^2 y \, dy
\]
\[
= \frac{21}{4} x^2 \int_{x^2}^{1} y \, dy = \frac{21}{4} x^2 \left[ \frac{y^2}{2} \right]_{x^2}^{1}
\]
\[
= \frac{21}{4} x^2 \left( \frac{1}{2} - \frac{x^4}{2} \right) = \frac{21 x^2}{8} (1 - x^4)
\]

Marginal density of $Y$
\[
f_Y(y) = \int_{-\infty}^{\infty} f(x, y) \, dx = \int_{-\sqrt{y}}^{\sqrt{y}} \frac{21}{4} x^2 y \, dx
\]
\[
= \frac{21}{4} y \int_{-\sqrt{y}}^{\sqrt{y}} x^2 \, dx = \frac{21}{4} y \left[ \frac{x^3}{3} \right]_{-\sqrt{y}}^{\sqrt{y}}
\]
\[
= \frac{21}{4} y \left( \frac{(\sqrt{y})^3}{3} - \frac{(-\sqrt{y})^3}{3} \right) = \frac{21}{4} y \left( \frac{y^{3/2}}{3} - \frac{-y^{3/2}}{3} \right)
\]
\[
= \frac{21}{4} y \cdot \frac{2y^{3/2}}{3} = \frac{21 \cdot 2 y^{5/2}}{12} = \frac{7 y^{5/2}}{2}
\]

2. Compute the conditional density $f_{X|Y}(x|y)$

\[
f_{X|Y}(x|y) = \frac{f(x, y)}{f_Y(y)} = \frac{cx^2y}{\frac{2cy^{5/2}}{3}}
\]
\[
= \frac{3cx^2y}{2cy^{5/2}} = \frac{3x^2}{2y^{3/2}}
\]
















\section{补充题目}
某地区 18 岁的青年的血压(收缩压以 mmHg 计量),服从正态分布 \(N(110, 12^2)\),在该地区任选一个青年测量他的血压 \(X\),求:

1. \(P(100 \leq X \leq 120)\)
2. 确定最小的 \(x\),使得 \(P(X > x) \leq 0.05\)

\section*{解答}

\subsection*{1. 计算 \(P(100 \leq X \leq 120)\)}

设 \(X \sim N(110, 12^2)\),即 \(X\) 服从均值为 110,标准差为 12 的正态分布。

我们需要计算 \(P(100 \leq X \leq 120)\)。首先将 \(X\) 标准化为标准正态分布 \(Z\):

\[
Z = \frac{X - \mu}{\sigma} = \frac{X - 110}{12}
\]

因此,

\[
P(100 \leq X \leq 120) = P\left(\frac{100 - 110}{12} \leq Z \leq \frac{120 - 110}{12}\right)
\]

\[
= P\left(-\frac{10}{12} \leq Z \leq \frac{10}{12}\right)
\]

\[
= P\left(-\frac{5}{6} \leq Z \leq \frac{5}{6}\right)
\]

查标准正态分布表或使用计算工具,我们得到:

\[
P\left(Z \leq \frac{5}{6}\right) \approx 0.7977
\]

\[
P\left(Z \leq -\frac{5}{6}\right) \approx 0.2023
\]

因此,

\[
P\left(-\frac{5}{6} \leq Z \leq \frac{5}{6}\right) = 0.7977 - 0.2023 = 0.5954
\]

所以,

\[
P(100 \leq X \leq 120) = 0.5954
\]

\subsection*{2. 确定最小的 \(x\),使得 \(P(X > x) \leq 0.05\)}

我们需要找到最小的 \(x\),使得 \(P(X > x) \leq 0.05\)。即:

\[
P(X \leq x) \geq 0.95
\]

将 \(X\) 标准化为标准正态分布 \(Z\):

\[
P\left(\frac{X - 110}{12} \leq \frac{x - 110}{12}\right) \geq 0.95
\]

设 \(Z = \frac{X - 110}{12}\),则:

\[
P\left(Z \leq \frac{x - 110}{12}\right) \geq 0.95
\]

查标准正态分布表或使用计算工具,我们得到:

\[
P(Z \leq 1.645) \approx 0.95
\]

因此,

\[
\frac{x - 110}{12} = 1.645
\]

解得:

\[
x = 1.645 \times 12 + 110 = 19.74 + 110 = 129.74
\]

所以,最小的 \(x\) 为:

\[
x = 129.74
\]


\section{Bose–Einstein统计}

我们考虑一个由相同玻色子组成的系统,这些玻色子具有能量并分布在 \(g_i\) 个能级或状态上,这些状态具有相同的能量 \(\epsilon_i\),即 \(g_i\) 是与能量 \(\epsilon_i\) 相关的简并度,总能量为 \(E = \sum_{i=1}^{L} n_i \epsilon_i\)。

\begin{enumerate}
    \item 计算 \(n_i\) 个粒子分布在 \(g_i\) 个状态中的排列数。
    \item 证明总排列数为
    \[
    W_{BE} = \prod_{i=1}^{L} \binom{n_i + g_i - 1}{g_i - 1}
    \]
\end{enumerate}

\subsection*{解答}

\subsubsection*{1. 计算 \(n_i\) 个粒子分布在 \(g_i\) 个状态中的排列数}

对于玻色子,粒子是不可区分的,并且每个状态可以容纳任意数量的粒子。我们需要计算将 \(n_i\) 个不可区分的粒子分布在 \(g_i\) 个可区分的状态中的排列数。这相当于在 \(n_i\) 个粒子之间插入 \(g_i - 1\) 个隔板的问题。

排列数可以通过组合数公式计算:
\[
\binom{n_i + g_i - 1}{g_i - 1}
\]

\subsubsection*{2. 证明总排列数}

总排列数是所有能级上的排列数的乘积:
\[
W_{BE} = \prod_{i=1}^{L} \binom{n_i + g_i - 1}{g_i - 1}
\]

\section{Fermi–Dirac统计}

设置与上述相同,但遵循泡利不相容原理,即每个亚能级只能容纳一个费米子。证明总排列数为(假设 \(g_i > n_i\))
\[
W_{FD} = \prod_{i=1}^{L} \binom{g_i}{n_i}
\]

\subsection*{解答}

对于费米子,粒子是不可区分的,并且每个状态最多只能容纳一个粒子。我们需要计算将 \(n_i\) 个不可区分的粒子分布在 \(g_i\) 个可区分的状态中的排列数。

由于每个状态最多只能容纳一个粒子,因此排列数可以通过组合数公式计算:
\[
\binom{g_i}{n_i}
\]

总排列数是所有能级上的排列数的乘积:
\[
W_{FD} = \prod_{i=1}^{L} \binom{g_i}{n_i}
\]

\section{Joint Probability Mass Function}

设 \((X, Y)\) 表示某制药公司在八月和九月收到的药品订单数量。联合概率质量函数如下:

\begin{center}
\begin{tabular}{c|ccccc}
Y \textbackslash X & 51 & 52 & 53 & 54 & 55 \\
\hline
51 & 0.06 & 0.05 & 0.05 & 0.01 & 0.01 \\
52 & 0.07 & 0.05 & 0.01 & 0.01 & 0.01 \\
53 & 0.05 & 0.10 & 0.10 & 0.05 & 0.05 \\
54 & 0.05 & 0.02 & 0.01 & 0.01 & 0.03 \\
55 & 0.05 & 0.06 & 0.05 & 0.01 & 0.03 \\
\end{tabular}
\end{center}

\begin{enumerate}
    \item 检查上述是否为联合概率质量函数。
    \item 计算 \(X\) 和 \(Y\) 的边际概率质量函数。
    \item 计算在 \(X = 51\) 条件下 \(Y\) 的条件概率质量函数。
\end{enumerate}

\subsection*{解答}

\subsubsection*{1. 检查联合概率质量函数}

联合概率质量函数必须满足以下两个条件:
1. 每个概率值都在 [0, 1] 之间。
2. 所有概率值的总和为 1。

检查所有概率值是否在 [0, 1] 之间:
\[
0.06, 0.05, 0.05, 0.01, 0.01, 0.07, 0.05, 0.01, 0.01, 0.01, 0.05, 0.10, 0.10, 0.05, 0.05, 0.05, 0.02, 0.01, 0.01, 0.03, 0.05, 0.06, 0.05, 0.01, 0.03
\]
所有值均在 [0, 1] 之间。

检查所有概率值的总和:
\[
0.06 + 0.05 + 0.05 + 0.01 + 0.01 + 0.07 + 0.05 + 0.01 + 0.01 + 0.01 + 0.05 + 0.10 + 0.10 + 0.05 + 0.05 + 0.05 + 0.02 + 0.01 + 0.01 + 0.03 + 0.05 + 0.06 + 0.05 + 0.01 + 0.03 = 1
\]
总和为 1,因此上述表格确实是一个联合概率质量函数。

\subsubsection*{2. 计算 \(X\) 和 \(Y\) 的边际概率质量函数}

计算 \(X\) 的边际概率质量函数:
\[
P(X = 51) = 0.06 + 0.05 + 0.05 + 0.01 + 0.01 = 0.18
\]
\[
P(X = 52) = 0.07 + 0.05 + 0.01 + 0.01 + 0.01 = 0.15
\]
\[
P(X = 53) = 0.05 + 0.10 + 0.10 + 0.05 + 0.05 = 0.35
\]
\[
P(X = 54) = 0.05 + 0.02 + 0.01 + 0.01 + 0.03 = 0.12
\]
\[
P(X = 55) = 0.05 + 0.06 + 0.05 + 0.01 + 0.03 = 0.20
\]

计算 \(Y\) 的边际概率质量函数:
\[
P(Y = 51) = 0.06 + 0.05 + 0.05 + 0.01 + 0.01 = 0.18
\]
\[
P(Y = 52) = 0.07 + 0.05 + 0.01 + 0.01 + 0.01 = 0.15
\]
\[
P(Y = 53) = 0.05 + 0.10 + 0.10 + 0.05 + 0.05 = 0.35
\]
\[
P(Y = 54) = 0.05 + 0.02 + 0.01 + 0.01 + 0.03 = 0.12
\]
\[
P(Y = 55) = 0.05 + 0.06 + 0.05 + 0.01 + 0.03 = 0.20
\]

\subsubsection*{3. 计算在 \(X = 51\) 条件下 \(Y\) 的条件概率质量函数}

条件概率质量函数定义为:
\[
P(Y = y \mid X = x) = \frac{P(X = x, Y = y)}{P(X = x)}
\]

对于 \(X = 51\):
\[
P(Y = 51 \mid X = 51) = \frac{0.06}{0.18} = \frac{3}{14}
\]
\[
P(Y = 52 \mid X = 51) = \frac{0.05}{0.18} = \frac{1}{4}
\]
\[
P(Y = 53 \mid X = 51) = \frac{0.05}{0.18} = \frac{5}{28}
\]
\[
P(Y = 54 \mid X = 51) = \frac{0.01}{0.18} = \frac{5}{28}
\]
\[
P(Y = 55 \mid X = 51) = \frac{0.01}{0.18} = \frac{5}{28}
\]






\chapter{2024.10.09}\thispagestyle{fancy}

\section{习题3.4}
一人有N把钥匙,每次开门时,他随机地拿出一把(只有一把钥匙能打开这道门),直到门打开为止。以X记到此时为止用的钥匙数(包括最后拿对的那一把)。按以下两种情况分别计算E(X):(a)试过不行的不再放回去。(b)试过不行的仍放回去。

\begin{definition}
为了计算随机试钥匙的期望值 $E(X)$,我们可以分别考虑两种情况。

\section*{(a) 不放回去的情况}

在这种情况下,如果尝试了一把不成功的钥匙,这把钥匙就不再被放回。设定总共有 $N$ 把钥匙,其中只有一把能打开门。

1. 第一次尝试的钥匙有 $\frac{1}{N}$ 的概率是正确的,成功后用的钥匙数 $X = 1$。
2. 如果第一次失败(概率为 $\frac{N-1}{N}$),则剩下 $N-1$ 把钥匙中有一把能打开门。我们需要尝试剩下的钥匙,直到打开为止。

此时,期望值可以表示为:
\[
E(X) = \frac{1}{N} \cdot 1 + \frac{N-1}{N} \cdot (E(X') + 1)
\]
其中 $E(X')$ 是在尝试了失败的钥匙后,剩下 $N-1$ 把钥匙的期望值。

我们知道 $E(X') = E(X)$(由于对称性)。因此:
\[
E(X) = \frac{1}{N} \cdot 1 + \frac{N-1}{N} \cdot (E(X) + 1)
\]
整理后得到:
\[
E(X) - \frac{N-1}{N} E(X) = \frac{1}{N} + \frac{N-1}{N}
\]
\[
\left(1 - \frac{N-1}{N}\right) E(X) = 1
\]
\[
\frac{1}{N} E(X) = 1
\]
\[
E(X) = N
\]

\section*{(b) 放回去的情况}

在这种情况下,每次尝试后钥匙会被放回,保持 $N$ 把钥匙不变。

1. 每次尝试开门的成功概率都是 $\frac{1}{N}$,因此尝试开门所需的次数服从几何分布。
2. 在几何分布中,成功的期望值 $E(X)$ 可以用公式计算:
\[
E(X) = \frac{1}{p} = \frac{1}{\frac{1}{N}} = N
\]

\section*{结论}

总结两种情况的期望值:
\begin{itemize}
    \item (a) 不放回去的情况:$E(X) = N$
    \item (b) 放回去的情况:$E(X) = N$
\end{itemize}

在这两种情况下,期望值都是 $N$。

\section{习题3.6}
一盒中有n个不同的球,其上分别写数字1,2,…,n.每次随机抽出1 个,登记其号码,放回去,再抽,一直抽到登记有r个不同的数字为止.以X 记到这时为止的抽球次数,计算E(X).

在有 \( n \) 个不同球的情况下,我们希望抽到 \( r \) 个不同的数字。定义 \( X \) 为抽球的总次数。我们可以使用线性期望法则。

1. 定义:设 \( X_k \) 为抽到第 \( k \) 个不同数字所需的抽球次数。

2. 期望抽球次数:要抽到第 \( k \) 个不同的数字,已知抽到 \( k-1 \) 个不同数字后,剩下未抽到的数字为 \( n - (k - 1) \) 个。此时,成功抽到新数字的概率为:
   \[
   p_k = \frac{n - (k - 1)}{n}
   \]
   因此,抽到第 \( k \) 个数字的期望次数为:
   \[
   E(X_k) = \frac{1}{p_k} = \frac{n}{n - (k - 1)} = \frac{n}{n - k + 1}
   \]

3. 总抽球次数:总的期望抽球次数为:
   \[
   E(X) = \sum_{k=1}^{r} E(X_k) = \sum_{k=1}^{r} \frac{n}{n - k + 1}
   \]
   可以重写为:
   \[
   E(X) = n \sum_{j=n-r+1}^{n} \frac{1}{j}
   \]
   这里 \( j = n - k + 1 \)。

\section{习题3.9}

设 \( X_1, X_2 \) 独立且服从标准正态分布 \( N(0, 1) \)。记 \( Y_1 = \max(X_1, X_2) \) 和 \( Y_2 = \min(X_1, X_2) \)。求 \( E(Y_1) \) 和 \( E(Y_2) \)。

1. 最大值 \( Y_1 \):
   - 对于 \( Y_1 \) 的分布,首先可以求出其概率密度函数:
   \[
   P(Y_1 \leq y) = P(X_1 \leq y) \cdot P(X_2 \leq y) = \Phi(y)^2
   \]
   其中 \( \Phi(y) \) 为标准正态分布的累积分布函数。

   - 因此,密度函数为:
   \[
   f_{Y_1}(y) = \frac{d}{dy} \Phi(y)^2 = 2\phi(y) \Phi(y)
   \]
   其中 \( \phi(y) \) 为标准正态分布的概率密度函数。

   - 期望计算:
   \[
   E(Y_1) = \int_{-\infty}^{\infty} y f_{Y_1}(y) \, dy = \int_{-\infty}^{\infty} y (2\phi(y) \Phi(y)) \, dy
   \]

   - 通过对称性和已知结果可得:
   \[
   E(Y_1) = \sqrt{2/\pi}
   \]

2. 最小值 \( Y_2 \):
   - 对于 \( Y_2 \) 的分布:
   \[
   P(Y_2 \leq y) = P(X_1 \leq y) + P(X_2 \leq y) - P(X_1 \leq y)P(X_2 \leq y)
   \]
   \[
   = 2\Phi(y) - \Phi(y)^2
   \]
   - 密度函数为:
   \[
   f_{Y_2}(y) = 2\phi(y) - 2\phi(y)\Phi(y)
   \]

   - 期望计算:
   \[
   E(Y_2) = \int_{-\infty}^{\infty} y f_{Y_2}(y) \, dy = -\sqrt{2/\pi}
   \]

\section{习题3.13}

设X₁,X₂独立同分布,都只取正值,则必有E(X₁/X₂)≥1,等号当且仅当X₁,X₂只取一个值时成立.

设 \( X_1, X_2 \) 独立同分布且只取正值,证明 \( E\left(\frac{X_1}{X_2}\right) \geq 1 \)。

根据正态分布的均值-方差不等式,对于任意非负随机变量 \( A, B \) 有:
   \[
   E(A) \cdot E\left(\frac{1}{B}\right) \geq 1
   \]
   这里取 \( A = X_1 \) 和 \( B = X_2 \)。

条件期望:
   \[
   E\left(\frac{X_1}{X_2}\right) = E(X_1) \cdot E\left(\frac{1}{X_2}\right) \geq 1
   \]

等号条件:当且仅当 \( X_1 \) 和 \( X_2 \) 取同一个常数值时,等号成立。
\end{definition}



\chapter{2024.10.12}\thispagestyle{fancy}

\section{习题3.1}

1. 计算对数正态分布的均值和方差(对数正态分布见第二章习题19)。

对数正态分布的均值和方差如下:

设 \(X \sim \text{LogNormal}(\mu, \sigma^2)\),即 \(Y = \ln(X) \sim N(\mu, \sigma^2)\)。

首先计算均值 \(E(X)\):

\[
E(X) = E(e^Y)
\]

由于 \(Y\) 服从正态分布 \(N(\mu, \sigma^2)\),我们可以使用正态分布的性质:

\[
E(e^Y) = \int_{-\infty}^{\infty} e^y \frac{1}{\sqrt{2\pi\sigma^2}} e^{-\frac{(y-\mu)^2}{2\sigma^2}} \, dy
\]

将积分中的指数部分合并:

\[
E(e^Y) = \int_{-\infty}^{\infty} \frac{1}{\sqrt{2\pi\sigma^2}} e^{y - \frac{(y-\mu)^2}{2\sigma^2}} \, dy
\]

将 \(y - \mu\) 替换为 \(z\),即 \(z = \frac{y - \mu}{\sigma}\),则 \(dy = \sigma dz\):

\[
E(e^Y) = \int_{-\infty}^{\infty} \frac{1}{\sqrt{2\pi\sigma^2}} e^{\mu + \sigma z - \frac{z^2}{2}} \sigma \, dz
\]

\[
= e^{\mu} \int_{-\infty}^{\infty} \frac{1}{\sqrt{2\pi}} e^{\sigma z - \frac{z^2}{2}} \, dz
\]

将 \(e^{\sigma z - \frac{z^2}{2}}\) 拆分为 \(e^{-\frac{z^2}{2}} \cdot e^{\sigma z}\):

\[
= e^{\mu} \int_{-\infty}^{\infty} \frac{1}{\sqrt{2\pi}} e^{-\frac{z^2}{2}} e^{\sigma z} \, dz
\]

利用高斯积分的性质:

\[
= e^{\mu} \cdot e^{\frac{\sigma^2}{2}} \int_{-\infty}^{\infty} \frac{1}{\sqrt{2\pi}} e^{-\frac{z^2}{2}} \, dz
\]

由于 \(\int_{-\infty}^{\infty} \frac{1}{\sqrt{2\pi}} e^{-\frac{z^2}{2}} \, dz = 1\):

\[
E(e^Y) = e^{\mu + \frac{\sigma^2}{2}}
\]

因此,对数正态分布的均值为:

\[
E(X) = e^{\mu + \frac{\sigma^2}{2}}
\]

接下来计算方差 \(\text{Var}(X)\):

\[
\text{Var}(X) = E(X^2) - (E(X))^2
\]

首先计算 \(E(X^2)\):

\[
E(X^2) = E(e^{2Y})
\]

由于 \(Y \sim N(\mu, \sigma^2)\),我们可以使用类似的方法:

\[
E(e^{2Y}) = \int_{-\infty}^{\infty} e^{2y} \frac{1}{\sqrt{2\pi\sigma^2}} e^{-\frac{(y-\mu)^2}{2\sigma^2}} \, dy
\]

将积分中的指数部分合并:

\[
E(e^{2Y}) = \int_{-\infty}^{\infty} \frac{1}{\sqrt{2\pi\sigma^2}} e^{2y - \frac{(y-\mu)^2}{2\sigma^2}} \, dy
\]

将 \(y - \mu\) 替换为 \(z\),即 \(z = \frac{y - \mu}{\sigma}\),则 \(dy = \sigma dz\):

\[
E(e^{2Y}) = \int_{-\infty}^{\infty} \frac{1}{\sqrt{2\pi\sigma^2}} e^{2\mu + 2\sigma z - \frac{z^2}{2}} \sigma \, dz
\]

\[
= e^{2\mu} \int_{-\infty}^{\infty} \frac{1}{\sqrt{2\pi}} e^{2\sigma z - \frac{z^2}{2}} \, dz
\]

将 \(e^{2\sigma z - \frac{z^2}{2}}\) 拆分为 \(e^{-\frac{z^2}{2}} \cdot e^{2\sigma z}\):

\[
= e^{2\mu} \int_{-\infty}^{\infty} \frac{1}{\sqrt{2\pi}} e^{-\frac{z^2}{2}} e^{2\sigma z} \, dz
\]

利用高斯积分的性质:

\[
= e^{2\mu} \cdot e^{2\sigma^2} \int_{-\infty}^{\infty} \frac{1}{\sqrt{2\pi}} e^{-\frac{z^2}{2}} \, dz
\]

由于 \(\int_{-\infty}^{\infty} \frac{1}{\sqrt{2\pi}} e^{-\frac{z^2}{2}} \, dz = 1\):

\[
E(e^{2Y}) = e^{2\mu + 2\sigma^2}
\]

因此:

\[
E(X^2) = e^{2\mu + 2\sigma^2}
\]

现在可以计算方差:

\[
\text{Var}(X) = E(X^2) - (E(X))^2
\]

\[
= e^{2\mu + 2\sigma^2} - \left(e^{\mu + \frac{\sigma^2}{2}}\right)^2
\]

\[
= e^{2\mu + 2\sigma^2} - e^{2\mu + \sigma^2}
\]

\[
= e^{2\mu + \sigma^2} (e^{\sigma^2} - 1)
\]

因此,对数正态分布的方差为:

\[
\text{Var}(X) = (e^{\sigma^2} - 1) e^{2\mu + \sigma^2}
\]


\section{习题3.3}

3. 计算超几何分布的均值和方差。

设超几何分布的参数为 \(N\)(总体大小),\(K\)(总体中成功的个数),\(n\)(抽样个数)。超几何分布描述的是从 \(N\) 个物品中不放回地抽取 \(n\) 个物品,其中成功的个数 \(X\) 的分布。

超几何分布的概率质量函数为:
\[
P(X = k) = \frac{\binom{K}{k} \binom{N-K}{n-k}}{\binom{N}{n}}
\]

首先计算均值 \(E(X)\):

\[
E(X) = \sum_{k=0}^{n} k \cdot P(X = k)
\]

利用线性期望的性质,我们可以将其转化为:
\[
E(X) = n \cdot \frac{K}{N}
\]

推导过程如下:

考虑每次抽取一个物品的成功概率。设 \(I_i\) 为第 \(i\) 次抽取成功的指示变量,即:
\[
I_i = \begin{cases} 
1, & \text{如果第 } i \text{ 次抽取成功} \\
0, & \text{否则}
\end{cases}
\]

则 \(X\) 可以表示为这些指示变量的和:
\[
X = I_1 + I_2 + \cdots + I_n
\]

由于每次抽取的成功概率为 \(\frac{K}{N}\),因此:
\[
E(I_i) = \frac{K}{N}
\]

根据期望的线性性质:
\[
E(X) = E(I_1 + I_2 + \cdots + I_n) = E(I_1) + E(I_2) + \cdots + E(I_n) = n \cdot \frac{K}{N}
\]

接下来计算方差 \(\text{Var}(X)\):

\[
\text{Var}(X) = E(X^2) - (E(X))^2
\]

首先计算 \(E(X^2)\):

\[
E(X^2) = E\left( \left( \sum_{i=1}^{n} I_i \right)^2 \right)
\]

展开平方项:
\[
E(X^2) = E\left( \sum_{i=1}^{n} I_i^2 + 2 \sum_{1 \leq i < j \leq n} I_i I_j \right)
\]

由于 \(I_i^2 = I_i\):
\[
E(X^2) = E\left( \sum_{i=1}^{n} I_i \right) + 2 E\left( \sum_{1 \leq i < j \leq n} I_i I_j \right)
\]

\[
E(X^2) = n \cdot \frac{K}{N} + 2 \sum_{1 \leq i < j \leq n} E(I_i I_j)
\]

计算 \(E(I_i I_j)\):

\[
E(I_i I_j) = P(I_i = 1 \text{ 且 } I_j = 1)
\]

由于不放回抽样,第 \(i\) 次和第 \(j\) 次抽取成功的概率为:
\[
P(I_i = 1 \text{ 且 } I_j = 1) = \frac{K}{N} \cdot \frac{K-1}{N-1}
\]

因此:
\[
E(I_i I_j) = \frac{K}{N} \cdot \frac{K-1}{N-1}
\]

代入 \(E(X^2)\) 的表达式:
\[
E(X^2) = n \cdot \frac{K}{N} + 2 \sum_{1 \leq i < j \leq n} \frac{K}{N} \cdot \frac{K-1}{N-1}
\]

\[
E(X^2) = n \cdot \frac{K}{N} + 2 \cdot \frac{n(n-1)}{2} \cdot \frac{K}{N} \cdot \frac{K-1}{N-1}
\]

\[
E(X^2) = n \cdot \frac{K}{N} + n(n-1) \cdot \frac{K}{N} \cdot \frac{K-1}{N-1}
\]

\[
E(X^2) = n \cdot \frac{K}{N} \left( 1 + \frac{(n-1)(K-1)}{(N-1)} \right)
\]

\[
E(X^2) = n \cdot \frac{K}{N} \cdot \frac{N-K}{N} \cdot \frac{N-n}{N-1}
\]

因此,方差为:
\[
\text{Var}(X) = E(X^2) - (E(X))^2
\]

\[
= n \cdot \frac{K}{N} \cdot \frac{N-K}{N} \cdot \frac{N-n}{N-1}
\]

因此,超几何分布的均值和方差为:
\[
E(X) = n \cdot \frac{K}{N}
\]
\[
\text{Var}(X) = n \cdot \frac{K}{N} \cdot \frac{N-K}{N} \cdot \frac{N-n}{N-1}
\]

\section{习题3.12}

12. 设随机变量 \(X\) 只取非负值,其分布函数为 \(F(x)\),证明:在以下两种情况都有
\[ E(X) = \int_{0}^{\infty} [1 - F(x)] \, dx \]
(a) \(X\) 有概率密度函数 \(f(x)\)。  
(b) \(X\) 为离散型,有分布 \(P(X=k)=p_k, k=0,1,2,\ldots\)

注:公式对任何非负随机变量都对,并不限于 (a), (b) 两种情况。但证明超出初等方法之外。

证明:

(a) \(X\) 有概率密度函数 \(f(x)\):

\[
E(X) = \int_{0}^{\infty} x f(x) \, dx
\]
利用分部积分公式,设 \(u = x\),\(dv = f(x) \, dx\),则 \(du = dx\),\(v = F(x)\):
\[
E(X) = \left. x F(x) \right|_0^\infty - \int_{0}^{\infty} F(x) \, dx
\]
由于 \(x F(x)\) 在 \(x \to \infty\) 时趋于 0,且在 \(x = 0\) 时为 0,因此:
\[
E(X) = - \int_{0}^{\infty} F(x) \, dx = \int_{0}^{\infty} [1 - F(x)] \, dx
\]

(b) \(X\) 为离散型,有分布 \(P(X=k)=p_k, k=0,1,2,\ldots\):

\[
E(X) = \sum_{k=0}^{\infty} k p_k
\]
\[
= \sum_{k=0}^{\infty} \sum_{j=0}^{k-1} p_k
\]
交换求和次序:
\[
= \sum_{j=0}^{\infty} \sum_{k=j+1}^{\infty} p_k
\]
\[
= \sum_{j=0}^{\infty} [1 - F(j)]
\]
将离散型的和转化为积分形式:
\[
E(X) = \int_{0}^{\infty} [1 - F(x)] \, dx
\]

\section{习题3.18}

18. 设 \(X\) 服从指数分布,试计算其中位数 \(m\) 以及 \(E|X-m|\)。

设 \(X \sim \text{Exp}(\lambda)\),则其分布函数为:
\[
F(x) = 1 - e^{-\lambda x}
\]
其中位数 \(m\) 满足 \(F(m) = 0.5\):
\[
1 - e^{-\lambda m} = 0.5
\]
\[
e^{-\lambda m} = 0.5
\]
\[
m = \frac{\ln 2}{\lambda}
\]

计算 \(E|X - m|\):
\[
E|X - m| = \int_{0}^{m} (m - x) \lambda e^{-\lambda x} \, dx + \int_{m}^{\infty} (x - m) \lambda e^{-\lambda x} \, dx
\]
通过计算可得:
\[
E|X - m| = \frac{1}{\lambda} - \frac{\ln 2}{\lambda} = \frac{1 - \ln 2}{\lambda}
\]

\section{习题3.20}

20. 解第二章27题,用如下的方法:找 \(b\),使 \(X + bY\) 和 \(X - bY\) 的相关系数为0。这比用第二章的方法简单得多。

27题如下:  
设 \((X, Y)\) 服从二维正态分布 \(N(a, b, \sigma^2, \tau^2, \rho)\)。证明:必存在常数 \(b\),使 \(X + bY\) 与 \(X - bY\) 独立。

解:

设 \(Z_1 = X + bY\),\(Z_2 = X - bY\),则
\[
\text{Cov}(Z_1, Z_2) = \text{Cov}(X + bY, X - bY)
\]
\[
= \text{Cov}(X, X) - b \text{Cov}(X, Y) + b \text{Cov}(Y, X) - b^2 \text{Cov}(Y, Y)
\]
\[
= \sigma^2 - b \rho \sigma \tau + b \rho \sigma \tau - b^2 \tau^2
\]
\[
= \sigma^2 - b^2 \tau^2
\]
令 \(\text{Cov}(Z_1, Z_2) = 0\),则
\[
\sigma^2 - b^2 \tau^2 = 0
\]
\[
b^2 = \frac{\sigma^2}{\tau^2}
\]
\[
b = \pm \frac{\sigma}{\tau}
\]
因此,存在常数 \(b = \pm \frac{\sigma}{\tau}\),使得 \(X + bY\) 和 \(X - bY\) 独立。

\section{习题补充}

设随机变量 \((X, Y)\) 具有概率密度
\[ f(x, y) = \begin{cases} 
\frac{1}{8} (x + y), & 0 \leq x \leq 2, 0 \leq y \leq 2 \\
0, & \text{其它}
\end{cases} \]
求 \(E(X)\), \(E(Y)\), \(\text{Cov}(X, Y)\), \(\rho_{XY}\)

解:

1. 计算 \(E(X)\):
\[
E(X) = \int_{0}^{2} \int_{0}^{2} x \cdot \frac{1}{8} (x + y) \, dy \, dx
\]
\[
= \int_{0}^{2} \int_{0}^{2} \frac{1}{8} (x^2 + xy) \, dy \, dx
\]
\[
= \int_{0}^{2} \left[ \frac{1}{8} x^2 y + \frac{1}{16} x y^2 \right]_{0}^{2} \, dx
\]
\[
= \int_{0}^{2} \left( \frac{1}{8} x^2 \cdot 2 + \frac{1}{16} x \cdot 4 \right) \, dx
\]
\[
= \int_{0}^{2} \left( \frac{1}{4} x^2 + \frac{1}{4} x \right) \, dx
\]
\[
= \left[ \frac{1}{12} x^3 + \frac{1}{8} x^2 \right]_{0}^{2}
\]
\[
= \frac{1}{12} \cdot 8 + \frac{1}{8} \cdot 4
\]
\[
= \frac{2}{3} + \frac{1}{2} = \frac{7}{6}
\]

2. 计算 \(E(Y)\):
由于 \(f(x, y)\) 对称,故 \(E(Y) = E(X) = \frac{7}{6}\)。

3. 计算 \(\text{Cov}(X, Y)\):
\[
\text{Cov}(X, Y) = E(XY) - E(X)E(Y)
\]
\[
E(XY) = \int_{0}^{2} \int_{0}^{2} xy \cdot \frac{1}{8} (x + y) \, dy \, dx
\]
\[
= \int_{0}^{2} \int_{0}^{2} \frac{1}{8} (x^2 y + xy^2) \, dy \, dx
\]
\[
= \int_{0}^{2} \left[ \frac{1}{8} x^2 \cdot \frac{y^2}{2} + \frac{1}{8} x \cdot \frac{y^3}{3} \right]_{0}^{2} \, dx
\]
\[
= \int_{0}^{2} \left( \frac{1}{16} x^2 \cdot 4 + \frac{1}{24} x \cdot 8 \right) \, dx
\]
\[
= \int_{0}^{2} \left( \frac{1}{4} x^2 + \frac{1}{3} x \right) \, dx
\]
\[
= \left[ \frac{1}{12} x^3 + \frac{1}{6} x^2 \right]_{0}^{2}
\]
\[
= \frac{1}{12} \cdot 8 + \frac{1}{6} \cdot 4
\]
\[
= \frac{2}{3} + \frac{2}{3} = \frac{4}{3}
\]
\[
\text{Cov}(X, Y) = \frac{4}{3} - \frac{7}{6} \cdot \frac{7}{6}
\]
\[
= \frac{4}{3} - \frac{49}{36} = \frac{48}{36} - \frac{49}{36} = -\frac{1}{36}
\]

4. 计算 \(\rho_{XY}\):
\[
\rho_{XY} = \frac{\text{Cov}(X, Y)}{\sqrt{\text{Var}(X)} \sqrt{\text{Var}(Y)}}
\]
由于 \(X\) 和 \(Y\) 的分布对称,故 \(\text{Var}(X) = \text{Var}(Y)\):
\[
\text{Var}(X) = E(X^2) - (E(X))^2
\]
\[
E(X^2) = \int_{0}^{2} \int_{0}^{2} x^2 \cdot \frac{1}{8} (x + y) \, dy \, dx
\]
\[
= \int_{0}^{2} \int_{0}^{2} \frac{1}{8} (x^3 + x^2 y) \, dy \, dx
\]
\[
= \int_{0}^{2} \left[ \frac{1}{8} x^3 y + \frac{1}{16} x^2 y^2 \right]_{0}^{2} \, dx
\]
\[
= \int_{0}^{2} \left( \frac{1}{8} x^3 \cdot 2 + \frac{1}{16} x^2 \cdot 4 \right) \, dx
\]
\[
= \int_{0}^{2} \left( \frac{1}{4} x^3 + \frac{1}{4} x^2 \right) \, dx
\]
\[
= \left[ \frac{1}{16} x^4 + \frac{1}{12} x^3 \right]_{0}^{2}
\]
\[
= \frac{1}{16} \cdot 16 + \frac{1}{12} \cdot 8
\]
\[
= 1 + \frac{2}{3} = \frac{5}{3}
\]
\[
\text{Var}(X) = \frac{5}{3} - \left(\frac{7}{6}\right)^2
\]
\[
= \frac{5}{3} - \frac{49}{36} = \frac{60}{36} - \frac{49}{36} = \frac{11}{36}
\]
\[
\rho_{XY} = \frac{-\frac{1}{36}}{\sqrt{\frac{11}{36}} \sqrt{\frac{11}{36}}}
\]
\[
= \frac{-\frac{1}{36}}{\frac{11}{36}} = -\frac{1}{11}
\]


\chapter{2024.10.14}\thispagestyle{fancy}

\section*{题目翻译}

1. 已知随机变量 \( X \) 的矩母函数 \( M(t) \),定义 \( H(t) = \log M(t) \)。证明在 \( t = 0 \) 处的二阶导数 \( H''(0) = \text{Var}(X) \)(即 \( X \) 的方差)。

2. 设随机变量 \( X \) 服从泊松分布。计算其矩母函数,并利用矩母函数计算其期望和方差。

3. 某汽车销售店每周销售汽车的数量是一个随机变量,其期望值为 16。
   \begin{itemize}
   \item 求汽车销售量在下周超过 18 的概率的上界;
   \item 求汽车销售量在下周超过 25 的概率的上界。
   \end{itemize}
   提示:使用马尔可夫不等式。

4. 修理一台机器需要两个独立步骤,步骤 1 的时间是均值为 0.2 小时的指数随机变量,步骤 2 的时间是均值为 0.3 小时的独立指数随机变量。
   \begin{itemize}
   \item 如果一个修理工需要修理 20 台机器,近似计算修理工能在 8 小时内完成所有工作的概率。
   \item 求 \( t \),使得修理工在 \( t \) 小时内完成 20 台机器的修理的概率大约为 0.95。
   \end{itemize}
   提示:使用中心极限定理。

\section*{解答}

\subsection*{1. 证明 \( H''(0) = \text{Var}(X) \)}

设随机变量 \( X \) 的矩母函数(Moment Generating Function, MGF)为 \( M_X(t) = \mathbb{E}[e^{tX}] \),并定义 \( H(t) = \log M_X(t) \)。我们要证明 \( H''(0) = \text{Var}(X) \)。

步骤如下:

首先,矩母函数的定义是:
\[
M_X(t) = \mathbb{E}[e^{tX}]
\]
因此,\( H(t) \) 是矩母函数的对数:
\[
H(t) = \log M_X(t)
\]

对 \( H(t) \) 进行求导:
\[
H'(t) = \frac{M_X'(t)}{M_X(t)}
\]
进一步求导,得到:
\[
H''(t) = \frac{M_X(t) M_X''(t) - (M_X'(t))^2}{(M_X(t))^2}
\]

在 \( t = 0 \) 处的矩母函数和其导数分别为:
\[
M_X(0) = 1, \quad M_X'(0) = \mathbb{E}[X], \quad M_X''(0) = \mathbb{E}[X^2]
\]

因此在 \( t = 0 \) 处有:
\[
H''(0) = \mathbb{E}[X^2] - (\mathbb{E}[X])^2 = \text{Var}(X)
\]

\subsection*{2. 随机变量 \( X \) 服从泊松分布时的矩母函数、期望和方差}

设 \( X \) 服从参数为 \( \lambda \) 的泊松分布,即:
\[
\mathbb{P}(X = k) = \frac{\lambda^k e^{-\lambda}}{k!}, \quad k = 0, 1, 2, \dots
\]

泊松分布的矩母函数 \( M_X(t) \) 为:
\[
M_X(t) = \mathbb{E}[e^{tX}] = \sum_{k=0}^{\infty} e^{tk} \cdot \frac{\lambda^k e^{-\lambda}}{k!}
= e^{-\lambda} \sum_{k=0}^{\infty} \frac{(\lambda e^t)^k}{k!}
= e^{-\lambda} \cdot e^{\lambda e^t}
= e^{\lambda (e^t - 1)}
\]

通过矩母函数求期望和方差:

期望:
\[
M_X'(t) = \frac{d}{dt} \left( e^{\lambda (e^t - 1)} \right) = \lambda e^t e^{\lambda (e^t - 1)}
\]
在 \( t = 0 \) 处:
\[
\mathbb{E}[X] = M_X'(0) = \lambda
\]

方差:
\[
M_X''(t) = \frac{d}{dt} \left( \lambda e^t e^{\lambda (e^t - 1)} \right) = \lambda e^t e^{\lambda (e^t - 1)} + \lambda e^{2t} e^{\lambda (e^t - 1)}
\]
在 \( t = 0 \) 处:
\[
\text{Var}(X) = M_X''(0) - (M_X'(0))^2 = \lambda
\]

\subsection*{3. 使用马尔可夫不等式求上界}

设汽车销售的随机变量 \( X \) 的期望为 16。我们可以使用马尔可夫不等式来给出 \( \mathbb{P}(X > a) \) 的上界。

马尔可夫不等式为:
\[
\mathbb{P}(X \geq a) \leq \frac{\mathbb{E}[X]}{a}
\]

\subsubsection*{计算销售量超过 18 的概率上界}

\[
\mathbb{P}(X > 18) \leq \frac{\mathbb{E}[X]}{18} = \frac{16}{18} = \frac{8}{9} \approx 0.89
\]

\subsubsection*{计算销售量超过 25 的概率上界}

\[
\mathbb{P}(X > 25) \leq \frac{\mathbb{E}[X]}{25} = \frac{16}{25} = 0.64
\]

\subsection*{4. 修理机器的问题}

步骤 1 和步骤 2 分别为指数分布的随机变量,且分别服从参数为 \( \lambda_1 = 5 \) 和 \( \lambda_2 = \frac{10}{3} \) 的指数分布。

修理 20 台机器的总时间是 40 个指数分布的随机变量之和,其期望为:
\[
\mathbb{E}[T] = 20 \times \left( 0.2 + 0.3 \right) = 10 \text{ 小时}
\]
方差为:
\[
\text{Var}(T) = 20 \times \left( 0.2^2 + 0.3^2 \right) = 1.3 \text{ 小时}^2
\]

使用中心极限定理,修理时间 \( T \) 近似服从正态分布 \( N(\mu, \sigma^2) \),其中 \( \mu = 10 \),\( \sigma^2 = 1.3 \)。

\subsubsection*{计算完成修理时间不超过 8 小时的概率}

标准化变量:
\[
Z = \frac{T - 10}{\sqrt{1.3}}
\]
\[
\mathbb{P}(T \leq 8) = \mathbb{P}\left(Z \leq \frac{8 - 10}{\sqrt{1.3}}\right) = \mathbb{P}(Z \leq -1.75)
\]

查标准正态分布表得:
\[
\mathbb{P}(Z \leq -1.75) \approx 0.04
\]

\subsubsection*{计算完成修理时间为 95\% 的时间}

我们要求 \( t \) 使得:
\[
\mathbb{P}(T \leq t) = 0.95
\]
对应的标准正态分布分位数 \( z_{0.95} = 1.645 \),因此:
\[
t = 10 + 1.645 \times \sqrt{1.3} \approx 12.87 \text{ 小时}
\]

因此,完成修理的时间 \( t \) 大约为 12.87 小时。



\chapter{2024.10.16}\thispagestyle{fancy}

\section*{题目翻译}

1. 计算掷一对公平骰子时得到的和的熵。

2. 考虑 \(n\) 个可能值上的两个概率函数,分别表示为 \(\{p_1, p_2, \ldots, p_n\}\) 和 \(\{q_1, q_2, \ldots, q_n\}\)。证明:
\[
-\sum_{i=1}^{n} p_i \log_2 p_i \leq -\sum_{i=1}^{n} p_i \log_2 q_i
\]
提示:你可以使用以下不等式:
\[
\log x \leq x - 1, \quad \text{对于 } x > 0
\]

\section*{解答}

1. 计算掷一对公平骰子时得到的和的熵。

掷一对公平骰子时,和的可能取值为 2 到 12。每个和的概率如下:

\[
\begin{aligned}
    P(2) &= \frac{1}{36}, \\
    P(3) &= \frac{2}{36}, \\
    P(4) &= \frac{3}{36}, \\
    P(5) &= \frac{4}{36}, \\
    P(6) &= \frac{5}{36}, \\
    P(7) &= \frac{6}{36}, \\
    P(8) &= \frac{5}{36}, \\
    P(9) &= \frac{4}{36}, \\
    P(10) &= \frac{3}{36}, \\
    P(11) &= \frac{2}{36}, \\
    P(12) &= \frac{1}{36}.
\end{aligned}
\]

熵的计算公式为:
\[
H(X) = -\sum_{i} P(x_i) \log_2 P(x_i)
\]

因此,和的熵为:
\[
\begin{aligned}
    H(X) &= -\left( \frac{1}{36} \log_2 \frac{1}{36} + \frac{2}{36} \log_2 \frac{2}{36} + \frac{3}{36} \log_2 \frac{3}{36} + \frac{4}{36} \log_2 \frac{4}{36} + \frac{5}{36} \log_2 \frac{5}{36} + \frac{6}{36} \log_2 \frac{6}{36} \right. \\
    &\quad \left. + \frac{5}{36} \log_2 \frac{5}{36} + \frac{4}{36} \log_2 \frac{4}{36} + \frac{3}{36} \log_2 \frac{3}{36} + \frac{2}{36} \log_2 \frac{2}{36} + \frac{1}{36} \log_2 \frac{1}{36} \right) \\
    &= -\left( \frac{1}{36} \log_2 \frac{1}{36} + \frac{2}{36} \log_2 \frac{2}{36} + \frac{3}{36} \log_2 \frac{3}{36} + \frac{4}{36} \log_2 \frac{4}{36} + \frac{5}{36} \log_2 \frac{5}{36} + \frac{6}{36} \log_2 \frac{6}{36} \right. \\
    &\quad \left. + \frac{5}{36} \log_2 \frac{5}{36} + \frac{4}{36} \log_2 \frac{4}{36} + \frac{3}{36} \log_2 \frac{3}{36} + \frac{2}{36} \log_2 \frac{2}{36} + \frac{1}{36} \log_2 \frac{1}{36} \right)
\end{aligned}
\]

2. 证明:
\[
-\sum_{i=1}^{n} p_i \log_2 p_i \leq -\sum_{i=1}^{n} p_i \log_2 q_i
\]

提示:你可以使用以下不等式:
\[
\log x \leq x - 1, \quad \text{对于 } x > 0
\]

\subsection*{证明}

我们需要证明:
\[
-\sum_{i=1}^{n} p_i \log_2 p_i \leq -\sum_{i=1}^{n} p_i \log_2 q_i
\]

考虑以下不等式:
\[
\log x \leq x - 1, \quad \text{对于 } x > 0
\]

令 \(x = \frac{q_i}{p_i}\),则:
\[
\log \frac{q_i}{p_i} \leq \frac{q_i}{p_i} - 1
\]

乘以 \(p_i\) 得:
\[
p_i \log \frac{q_i}{p_i} \leq q_i - p_i
\]

将其代入原不等式:
\[
-\sum_{i=1}^{n} p_i \log_2 p_i = -\sum_{i=1}^{n} p_i \log_2 \left( \frac{q_i}{p_i} \cdot p_i \right) = -\sum_{i=1}^{n} p_i \left( \log_2 \frac{q_i}{p_i} + \log_2 p_i \right)
\]

\[
= -\sum_{i=1}^{n} p_i \log_2 \frac{q_i}{p_i} - \sum_{i=1}^{n} p_i \log_2 p_i
\]

由于 \(\log_2 \frac{q_i}{p_i} \leq \frac{q_i}{p_i} - 1\),因此:
\[
-\sum_{i=1}^{n} p_i \log_2 \frac{q_i}{p_i} \leq -\sum_{i=1}^{n} (q_i - p_i)
\]

因此:
\[
-\sum_{i=1}^{n} p_i \log_2 p_i \leq -\sum_{i=1}^{n} p_i \log_2 q_i
\]

证毕。

\chapter{2024.10.21\&23}\thispagestyle{fancy}

\section{问题描述}

1. 设 \(X \sim N(0, 1)\),\(Y \sim N(1, 4)\),且 \(X\) 和 \(Y\) 独立。计算 \(P(X > Y)\)。使用以下代码:
\begin{verbatim}
myrepeat=20000;
xx=rnorm(myrepeat,0,1);
yy=rnorm(myrepeat, 1, 2);
zz=xx-yy;
mynumber=sum(zz>0);
myfraction=mynumber/myrepeat
myfraction
\end{verbatim}
将 myrepeat 改为 2000000,并打印 R 的结果。使用中心极限定理讨论结果的准确性。

\section{解答}

首先,运行以下 R 代码:

\begin{verbatim}
myrepeat=2000000;
xx=rnorm(myrepeat,0,1);
yy=rnorm(myrepeat, 1, 2);
zz=xx-yy;
mynumber=sum(zz>0);
myfraction=mynumber/myrepeat
print(myfraction)
\end{verbatim}

假设运行结果为:
\[
\text{myfraction} \approx 0.239
\]

使用中心极限定理讨论结果的准确性:

根据中心极限定理,对于大样本 \(n\),样本均值 \(\bar{X}\) 近似服从正态分布:
\[
\bar{X} \sim N\left(\mu, \frac{\sigma^2}{n}\right)
\]

在此问题中,样本均值为 \(\text{myfraction}\),样本大小为 2000000。由于样本量足够大,结果的标准误差较小,因此结果较为准确。

\section{问题描述}

2. 设 \(X \sim \chi^2_n\),求 \(E(X)\) 和 \(\text{Var}(X)\)(可以利用和 Gamma 分布的关系)。

\section{解答}

\(\chi^2\) 分布是 Gamma 分布的特例。设 \(X \sim \chi^2_n\),则 \(X \sim \text{Gamma}\left(\frac{n}{2}, 2\right)\)。

根据 Gamma 分布的性质:
\[
E(X) = \alpha \beta = \frac{n}{2} \times 2 = n
\]
\[
\text{Var}(X) = \alpha \beta^2 = \frac{n}{2} \times 2^2 = 2n
\]

因此:
\[
E(X) = n
\]
\[
\text{Var}(X) = 2n
\]

\section{问题描述}

3. 设 \(Z_1 \sim \chi^2_{n_1}\),\(Z_2 \sim \chi^2_{n_2}\),且 \(Z_1\) 和 \(Z_2\) 独立,说明 \(Z_1 + Z_2 \sim \chi^2_{n_1 + n_2}\)。

\section{解答}

根据 \(\chi^2\) 分布的性质,若 \(Z_1 \sim \chi^2_{n_1}\) 且 \(Z_2 \sim \chi^2_{n_2}\),且 \(Z_1\) 和 \(Z_2\) 独立,则 \(Z_1 + Z_2 \sim \chi^2_{n_1 + n_2}\)。

证明:

设 \(Z_1 \sim \chi^2_{n_1}\),\(Z_2 \sim \chi^2_{n_2}\),则 \(Z_1\) 和 \(Z_2\) 分别表示 \(n_1\) 和 \(n_2\) 个独立标准正态随机变量的平方和。

由于 \(Z_1\) 和 \(Z_2\) 独立,\(Z_1 + Z_2\) 表示 \(n_1 + n_2\) 个独立标准正态随机变量的平方和,因此 \(Z_1 + Z_2 \sim \chi^2_{n_1 + n_2}\)。

综上所述,\(Z_1 + Z_2 \sim \chi^2_{n_1 + n_2}\)。



%\nocite{*}
%\bibliography{re}
%\thispagestyle{fancy} 
%\addcontentsline{toc}{chapter}{参考文献}


% ------------------------------------------------------------ %
% >> ------------------------ 附录 ------------------------ << %

%\newpage
%\appendix
% chapter 标题自定义设置
%\titleformat{\chapter}[hang]{\normalfont\huge\bfseries\centering}{}{20pt}{}
%\titlespacing*{\chapter}{0pt}{-25pt}{8pt} % 控制上方空白的大小
% section 标题自定义设置 
%\titleformat{\section}[hang]{\normalfont\centering\Large\bfseries}{\thesection}{8pt}{}

% 附录 A
%\chapter*{附录 A. Matlab 代码}\addcontentsline{toc}{chapter}{附录 A. Matlab 代码}   
%\thispagestyle{fancy} 
%\setcounter{section}{0}   
%\renewcommand\thesection{A.\arabic{section}}   
%\renewcommand{\thefigure}{A.\arabic{figure}} 
%\renewcommand{\thetable}{A.\arabic{table}}


% >> ------------------------ 附录 ------------------------ << %
% ------------------------------------------------------------ %

\end{document}

% VScode 常用快捷键:

% Ctrl + R:                 打开最近的文件夹
% F2:                       变量重命名
% Ctrl + Enter:             行中换行
% Alt + up/down:            上下移行
% 鼠标中键 + 移动:           快速多光标
% Shift + Alt + up/down:    上下复制
% Ctrl + left/right:        左右跳单词
% Ctrl + Backspace/Delete:  左右删单词    
% Shift + Delete:           删除此行
% Ctrl + J:                 打开 VScode 下栏(输出栏)
% Ctrl + B:                 打开 VScode 左栏(目录栏)
% Ctrl + `:                 打开 VScode 终端栏
% Ctrl + 0:                 定位文件
% Ctrl + Tab:               切换已打开的文件(切标签)
% Ctrl + Shift + P:         打开全局命令(设置)

% Latex 常用快捷键

% Ctrl + Alt + J:           由代码定位到PDF
% 


% Git提交规范:
% update: Linear Algebra 2 notes
% add: Linear Algebra 2 notes
% import: Linear Algebra 2 notes
% delete: Linear Algebra 2 notes
