% 若编译失败,且生成 .synctex(busy) 辅助文件,可能有两个原因:
% 1. 需要插入的图片不存在:Ctrl + F 搜索 'figure' 将这些代码注释/删除掉即可
% 2. 路径/文件名含中文或空格:更改路径/文件名即可

% ------------------------------------------------------------- %
% >> ------------------ 文章宏包及相关设置 ------------------ << %
% 设定文章类型与编码格式
    \documentclass[UTF8]{report}		

% 本 .tex 专属的宏定义
    \def\V{\ \mathrm{V}}
    \def\mV{\ \mathrm{mV}}
    \def\kV{\ \mathrm{KV}}
    \def\KV{\ \mathrm{KV}}
    \def\MV{\ \mathrm{MV}}
    \def\A{\ \mathrm{A}}
    \def\mA{\ \mathrm{mA}}
    \def\kA{\ \mathrm{KA}}
    \def\KA{\ \mathrm{KA}}
    \def\MA{\ \mathrm{MA}}
    \def\O{\ \Omega}
    \def\mO{\ \Omega}
    \def\kO{\ \mathrm{K}\Omega}
    \def\KO{\ \mathrm{K}\Omega}
    \def\MO{\ \mathrm{M}\Omega}
    \def\Hz{\ \mathrm{Hz}}

% 自定义宏定义
    \def\N{\mathbb{N}}
    \def\F{\mathbb{F}}
    \def\Z{\mathbb{Z}}
    \def\Q{\mathbb{Q}}
    \def\R{\mathbb{R}}
    \def\C{\mathbb{C}}
    \def\T{\mathbb{T}}
    \def\S{\mathbb{S}}
    \def\A{\mathbb{A}}
    \def\I{\mathscr{I}}
    \def\Im{\mathrm{Im\,}}
    \def\Re{\mathrm{Re\,}}
    \def\d{\mathrm{d}}
    \def\p{\partial}
% 导入基本宏包
    \usepackage[UTF8]{ctex}     % 设置文档为中文语言
    \usepackage[colorlinks, linkcolor=blue, anchorcolor=blue, citecolor=blue, urlcolor=blue]{hyperref}  % 宏包:自动生成超链接 (此宏包与标题中的数学环境冲突)
    % \usepackage{docmute}    % 宏包:子文件导入时自动去除导言区,用于主/子文件的写作方式,\include{./51单片机笔记}即可。注:启用此宏包会导致.tex文件capacity受限。
    \usepackage{amsmath}    % 宏包:数学公式
    \usepackage{mathrsfs}   % 宏包:提供更多数学符号
    \usepackage{amssymb}    % 宏包:提供更多数学符号
    \usepackage{pifont}     % 宏包:提供了特殊符号和字体
    \usepackage{extarrows}  % 宏包:更多箭头符号


% 文章页面margin设置
    \usepackage[a4paper]{geometry}
        \geometry{top=1in}
        \geometry{bottom=1in}
        \geometry{left=0.75in}
        \geometry{right=0.75in}   % 设置上下左右页边距
        \geometry{marginparwidth=1.75cm}    % 设置边注距离(注释、标记等)

% 配置数学环境
    \usepackage{amsthm} % 宏包:数学环境配置
    % theorem-line 环境自定义
        \newtheoremstyle{MyLineTheoremStyle}% <name>
            {11pt}% <space above>
            {11pt}% <space below>
            {}% <body font> 使用默认正文字体
            {}% <indent amount>
            {\bfseries}% <theorem head font> 设置标题项为加粗
            {:}% <punctuation after theorem head>
            {.5em}% <space after theorem head>
            {\textbf{#1}\thmnumber{#2}\ \ (\,\textbf{#3}\,)}% 设置标题内容顺序
        \theoremstyle{MyLineTheoremStyle} % 应用自定义的定理样式
        \newtheorem{LineTheorem}{Theorem.\,}
    % theorem-block 环境自定义
        \newtheoremstyle{MyBlockTheoremStyle}% <name>
            {11pt}% <space above>
            {11pt}% <space below>
            {}% <body font> 使用默认正文字体
            {}% <indent amount>
            {\bfseries}% <theorem head font> 设置标题项为加粗
            {:\\ \indent}% <punctuation after theorem head>
            {.5em}% <space after theorem head>
            {\textbf{#1}\thmnumber{#2}\ \ (\,\textbf{#3}\,)}% 设置标题内容顺序
        \theoremstyle{MyBlockTheoremStyle} % 应用自定义的定理样式
        \newtheorem{BlockTheorem}[LineTheorem]{Theorem.\,} % 使用 LineTheorem 的计数器
    % definition 环境自定义
        \newtheoremstyle{MySubsubsectionStyle}% <name>
            {11pt}% <space above>
            {11pt}% <space below>
            {}% <body font> 使用默认正文字体
            {}% <indent amount>
            {\bfseries}% <theorem head font> 设置标题项为加粗
            {\\ \indent}% <punctuation after theorem head>
            {0pt}% <space after theorem head>
            {\textbf{#3}}% 设置标题内容顺序
        \theoremstyle{MySubsubsectionStyle} % 应用自定义的定理样式
        \newtheorem{definition}{}

%宏包:有色文本框(proof环境)及其设置
    \usepackage[dvipsnames,svgnames]{xcolor}    %设置插入的文本框颜色
    \usepackage[strict]{changepage}     % 提供一个 adjustwidth 环境
    \usepackage{framed}     % 实现方框效果
        \definecolor{graybox_color}{rgb}{0.95,0.95,0.96} % 文本框颜色。修改此行中的 rgb 数值即可改变方框纹颜色,具体颜色的rgb数值可以在网站https://colordrop.io/ 中获得。(截止目前的尝试还没有成功过,感觉单位不一样)(找到喜欢的颜色,点击下方的小眼睛,找到rgb值,复制修改即可)
        \newenvironment{graybox}{%
        \def\FrameCommand{%
        \hspace{1pt}%
        {\color{gray}\small \vrule width 2pt}%
        {\color{graybox_color}\vrule width 4pt}%
        \colorbox{graybox_color}%
        }%
        \MakeFramed{\advance\hsize-\width\FrameRestore}%
        \noindent\hspace{-4.55pt}% disable indenting first paragraph
        \begin{adjustwidth}{}{7pt}%
        \vspace{2pt}\vspace{2pt}%
        }
        {%
        \vspace{2pt}\end{adjustwidth}\endMakeFramed%
        }

% 外源代码插入设置
    % matlab 代码插入设置
    %\usepackage{matlab-prettifier}
    %    \lstset{
    %        style=Matlab-editor,  % 继承matlab代码颜色等
    %    }
    %\usepackage[most]{tcolorbox} % 引入tcolorbox包 
    %\usepackage{listings} % 引入listings包
    %    \tcbuselibrary{listings, skins, breakable}
    %    \newfontfamily\codefont{Consolas} % 定义需要的 codefont 字体
    %    \lstdefinestyle{matlabstyle}{
    %        language=Matlab,
    %        basicstyle=\small\ttfamily\codefont,    % ttfamily 确保等宽 
    %        breakatwhitespace=false,
    %        breaklines=true,
    %        captionpos=b,
    %        keepspaces=true,
    %        numbers=left,
    %        numbersep=15pt,
    %        showspaces=false,
    %        showstringspaces=false,
    %        showtabs=false,
    %        tabsize=2
    %    }
    %    \newtcblisting{matlablisting}{
    %        arc=2pt,        % 圆角半径
    %        top=-5pt,
    %        bottom=-5pt,
    %        left=1mm,
    %        listing only,
    %        listing style=matlabstyle,
    %        breakable,
    %        colback=white   % 选一个合适的颜色
    %    }

% table 支持
    %\usepackage{booktabs}   % 宏包:三线表
    %\usepackage{tabularray} % 宏包:表格排版
    %\usepackage{longtable}  % 宏包:长表格
    %\usepackage[longtable]{multirow} % 宏包:multi 行列

% figure 设置
    \usepackage{graphicx}   % 支持 jpg, png, eps, pdf 图片 
    %\usepackage{svg}        % 支持 svg 图片
    %\usepackage{subcaption} % 支持子图
    %    \svgsetup{
            % 指向 inkscape.exe 的路径
    %        inkscapeexe = C:/aa_MySame/inkscape/bin/inkscape.exe, 
            % 一定程度上修复导入后图片文字溢出几何图形的问题
    %        inkscapelatex = false                 
    %    }

% 图表进阶设置
    %\usepackage{caption}    % 图注、表注
    %    \captionsetup[figure]{name=图}  
    %    \captionsetup[table]{name=表}
    %    \captionsetup{labelfont=bf, font=small}
    %\usepackage{float}     % 图表位置浮动设置 

% 圆圈序号自定义
    \newcommand*\circled[1]{\tikz[baseline=(char.base)]{\node[shape=circle,draw,inner sep=0.8pt, line width = 0.03em] (char) {\small \bfseries #1};}}   % TikZ solution

% 列表设置
    \usepackage{enumitem}   % 宏包:列表环境设置
        \setlist[enumerate]{
            label=\bfseries(\arabic*) ,   % 设置序号样式为加粗的 (1) (2) (3)
            ref=\arabic*, % 如果需要引用列表项,这将决定引用格式(这里仍然使用数字)
            itemsep=0pt, parsep=0pt, topsep=0pt, partopsep=0pt, leftmargin=3.5em} 
        \setlist[itemize]{itemsep=0pt, parsep=0pt, topsep=0pt, partopsep=0pt, leftmargin=3.5em}
        \newlist{circledenum}{enumerate}{1} % 创建一个新的枚举环境  
        \setlist[circledenum,1]{  
            label=\protect\circled{\arabic*}, % 使用 \arabic* 来获取当前枚举计数器的值,并用 \circled 包装它  
            ref=\arabic*, % 如果需要引用列表项,这将决定引用格式(这里仍然使用数字)
            itemsep=0pt, parsep=0pt, topsep=0pt, partopsep=0pt, leftmargin=3.5em
        }  
    

% 文章默认字体设置
    \usepackage{fontspec}   % 宏包:字体设置
        \setmainfont{SimSun}    % 设置中文字体为宋体字体
        \setCJKmainfont[AutoFakeBold=3]{SimSun} % 设置加粗字体为 SimSun 族,AutoFakeBold 可以调整字体粗细
        \setmainfont{Times New Roman} % 设置英文字体为Times New Roman

% 其它设置
    % 脚注设置
    \renewcommand\thefootnote{\ding{\numexpr171+\value{footnote}}}
    % 参考文献引用设置
        \bibliographystyle{unsrt}   % 设置参考文献引用格式为unsrt
        \newcommand{\upcite}[1]{\textsuperscript{\cite{#1}}}     % 自定义上角标式引用
    % 文章序言设置
        \newcommand{\cnabstractname}{序言}
        \newenvironment{cnabstract}{%
            \par\Large
            \noindent\mbox{}\hfill{\bfseries \cnabstractname}\hfill\mbox{}\par
            \vskip 2.5ex
            }{\par\vskip 2.5ex}
% 各级标题自定义设置
    \usepackage{titlesec}   
    % chapter
        \titleformat{\chapter}[hang]{\normalfont\Large\bfseries\centering}{Homework \thechapter }{10pt}{}
        \titlespacing*{\chapter}{0pt}{-30pt}{10pt} % 控制上方空白的大小
    % section
        \titleformat{\section}[hang]{\normalfont\large\bfseries}{\thesection}{8pt}{}
    % subsection
        %\titleformat{\subsubsection}[hang]{\normalfont\bfseries}{}{8pt}{}
    % subsubsection
        %\titleformat{\subsubsection}[hang]{\normalfont\bfseries}{}{8pt}{}

% >> ------------------ 文章宏包及相关设置 ------------------ << %
% ------------------------------------------------------------- %



% ----------------------------------------------------------- %
% >> --------------------- 文章信息区 --------------------- << %
% 页眉页脚设置

\usepackage{fancyhdr}   %宏包:页眉页脚设置
    \pagestyle{fancy}
    \fancyhf{}
    \cfoot{\thepage}
    \renewcommand\headrulewidth{1pt}
    \renewcommand\footrulewidth{0pt}
    \chead{信号与系统作业,\ 尹超,\ 2023K8009926003}
    \lhead{Homework \thechapter}
    \rhead{yinchao23@mails.ucas.ac.cn}

%文档信息设置
\title{信号与系统作业\\ Homework}
\author{尹超\\ \footnotesize 中国科学院大学,北京 100049\\ Carter Yin \\ \footnotesize University of Chinese Academy of Sciences, Beijing 100049, China}
\date{\footnotesize 2024.8 -- 2025.1}
% >> --------------------- 文章信息区 --------------------- << %
% ----------------------------------------------------------- %

% 开始编辑文章

\begin{document}
\zihao{5}           % 设置全文字号大小

% --------------------------------------------------------------- %
% >> --------------------- 封面序言与目录 --------------------- << %
% 封面
    \maketitle\newpage  
    \pagenumbering{Roman} % 页码为大写罗马数字
    \thispagestyle{fancy}   % 显示页码、页眉等

% 序言
    \begin{cnabstract}\normalsize 
        本文为笔者信号与系统的作业。\par
        望老师批评指正。
    \end{cnabstract}
    \addcontentsline{toc}{chapter}{序言} % 手动添加为目录

% 不换页目录
    \setcounter{tocdepth}{0}
    \noindent\rule{\textwidth}{0.1em}   % 分割线
    \noindent\begin{minipage}{\textwidth}\centering 
        \vspace{1cm}
        \tableofcontents\thispagestyle{fancy}   % 显示页码、页眉等   
    \end{minipage}  
    \addcontentsline{toc}{chapter}{目录} % 手动添加为目录

% 收尾工作
    \newpage    
    \pagenumbering{arabic} 

% >> --------------------- 封面序言与目录 --------------------- << %
% --------------------------------------------------------------- %


\chapter{2024.09.03}\thispagestyle{fancy}
\section{习题总结}

\begin{definition}
    $ Ev \left\{\sin(4\pi t)u(t)\right\} = \frac{1}{2}\sin(4\pi t)u(t) + \frac{1}{2}\sin(-4\pi t)u(-t) = \frac{1}{2}\sin(4\pi t)u(t) $
    $ Od \left\{\sin(4\pi t)u(t)\right\} = \frac{1}{2}\sin(4\pi t)u(t) - \frac{1}{2}\sin(-4\pi t)u(-t) = \frac{1}{2}\sin(4\pi t)u(t) $
\end{definition}

\begin{definition}
    求基波周期时,先假设一个周期T,代入函数中,求解得到T,或者无解。
\end{definition}




\chapter{2024.9.5}\thispagestyle{fancy}

\section{习题总结}

\begin{definition}
    一般复指数信号
        \[
        C, a \text{ 分别写成 } C = |C|e^{j\theta}, a = r + j\omega_0
        \]
        \[
        x(t) = Ce^{at} = |C| e^{j\theta} e^{(r + j\omega_0)t} = |C| e^{rt} e^{j(\omega_0 t + \theta)} = |C| e^{rt} \cos(\omega_0 t + \theta) + j |C| e^{rt} \sin(\omega_0 t + \theta)
        \]
        \begin{itemize}
            \item $|C| e^{rt}$ 提供包络线,显示变化趋势
        \end{itemize}

        周期性质($C, \alpha$ 一般为复常数和连续时间复指数信号完全不同)
        \begin{itemize}
            \item 连续时间复指数信号 $x(t) = e^{j\omega_0 t}$ 是以 $T_0 = \frac{2\pi}{|\omega_0|}$(s)为基波周期的信号,$\omega_0$ 越大周期越短,振荡越快
            \item 在频率上,显然有 $e^{j\omega_0 n} = e^{j\omega_0 n} e^{j2k\pi n} = e^{j(\omega_0 + 2k\pi)n}$,$k = 0, \pm1, \pm2, \ldots$,因此频率是以 $2\pi$ 为周期的,低频部分在 $0$ 和 $2\pi$ 附近,高频部分在 $\pi$ 附近;只需在任意的一个 $2\pi$ 间隔内考虑 $\omega_0$,一般取 $[0, 2\pi)$ 或者 $[-\pi, \pi)$
            \item 在(时间)周期上,假设周期为 $N$,那么有
            \[
            e^{j\omega_0 n} = e^{j\omega_0 (n+N)} = e^{j\omega_0 n} e^{j\omega_0 N} \Rightarrow e^{j\omega_0 N} = 1 \Rightarrow \omega_0 N = 2\pi m \Rightarrow \frac{\omega_0}{2\pi} = \frac{m}{N}
            \]
            当 $\frac{\omega_0}{2\pi}$ 为有理数,$e^{j\omega_0 n}$ 才是周期信号,其基波频率为(假设 $N, m$ 无公因子):
            \[
            \frac{2\pi}{N} = \frac{\omega_0}{m} \quad \text{(不是 $\omega_0$)}
            \]
            基波周期为:$N = \frac{2\pi m}{|\omega_0|}$

            例:$x[n] = \cos\left(\frac{8\pi n}{31}\right)$,$\omega_0 = \frac{8\pi}{31}$,$m = 4$,$N = 31$;基波频率为:$\frac{2\pi}{31}$,样本每隔 31 个点才重复
        \end{itemize}
\end{definition}

\section{系统及基本性质}

\begin{definition}
    \begin{itemize}
        \item 连续时间系统:系统的输入和输出信号皆为连续时间信号,$x(t) \rightarrow y(t)$
        统一成一阶常系数线性微分方程的形式:
        \[
        \frac{d y(t)}{d t} + a y(t) = b x(t)
        \]

        \item 离散时间系统:系统的输入和输出信号皆为离散时间信号,$x[n] \rightarrow y[n]$
        统一成一阶常系数线性差分方程的形式:
        \[
        y[n] - a y[n - 1] = b x[n]
        \]

        \item 有记忆系统与无记忆系统
        一个系统的输出如果只取决于系统的当前输入,称为无记忆系统(又称即时系统)
        例如:
        \[
        y(t) = R x(t) \quad \text{(电阻电路)}
        \]
        \[
        y(t) = x(t)
        \]
        \[
        y[n] = x[n] \quad \text{(恒等系统)}
        \]
        否则称为有记忆系统(动态系统)
        例如:
        \[
        y(t) = \frac{1}{C} \int_{-\infty}^{t} x(\tau) d\tau \quad \text{(电容)}
        \]
        \[
        y[n] = \sum_{k=-\infty}^{n} x[k] = \sum_{k=-\infty}^{n-1} x[k] + x[n] \Rightarrow y[n] - y[n-1] = x[n] \quad \text{(累加器)}
        \]
        \[
        y[n] = x[n-1] \quad \text{(延时器)}
        \]
        \[
        y[n] = x[n] - x[n-1] \quad \text{(差分器)}
        \]

        \item 可逆性与可逆系统\par
        一个系统如果输入不同则输出不同,那么系统就是可逆的;
        一个系统如果是可逆的,那么存在一个逆系统,级联原系统后的输出就等于原系统的输入
        \begin{itemize}
            \item 乘法器 $y(t) = 2x(t)$ 的逆系统为 $w(t) = 0.5y(t)$
            \item 累加器 $y[n] = \sum_{k=-\infty}^{n} x[k]$ 逆系统为差分器 $w[n] = y[n] - y[n-1]$
            \item $y(t) = x^2(t)$ 不是可逆的;通信中的编码器是一个典型的可逆系统,解码器是其逆系统
        \end{itemize}

        \item 因果性\par
        如果一个系统在任何时刻的输出只取决于现在及过去的输入,就称为因果系统
        \begin{itemize}
            \item 因果系统例子:累加器、(后向)差分器($y[n] - y[n-1]$)、$y(t) = x(t) - \cos(t + 1)$
            \item 非因果系统例子:$y(t) = x(t) - x(t + 1)$、$y[n] = \frac{1}{2M+1} \sum_{k=-M}^{M} x[n - k]$
            \item 一切以时间为自变量的可物理实现的系统都是因果的
            \item 所有的无记忆系统都是因果的
        \end{itemize}

        \item 稳定性\par
        一个系统如果输入信号是有界的,输出信号也是有界的,则称系统是稳定的。
        \begin{itemize}
            \item 稳定系统例子:RC电路、汽车系统(输入为力$f(t)$,输出为速度$v(t)$,平衡点为$V = \frac{F}{\rho}$)、$y(t) = e^{x(t)}$
            \item 不稳定系统例子:$y[n] = \sum_{k=-\infty}^{n} u[k]=(n+1)u[n]$、$y(t) = t x(t)$
            \item 实际系统一般存在能量消耗,如电阻的耗能、摩擦的耗能;想象一下如果汽车的摩擦系数$\rho = 0$?
        \end{itemize}

        \item 时不变性\par
        直观来讲,系统的特性和行为不随时间而改变的系统为时不变系统。
        \begin{itemize}
            \item RC电路中的$R$和$C$不随时间而变化
            \item 汽车系统中的摩擦系数$\rho$和质量$m$不随时间而变化
            \item 数学上:当输入信号有一个时移,在输出信号产生同样的时移,则为时不变系统
            \begin{itemize}
                \item 如果$x(t) \rightarrow y(t)$,则有$x(t - t_0) \rightarrow y(t - t_0)$
                \item 如果$x[n] \rightarrow y[n]$,则有$x[n - n_0] \rightarrow y[n - n_0]$
            \end{itemize}
        \end{itemize}

        判断一个系统是否时不变方法:
        \begin{itemize}
            \item 直观判断方法: 若$y(\cdot)$前出现变系数、或有反转、展缩变换,则系统为时变系统
            \item 一般性步骤:
            \begin{enumerate}
                \item 令输入为$x_1(t) = x(t - t_0)$,得到输出为$y_1(t)$
                \item 如果$y(t - t_0)$与$y_1(t)$相同,则为时不变系统
            \end{enumerate}
        \end{itemize}

        \item 例1:$y(t) = \sin[x(t)]$
    \begin{itemize}
        \item 直观判断为时不变系统
        \item 证明:
        \begin{itemize}
            \item 步骤1:$x_1(t) = x(t - t_0) \Rightarrow y_1(t) = \sin[x(t - t_0)]$
            \item 步骤2:$y(t - t_0) = \sin[x(t - t_0)] = y_1(t)$
        \end{itemize}
        因此为时不变系统
    \end{itemize}

    \item 例2:$y[n] = nx[n]$
        \begin{itemize}
            \item 直观判断不是时不变系统
            \item 证明:
            \begin{itemize}
                \item 步骤1:$x_1[n] = x[n - n_0] \Rightarrow y_1[n] = nx[n - n_0]$
                \item 步骤2:$y[n - n_0] = (n - n_0)x[n - n_0] \neq y_1[n]$
            \end{itemize}
            因此不是时不变系统
        \end{itemize}

        \item 线性\par
        满足可加性和齐次性的系统是线性系统
        \begin{itemize}
            \item 可加性:$x_1 \rightarrow y_1, x_2 \rightarrow y_2$,则有$x_1 + x_2 \rightarrow y_1 + y_2$
            \item 齐次性(比例性):$x \rightarrow y$,则有$ax \rightarrow ay$,$a$为任意复常数
            \item 零输入产生零输出(齐次性)
        \end{itemize}
        两者组合得到线性系统的叠加性

        叠加性:如果系统对信号$x_k(t)$的响应为$y_k(t)$,$k=1,2,3,\ldots$,那么系统对信号$x(t) = \sum_{k} x_k(t)$的响应$y(t)$为
        \[
        y(t) = \sum_{k} y_k(t)
        \]
        离散时间线性系统同上

        判断一个系统“$\rightarrow$”是否线性系统的一般性步骤($a, b, x(t), y(t)$可为复数):
        \begin{enumerate}
            \item $x_1(t) \rightarrow y_1(t)$,$x_2(t) \rightarrow y_2(t)$
            \item 设$x_3(t) = a x_1(t) + b x_2(t) \rightarrow y_3(t)$
            \item 如果$y_3(t) = a y_1(t) + b y_2(t)$,则系统为线性的
        \end{enumerate}
        本质利用“线性”的定义

        \item 例1:证明系统$y(t) = t x(t)$是线性的
        \begin{itemize}
            \item 步骤1:$y_1(t) = t x_1(t)$,$y_2(t) = t x_2(t)$
            \item 步骤2:$x_3(t) = a x_1(t) + b x_2(t) \Rightarrow y_3(t) = t [a x_1(t) + b x_2(t)]$
            \item 步骤3:$a y_1(t) + b y_2(t) = a t x_1(t) + b t x_2(t) = y_3(t)$
            \item 因此系统是线性的
        \end{itemize}

        \item 例2:证明$y(t) = x^2(t)$是非线性的
        \begin{itemize}
            \item 步骤1:$y_1(t) = x_1^2(t)$,$y_2(t) = x_2^2(t)$
            \item 步骤2:$x_3(t) = a x_1(t) + b x_2(t) \Rightarrow y_3(t) = [a x_1(t) + b x_2(t)]^2$
            \item 步骤3:$a y_1(t) + b y_2(t) = a x_1^2(t) + b x_2^2(t) \neq y_3(t)$
            \item 因此系统不是线性的
        \end{itemize}

        \item 例3:判断$y(t) = \Re\{x(t)\}$是否为线性
        \begin{itemize}
            \item 步骤1:$y_1(t) = \Re\{x_1(t)\}$,$y_2(t) = \Re\{x_2(t)\}$
            \item 步骤2:$x_3(t) = j x_1(t) + j x_2(t) \Rightarrow y_3(t) = -\Im\{x_1(t) + x_2(t)\}$
            \item 步骤3:$j \Re\{x_1(t)\} + j \Re\{x_2(t)\} \neq y_3(t)$
            \item 因此系统不是线性的
        \end{itemize}

        \item 例4
        \begin{itemize}
            \item 例4:$y[n] = 2x[n] + 3$
            \item 容易验证系统不是线性的。但此系统输出信号的增量与输入信号的增量满足线性关系,此类系统称为增量线性系统
            \item 当 $y_0 = 0$,系统输出 $y$ 只取决于输入 $x$,称为系统的零状态响应(线性系统)
            \item 当 $x = 0$,系统输出 $y$ 只取决于 $y_0$,称为系统的零输入响应
            \item 完全响应 = 零状态响应 + 零输入响应
        \end{itemize}
    \end{itemize}
\end{definition}

\begin{figure}[ht]
    \centering
    \includegraphics[width=0.8\textwidth]{xiti21.png}
\end{figure}

\begin{figure}[ht]
    \centering
    \includegraphics[width=0.8\textwidth]{xiti2.png}
\end{figure}




\chapter{2024.9.10}\thispagestyle{fancy}

\begin{figure}[ht]
    \centering
    \includegraphics[width=0.8\textwidth]{xiti3.png}
\end{figure}

计算下列各对信号的卷积 $y[n] = x[n] \ast h[n]$。

\section*{(a) $x[n] = \alpha^n u[n], h[n] = \beta^n u[n], \alpha \neq \beta$}

卷积的定义为:
\[
y[n] = \sum_{k=-\infty}^{\infty} x[k] h[n-k]
\]
代入 $x[n] = \alpha^n u[n]$ 和 $h[n] = \beta^n u[n]$,得到:
\[
y[n] = \sum_{k=0}^{n} \alpha^k \beta^{n-k}
\]
因为 $\alpha \neq \beta$,使用几何级数的求和公式:
\[
y[n] = \frac{\alpha^{n+1} - \beta^{n+1}}{\alpha - \beta} u[n]
\]

\section*{(b) $x[n] = h[n] = \alpha^n u[n]$}

这时,两个信号相同,代入卷积公式:
\[
y[n] = \sum_{k=0}^{n} \alpha^k \alpha^{n-k}
\]
\[
y[n] = \alpha^n \sum_{k=0}^{n} 1 = (n+1)\alpha^n u[n]
\]

\section*{(c) $x[n] = \left( -\frac{1}{2} \right)^n u[n-4], h[n] = 4^n u[2-n]$}

首先,将卷积定义代入:
\[
y[n] = \sum_{k=-\infty}^{\infty} x[k] h[n-k]
\]
对于 $x[n] = \left( -\frac{1}{2} \right)^n u[n-4]$ 和 $h[n] = 4^n u[2-n]$,我们可以观察到它们的支持范围并计算卷积。由于 $x[n]$ 和 $h[n]$ 都是有限支持的,非零的卷积求和范围可以限制在有效值上。

\section*{(d) 如图 P2.21 所示的卷积}

给定图 $x[n]$ 和 $h[n]$ 的信号,我们通过使用卷积的定义:
\[
y[n] = \sum_{k=-\infty}^{\infty} x[k] h[n-k]
\]
可以看到,$x[n]$ 从 $n = 0$ 到 $n = 5$ 有值,而 $h[n]$ 从 $n = 9$ 到 $n = 16$ 有值,因此可以手动计算每一个 $n$ 值下的卷积结果。注意,$y[n]$ 只在 $n = 9$ 到 $n = 21$ 之间有非零值。

通过图像上的卷积积分计算,我们得到 $y[n]$ 在每个 $n$ 值下的精确值。



\chapter{2024.9.19}\thispagestyle{fancy}

\section{习题2.22}
\begin{figure}[ht]
    \centering
    \includegraphics[width=0.8\textwidth]{xiti4.png}
\end{figure}

\section*{题目: 对以下各对波形求单位冲激响应为 $h(t)$ 的线性时不变系统对输入 $x(t)$ 的响应 $y(t)$}

\subsection*{(a) $x(t) = e^{-\alpha t} u(t)$ 和 $h(t) = e^{-\beta t} u(t)$ 的卷积}

卷积公式为:
\[
y(t) = \int_{-\infty}^{\infty} x(\tau) h(t - \tau) d\tau
\]

带入 $x(t) = e^{-\alpha t} u(t)$ 和 $h(t) = e^{-\beta t} u(t)$,卷积式变为:
\[
y(t) = \int_0^t e^{-\alpha \tau} e^{-\beta (t - \tau)} d\tau
\]

化简为:
\[
y(t) = e^{-\beta t} \int_0^t e^{(\beta - \alpha) \tau} d\tau
\]

分为两种情况:

- 当 $\alpha \neq \beta$ 时:
  \[
  y(t) = e^{-\beta t} \left[ \frac{e^{(\beta - \alpha) \tau}}{\beta - \alpha} \right]_0^t = \frac{e^{-\alpha t} - e^{-\beta t}}{\beta - \alpha} u(t)
  \]

- 当 $\alpha = \beta$ 时:
  \[
  y(t) = e^{-\alpha t} \int_0^t 1 d\tau = t e^{-\alpha t} u(t)
  \]

因此卷积的结果为:
\[
y(t) =
\begin{cases}
\frac{e^{-\alpha t} - e^{-\beta t}}{\beta - \alpha} u(t), & \alpha \neq \beta \\
t e^{-\alpha t} u(t), & \alpha = \beta
\end{cases}
\]

\subsection*{(b) $x(t) = u(t) - 2u(t-2) + u(t-5)$ 和 $h(t) = e^{2t} u(1-t)$ 的卷积}

首先,分段定义 $x(t)$:
\[
x(t) =
\begin{cases}
1, & 0 \leq t < 2 \\
-1, & 2 \leq t < 5 \\
0, & \text{其他}
\end{cases}
\]
而 $h(t) = e^{2t} u(1 - t)$ 仅在 $t \leq 1$ 时非零。

卷积公式为:
\[
y(t) = \int_{-\infty}^{\infty} x(\tau) h(t - \tau) d\tau
\]

在 $t \leq 1$ 的区间内,卷积积分为:
\[
y(t) = \int_0^t e^{2(t - \tau)} x(\tau) d\tau
\]

根据 $x(t)$ 的分段定义,可以分别计算不同区间内的积分,结合 $h(t)$ 的支撑区域,得出分段结果。

\subsection*{(c) 如图 P2.22(a) 所示的卷积}

$x(t)$ 是一个周期为 $2$ 的正弦波,$h(t)$ 是一个方波,支持范围在 $1 \leq t \leq 3$。

卷积公式为:
\[
y(t) = \int_{-\infty}^{\infty} x(\tau) h(t - \tau) d\tau
\]

由于 $h(t)$ 的支持范围为 $1 \leq t \leq 3$,在此区间内对正弦波进行卷积,计算重叠区域内的积分,得到结果。

\subsection*{(d) 如图 P2.22(b) 所示的卷积}

$x(t)$ 是一个斜坡函数:
\[
x(t) = at + b, \quad t \geq 0
\]
而 $h(t)$ 是一个宽度为 $1$,高度为 $\frac{1}{3}$ 的矩形脉冲,定义为:
\[
h(t) = \frac{1}{3} u(t) - \frac{1}{3} u(t-1)
\]

卷积公式为:
\[
y(t) = \int_{-\infty}^{\infty} x(\tau) h(t - \tau) d\tau
\]

将 $h(t)$ 的定义带入后,可以分区间对斜坡函数和矩形脉冲进行卷积计算,最终得出卷积结果。

\subsection*{(e) 如图 P2.22(c) 所示的卷积}

$x(t)$ 是一个阶跃信号,而 $h(t)$ 是一个从 1 线性下降到 0 的三角脉冲,定义为:
\[
h(t) =
\begin{cases}
1 - t, & 0 \leq t \leq 1 \\
0, & \text{其他}
\end{cases}
\]

卷积公式为:
\[
y(t) = \int_{-\infty}^{\infty} x(\tau) h(t - \tau) d\tau
\]

由于 $h(t)$ 是分段函数,卷积将分段进行计算。结合 $x(t)$ 的分段定义,计算每个区间的重叠区域,得出卷积结果。

\begin{figure}[ht]
    \centering
    \includegraphics[width=0.8\textwidth]{xiti41.png}
\end{figure}

\begin{figure}[ht]
    \centering
    \includegraphics[width=0.8\textwidth]{xiti42.png}
\end{figure}

\begin{figure}[ht]
    \centering
    \includegraphics[width=0.8\textwidth]{xiti43.png}
\end{figure}

\textbf{\textcolor{red}{有记忆和无记忆系统}}
    \begin{itemize}
        \item 一个系统的输出如果只取决于系统的当前输入,称为无记忆系统(又称即时系统),否则称为有记忆系统
        \item 对于离散LTI系统,$y[n] = \sum_{k=-\infty}^{\infty} h[k] x[n - k]$,无记忆意味着对于 $k \neq 0$ 有 $h[k] = 0$,即 $y[n] = h[0] x[n] = K x[n]$,$K = h[0]$ 为常数;令 $x[n] = \delta[n]$,得到 $h[n] = K \delta[n]$
        \item 无记忆LTI系统的单位脉冲响应为脉冲函数
        \item 对于连续LTI系统,$y(t) = \int_{-\infty}^{\infty} x(\tau) h(t - \tau) d\tau$,无记忆意味着对于 $\tau \neq t$ 有 $h(t - \tau) = 0$,也即 $y(t) = \int_{-\infty}^{\infty} x(t) h(0) d\tau = K x(t)$,$K = \int_{-\infty}^{\infty} h(0) d\tau$ 为常数;令输入为 $\delta(t)$,得到 $h(t) = K \delta(t)$
        \item 系统的单位冲激响应为冲激函数
        \item 令 $K = 1$,得到 $y[n] = x[n] = x[n] * \delta[n]$,$y(t) = x(t) = x(t) * \delta(t)$
        \item 信号与单位冲激(脉冲)函数卷积得到自身,也即单位冲激(脉冲)函数的筛选性质
    \end{itemize}
\vspace{1em}
    \textbf{\textcolor{red}{可逆性}}
    \begin{itemize}
        \item 一个系统如果是可逆的,那么存在一个逆系统,级联原系统后的输出就等于原系统的输入,也即级联后的系统冲激响应为 $\delta(t)$,也即
    \[
    h(t) * h_1(t) = \delta(t) \quad \text{或} \quad h[n] * h_1[n] = \delta[n]
    \]
    
        \item 例1
    $y(t) = x(t - t_0)$,求其逆系统。
    
    令 $x(t) = \delta(t)$,得到 $h(t) = \delta(t - t_0)$,
    有 $y(t) = x(t - t_0) = x(t) * h(t) = x(t) * \delta(t - t_0)$。
    
    一个信号与时移冲激(脉冲)信号的卷积就是信号的时移:
    \[
    \delta(t - t_0) * \delta(t + t_0) = \delta(t) \Rightarrow h_1(t) = \delta(t + t_0)
    \]
    
        \item 例2
    $h[n] = u[n]$,求其逆系统。
    
    有 $y[n] = \sum_{k=-\infty}^{\infty} x[k] h[n - k] = \sum_{k=-\infty}^{n} x[k]$。
    
    一个信号与阶跃脉冲(冲激)函数的卷积是信号的累加(积分)。
    
    已知:$u[n] - u[n - 1] = \delta[n]$,
    \[
    h[n] * h_1[n] = u[n] - u[n - 1] = u[n] * \delta[n] - u[n] * \delta[n - 1] = u[n] * (\delta[n] - \delta[n - 1])
    \]
    因此,$h_1[n] = \delta[n] - \delta[n - 1]$。
    \end{itemize}
\vspace{1em}
    \textbf{\textcolor{red}{因果性}}\par
    一个因果系统的输出只取决于系统的过去和现在的输入。
    
    对于离散时间因果LTI系统,
    \[
    y[n] = \sum_{k=-\infty}^{\infty} x[k] h[n - k]
    \]
    意味着当 $k > n$,必有 $h[n - k] = 0$,因此当 $n < 0$,有 $h[n] = 0$(激励出现之前,响应为零),也即
    \[
    y[n] = \sum_{k=-\infty}^{n} x[k] h[n - k] \quad \text{或} \quad y[n] = \sum_{k=0}^{\infty} h[k] x[n - k]
    \]
    
    一个因果系统的输出只取决于系统的过去和现在的输入。
    
    对于一个连续时间因果LTI系统,类似也有当 $t < 0$,有 $h(t) = 0$,
    \[
    y(t) = \int_{\tau = -\infty}^{t} x(\tau) h(t - \tau) d\tau \quad \text{或} \quad y(t) = \int_{\tau = 0}^{\infty} h(\tau) x(t - \tau) d\tau
    \]
    
    一个因果系统的输入在某个时刻前为0,则系统的响应在那个时刻之前也必为0,这就是系统的初始松弛条件;对于线性系统,因果性等价初始松弛条件;也就是系统只有零状态响应。
    
    因果性是系统的性质,但一般也将 $n < 0$ 或 $t < 0$ 时为零的信号也称为因果信号,经常表示为 $x(t)u(t)$ 或者 $x[n]u[n]$:一个LTI系统的因果性等价于系统的冲激(脉冲)响应是一个因果信号。\par
\vspace{1em}
\raggedright
    \textbf{\textcolor{red}{稳定性}}
    \begin{itemize}

    \item 如果一个系统的输入是有界的,那么输出也是有界的,该系统就是稳定的。
    
    对于离散时间LTI系统,
    \[
    |y[n]| = |\sum_{k=-\infty}^{\infty} h[k] x[n - k]| \leq \sum_{k=-\infty}^{\infty} |h[k]| |x[n - k]| \leq B \sum_{k=-\infty}^{\infty} |h[k]|
    \]
    因此,$|y[n]| < \infty$ 意味着 $\sum_{k=-\infty}^{\infty} |h[k]| < \infty$,称系统的单位脉冲响应绝对可和,这个条件是离散时间LTI系统稳定的充要条件。
    
    对于连续时间LTI系统,类似可以得到
    \[
    \int_{t = -\infty}^{\infty} |h(t)| dt < \infty
    \]
    称系统的单位冲激响应绝对可积,这个条件是连续时间LTI系统稳定的充要条件。
 
        \item 例1\\
    纯时移系统,$h[n] = \delta[n - n_0]$ 或 $h(t) = \delta(t - t_0)$
    \[
    \sum_{n=-\infty}^{\infty} |h[n]| = \sum_{n=-\infty}^{\infty} |\delta[n - n_0]| = 1
    \]
    \[
    \int_{t = -\infty}^{\infty} |h(t)| dt = \int_{t = -\infty}^{\infty} |\delta(t - t_0)| dt = 1
    \]
    因此,纯时移系统是稳定系统。
    
    \item 例2\\
    累加器或积分器,$h[n] = u[n]$ 或 $h(t) = u(t)$
    \[
    \sum_{n=-\infty}^{\infty} |h[n]| = \sum_{n=-\infty}^{\infty} |u[n]| = \sum_{n=0}^{\infty} 1 = \infty
    \]
    \[
    \int_{t = -\infty}^{\infty} |h(t)| dt = \int_{t = -\infty}^{\infty} |u(t)| dt = \int_{t = 0}^{\infty} u(t) dt = \infty
    \]
    因此,累加器或积分器不是稳定系统。
    \end{itemize}



\chapter{2024.9.23}\thispagestyle{fancy}

\section{习题 2.18}
\begin{itemize}
        \item 考虑一个因果线性时不变系统,其输入$x[n]$ 和输出 $y[n]$ 之间的关系由下面的差分方程给出:
        \[
            y[n] = \frac{1}{4}y[n-1] + x[n]
        \]
        若 $x[n] = \delta[n-1]$,求 $y[n]$ 的表达式。
        
            \item 要解决这个问题,我们可以利用差分方程的递推关系。我们从给定的输入 $x[n] = \delta[n-1]$ 开始,逐步计算输出 $y[n]$。
\end{itemize}


\section*{1. 确定初始条件}

由于系统是因果的,我们通常假设 $y[n] = 0$ 对于 $n < 0$。

\section*{2. 计算输出}

使用给定的差分方程:

\begin{equation}
    y[n] = \frac{1}{4} y[n-1] + x[n]
\end{equation}

\subsection*{当 $n = 0$}

\begin{align*}
    y[0] &= \frac{1}{4} y[-1] + x[0] \\
          &= \frac{1}{4} \cdot 0 + 0 \\
          &= 0
\end{align*}

\subsection*{当 $n = 1$}

\begin{align*}
    y[1] &= \frac{1}{4} y[0] + x[1] \\
          &= \frac{1}{4} \cdot 0 + 1 \\
          &= 1
\end{align*}

\subsection*{当 $n = 2$}

\begin{align*}
    y[2] &= \frac{1}{4} y[1] + x[2] \\
          &= \frac{1}{4} \cdot 1 + 0 \\
          &= \frac{1}{4}
\end{align*}

\subsection*{当 $n = 3$}

\begin{align*}
    y[3] &= \frac{1}{4} y[2] + x[3] \\
          &= \frac{1}{4} \cdot \frac{1}{4} + 0 \\
          &= \frac{1}{16}
\end{align*}

\subsection*{当 $n = 4$}

\begin{align*}
    y[4] &= \frac{1}{4} y[3] + x[4] \\
          &= \frac{1}{4} \cdot \frac{1}{16} + 0 \\
          &= \frac{1}{64}
\end{align*}

\section*{3. 归纳出一般形式}

可以观察到,输出 $y[n]$ 在每一步中都在乘以 $\frac{1}{4}$。所以我们可以归纳出:

\begin{equation}
    y[n] = \left(\frac{1}{4}\right)^{n-1} \quad \text{对于} \quad n \geq 1
\end{equation}

而对于 $n < 1$,输出仍然为 0。因此可以总结为:

\begin{equation}
    y[n] = 
    \begin{cases} 
    0 & n < 1 \\ 
    \left(\frac{1}{4}\right)^{n-1} & n \geq 1 
    \end{cases}
\end{equation}

\section*{4. 最终结果}

输出 $y[n]$ 的表达式为:

\begin{equation}
    y[n] = \left(\frac{1}{4}\right)^{n-1} u[n-1]
\end{equation}

其中 $u[n-1]$ 是单位阶跃函数。




\section{习题 2.24}
\begin{itemize}
    \item 考虑图中的三个因果线性时不变系统的级联,单位脉冲响应 $h_2[n]$ 为
    \begin{equation}
        h_2[n] = u[n] - u[n-2]
    \end{equation}
    \item 整个系统的单位脉冲响应如图所示。
\end{itemize}

\begin{itemize}
    \item[(a)] 求 $h_1[n]$.
    \item[(b)] 求整个系统对输入 $x[n] = \delta[n] - \delta[n-1]$ 的响应.
\end{itemize}


\section*{解答}

\subsection*{(a) 求 $h_1[n]$}

已知整个系统是由三级子系统串联组成的,系统的单位脉冲响应为:

\[
h[n] = h_1[n] * h_2[n] * h_3[n]
\]

其中,$h_2[n] = h_3[n] = u[n] - u[n-2]$,因此可以写作:

\[
h_2[n] = h_3[n] = 
\begin{cases}
1, & n = 0 \\
1, & n = 1 \\
0, & \text{其他}
\end{cases}
\]

这表示 $h_2[n]$ 和 $h_3[n]$ 都是一个有限长为 2 的 FIR 滤波器。

整个系统的单位脉冲响应 $h[n]$ 已知为:

\[
h[n] = 
\begin{cases}
1, & n = 0 \\
5, & n = 1 \\
10, & n = 2 \\
11, & n = 3 \\
8, & n = 4 \\
4, & n = 5 \\
1, & n = 6 \\
0, & \text{其他} \\
\end{cases}
\]

现在,我们首先计算 $h_2[n] * h_3[n]$。

\subsection*{1. 计算 $h_2[n] * h_3[n]$}

由于 $h_2[n]$ 和 $h_3[n]$ 都是有限长的脉冲响应,我们可以直接进行卷积:

\[
h_{23}[n] = h_2[n] * h_3[n]
\]

计算卷积:

\[
h_{23}[n] = 
\begin{cases}
1, & n = 0 \\
2, & n = 1 \\
1, & n = 2 \\
0, & \text{其他}
\end{cases}
\]

因此,$h_{23}[n] = [1, 2, 1]$。

\subsection*{2. 计算 $h_1[n]$}

现在,我们有:

\[
h[n] = h_1[n] * h_{23}[n]
\]

设 $h_1[n] = [h_1[0], h_1[1], h_1[2], \dots]$,根据卷积定义,可以得到以下方程:

\[
h[0] = h_1[0] \cdot h_{23}[0]
\]
\[
h[1] = h_1[1] \cdot h_{23}[0] + h_1[0] \cdot h_{23}[1]
\]
\[
h[2] = h_1[2] \cdot h_{23}[0] + h_1[1] \cdot h_{23}[1] + h_1[0] \cdot h_{23}[2]
\]
\[
h[3] = h_1[3] \cdot h_{23}[0] + h_1[2] \cdot h_{23}[1] + h_1[1] \cdot h_{23}[2]
\]
依此类推。

将 $h_{23}[n] = [1, 2, 1]$ 和 $h[n] = [1, 5, 10, 11, 8, 4, 1]$ 代入,逐项解出 $h_1[n]$:

\[
h[0] = 1 = h_1[0] \cdot 1 \quad \Rightarrow \quad h_1[0] = 1
\]
\[
h[1] = 5 = h_1[1] \cdot 1 + h_1[0] \cdot 2 \quad \Rightarrow \quad 5 = h_1[1] + 2 \times 1 \quad \Rightarrow \quad h_1[1] = 3
\]
\[
h[2] = 10 = h_1[2] \cdot 1 + h_1[1] \cdot 2 + h_1[0] \cdot 1 \quad \Rightarrow \quad 10 = h_1[2] + 2 \times 3 + 1 \times 1 \quad \Rightarrow \quad h_1[2] = 3
\]
\[
h[3] = 11 = h_1[3] \cdot 1 + h_1[2] \cdot 2 + h_1[1] \cdot 1 \quad \Rightarrow \quad 11 = h_1[3] + 2 \times 3 + 3 \quad \Rightarrow \quad h_1[3] = 2
\]

\subsection*{结论}

因此,$h_1[n]$ 的解为:

\[
h_1[n] = [1, 3, 3, 2]
\]


\subsection*{(b) 求整个系统对输入 $x[n] = \delta[n] - \delta[n-1]$ 的响应}

输入信号为:

\[
x[n] = \delta[n] - \delta[n-1]
\]

整个系统的输出为:

\[
y[n] = x[n] * h[n] = h[n] - h[n-1]
\]

将 $h[n]$ 代入:

\[
y[0] = h[0] = 1
\]
\[
y[1] = h[1] - h[0] = 5 - 1 = 4
\]
\[
y[2] = h[2] - h[1] = 10 - 5 = 5
\]
\[
y[3] = h[3] - h[2] = 11 - 10 = 1
\]
\[
y[4] = h[4] - h[3] = 8 - 11 = -3
\]
\[
y[5] = h[5] - h[4] = 4 - 8 = -4
\]
\[
y[6] = h[6] - h[5] = 1 - 4 = -3
\]
\[
y[7] = h[7] - h[6] = 0 - 1 = -1
\]

因此,系统的输出响应为:

\[
y[n] = [1, 4, 5, 1, -3, -4, -3, -1]
\]


\chapter{2024.10.8}\thispagestyle{fancy}

\section{习题 3.23}

给出下面周期为 4 的各连续时间信号的傅里叶级数系数,求每一个 \( x(t) \) 信号:

对于周期为 \( T \) 的周期信号 \( x(t) \),傅里叶级数逆变换为:
\[
x(t) = \sum_{k=-\infty}^{\infty} a_k e^{j \frac{2 \pi k t}{T}}
\]

\textbf{(a) 系数}:
\[
a_k =
\begin{cases}
0, & k = 0 \\
(j)^k \frac{\sin(k \pi / 4)}{k \pi}, & \text{其他}
\end{cases}
\]

信号 \( x(t) \) 的重建公式为:
\[
x(t) = \sum_{k=-\infty}^{\infty} a_k e^{j \frac{\pi k t}{2}}
\]

\textbf{(b) 系数}:
\[
a_k = 
\begin{cases}
\frac{1}{16}, & k = 0 \\
(-1)^k \frac{\sin(k \pi / 8)}{2k \pi}, & \text{其他}
\end{cases}
\]

信号 \( x(t) \) 的重建公式为:
\[
x(t) = \sum_{k=-\infty}^{\infty} a_k e^{j \frac{\pi k t}{2}}
\]

\textbf{(c) 系数(第一种)}:
\[
a_k =
\begin{cases}
jk, & |k| < 3 \\
0, & \text{其他}
\end{cases}
\]

信号 \( x(t) \) 的重建公式为:
\[
x(t) = \sum_{k=-2}^{2} a_k e^{j \frac{\pi k t}{2}}
\]

\textbf{(d) 系数(第二种)}:
\[
a_k =
\begin{cases}
1, & k \text{为偶数} \\
2, & k \text{为奇数}
\end{cases}
\]

该信号的傅里叶级数重建公式为:
\[
x(t) = \sum_{k=-\infty}^{\infty} a_k e^{j \frac{\pi k t}{2}}
\]

\section{习题 3.25}

下面三个连续时间周期信号的基波周期 \( T = 1/2 \):
\[ x(t) = \cos(4\pi t), \quad y(t) = \sin(4\pi t), \quad z(t) = x(t)y(t) \]

(a) 求 \( x(t) \) 的傅里叶级数系数。\par
(b) 求 \( y(t) \) 的傅里叶级数系数。\par
(c) 利用 (a) 和 (b) 的结果,按照连续时间傅里叶级数的相乘性质,求 \( z(t) = x(t)y(t) \) 的傅里叶级数系数。\par

\textbf{(a) 求 \( x(t) = \cos(4\pi t) \) 的傅里叶级数系数}

由于 \( \cos(4\pi t) \) 是一个正弦信号,频率为 4(基波频率的 2 倍)。其傅里叶级数表达式为:
\[
x(t) = \cos(4\pi t) = \frac{1}{2}\left(e^{j4\pi t} + e^{-j4\pi t}\right)
\]

因此,傅里叶级数系数 \( a_k \) 为:
\[
a_k =
\begin{cases}
\frac{1}{2}, & k = 2 \\
\frac{1}{2}, & k = -2 \\
0, & \text{其他}
\end{cases}
\]

\textbf{(b) 求 \( y(t) = \sin(4\pi t) \) 的傅里叶级数系数}

同样,\( \sin(4\pi t) \) 是一个正弦信号,频率也是 4。其傅里叶级数表达式为:
\[
y(t) = \sin(4\pi t) = \frac{1}{2j}\left(e^{j4\pi t} - e^{-j4\pi t}\right)
\]

因此,傅里叶级数系数 \( b_k \) 为:
\[
b_k =
\begin{cases}
-\frac{j}{2}, & k = 2 \\
\frac{j}{2}, & k = -2 \\
0, & \text{其他}
\end{cases}
\]

\textbf{(c) 利用 (a) 和 (b) 的结果,按照傅里叶级数的乘积性质,求 \( z(t) = x(t)y(t) \) 的傅里叶级数系数}

根据傅里叶级数的卷积性质,两个信号 \( x(t) \) 和 \( y(t) \) 的乘积 \( z(t) = x(t)y(t) \) 的傅里叶级数系数是 \( x(t) \) 和 \( y(t) \) 的傅里叶系数的卷积。

设 \( z(t) \) 的傅里叶系数为 \( c_k \),则有:
\[
c_k = \sum_{m=-\infty}^{\infty} a_m b_{k-m}
\]

其中 \( a_m \) 是 \( x(t) \) 的傅里叶系数,\( b_m \) 是 \( y(t) \) 的傅里叶系数。

由于 \( a_k \) 和 \( b_k \) 的非零项只在 \( k = \pm 2 \),我们只需要考虑这些项的卷积:
\[
c_4 = a_2 b_2 = \frac{1}{2} \times -\frac{j}{2} = -\frac{j}{4}
\]

\[
c_0 = a_2 b_{-2} + a_{-2} b_2 = \frac{1}{2} \times \frac{j}{2} + \frac{1}{2} \times -\frac{j}{2} = 0
\]

\[
c_{-4} = a_{-2} b_{-2} = \frac{1}{2} \times \frac{j}{2} = \frac{j}{4}
\]

因此,信号 \( z(t) = x(t) y(t) \) 的傅里叶级数系数为:
\[
c_k =
\begin{cases}
-\frac{j}{4}, & k = 4 \\
0, & k = 0 \\
\frac{j}{4}, & k = -4 \\
0, & \text{其他}
\end{cases}
\]


\chapter{2024.10.15}\thispagestyle{fancy}


\section{习题 3.28}

对下面每一个离散时间周期信号,求其傅里叶级数系数,并画出每一组系数 $a_k$ 的模和相位。

\begin{itemize}
    \item[(a)] 图 P3.28(a) 至图 P3.28(c) 中的每一个 $x[n]$。

对图 (a),周期 $N = 7$,
\[ 
x[n] = \begin{cases}
    1, & 0 \leq n \leq 4 \\
    0, & 5 \leq n \leq 6
\end{cases}
\]

傅里叶级数系数 $a_k$ 的计算公式为:
\[
a_k = \frac{1}{N} \sum_{n=0}^{N-1} x[n] e^{-j \frac{2\pi k n}{N}} = \frac{1}{7} \sum_{n=0}^{4} x[n] e^{-j \frac{2\pi k n}{7}}
\]
代入 $x[n]$ 的值:
\[
a_k = \frac{1}{7} \frac{e^{-jk\frac{4\pi}{7}} \sin\left(\frac{5\pi k}{7}\right)}{\sin\left(\frac{\pi k}{7}\right)}
\]

对图 (b),周期 $N = 6$,
\[ 
x[n] = \begin{cases}
    1, & 0 \leq n \leq 3 \\
    0, & 4 \leq n \leq 5
\end{cases}
\]

傅里叶级数系数 $a_k$ 的计算公式为:
\[
a_k = \frac{1}{6} \sum_{n=0}^{3} x[n] e^{-j \frac{2\pi k n}{6}}
\]
代入 $x[n]$ 的值:
\[
a_k = \frac{1}{6} \frac{e^{-jk\frac{\pi}{2}} \sin\left(\frac{2\pi k}{3}\right)}{\sin\left(\frac{\pi k}{6}\right)}
\]

对图 (c),周期 $N = 6$,
\[ 
x[n] = \begin{cases}
    1, & n = 0 \\
    0, & n = 3 \\
    -1, & n = 2, 4 \\
    2, & n = -1, 1
\end{cases}
\]

傅里叶级数系数 $a_k$ 的计算公式为:
\[
a_k = \frac{1}{6} \sum_{n=-1}^{4} x[n] e^{-j \frac{2\pi k n}{6}}
\]
代入 $x[n]$ 的值:
\[
a_k = \frac{1}{6} \left[ 1 + 4\cos\left(\frac{k\pi}{3}\right) - 2\cos\left(\frac{2k\pi}{3}\right) \right]
\]









    \item[(b)] $x[n] = \sin\left( \frac{2\pi n}{3} \right)\cos\left( \frac{\pi n}{2} \right)$。
    
    \[
    x[n] = \sin\left( \frac{2\pi n}{3} \right)\cos\left( \frac{\pi n}{2} \right) = \frac{1}{2j}\left( e^{j\frac{2\pi n}{3}} - e^{-j\frac{2\pi n}{3}} \right) \times \frac{1}{2}\left( e^{j\frac{\pi n}{2}} + e^{-j\frac{\pi n}{2}} \right)
    \]
    \[ 
        = \frac{1}{4j}\left( e^{j\frac{7\pi n}{6}} + e^{j\frac{\pi n}{6}} - e^{-j\frac{7\pi n}{6}} - e^{-j\frac{\pi n}{6}} \right)
    \]
    \[
    = \frac{1}{4j}\left( e^{j7\frac{2\pi}{12}n} \right) - \frac{1}{4j}\left( e^{-j7\frac{2\pi}{12}n} \right) + \frac{1}{4j}\left( e^{j\frac{2\pi}{12}n} - e^{-j\frac{2\pi}{12}n} \right)
    \]
    即x[n]的周期为12,非零的傅里叶级数系数为:
    \[
    a_1 = a_{-1}^* = \frac{1}{4j} = -\frac{1}{4}j, \quad a_7 = a_{-7}^* = \frac{1}{4j} = -\frac{1}{4}j
    \]

    \item[(c)] $x[n]$ 的周期为 4,且有 $x[n] = 1 - \sin\left( \frac{\pi n}{4} \right)$,$0 \leq n \leq 3$。
    
    由于 $x[n]$ 的周期为 4,我们可以写出傅里叶级数:
    \[
    x[n] = \sum_{k=-\infty}^{\infty} a_k e^{j \frac{2\pi k n}{4}}
    \]
    计算傅里叶系数 $a_k$:
    \[
    a_k = \frac{1}{4} \sum_{n=0}^{3} x[n] e^{-j \frac{2\pi k n}{4}}
    \]
    代入 $x[n] = 1 - \sin\left( \frac{\pi n}{4} \right)$ 计算每个 $a_k$。

    \[
    a_k = \frac{1}{4} \left[1 + (2-\sqrt{2}\cos (\frac{\pi}{2} k) )  \right]
    \]

    \item[(d)] $x[n]$ 的周期为 12,且有 $x[n] = 1 - \sin\left( \frac{\pi n}{4} \right)$,$0 \leq n \leq 11$。
    
    由于 $x[n]$ 的周期为 12,我们可以写出傅里叶级数:
    \[
    x[n] = \sum_{k=-\infty}^{\infty} a_k e^{j \frac{2\pi k n}{12}}
    \]
    计算傅里叶系数 $a_k$:
    \[
    a_k = \frac{1}{12} \sum_{n=0}^{11} x[n] e^{-j \frac{2\pi k n}{12}}
    \]
    代入 $x[n] = 1 - \sin\left( \frac{\pi n}{4} \right)$ 计算每个 $a_k$。
    \[
    a_k = \frac{1}{12} \left[1 + (2 - \sqrt{2})\cos(\frac{k\pi}{6}) + (2 - \sqrt{2})\cos(\frac{k\pi}{2}) + (2 + \sqrt{2})\cos(\frac{5k\pi}{6})+2\cos(\frac{2k\pi}{3})+2(-1)^k\right]
    \]
\end{itemize}


\section{习题 3.36}

考虑一个因果离散时间线性时不变系统,其输入 $x[n]$ 和输出 $y[n]$ 由下列差分方程所关联:
\[
y[n] - \frac{1}{4}y[n-1] = x[n]
\]

在下面两种输入下,求输出 $y[n]$ 的傅里叶级数表示:

\begin{enumerate}
    \item $x[n] = \sin\left(\frac{3\pi n}{4}\right)$
    
    \item $x[n] = \cos\left(\frac{\pi n}{4}\right) + 2\cos\left(\frac{\pi n}{2}\right)$
\end{enumerate}

\subsection*{解答}
(a) 先求系统响应 $H(e^{j\omega})$。

对于 LTI 系统,当输入 $x[n] = e^{j\omega n}$ 时,输出为 $y[n] = H(e^{j\omega})e^{j\omega n}$,其中 $H(e^{j\omega})$ 为系统的频率响应。

将 $x[n]$ 和 $y[n]$ 代入差分方程,得到:
\[
H(e^{j\omega}) = \frac{1}{1-\frac{1}{4}e^{-j\omega}}
\]

对于输入 $x[n] = \sin\left(\frac{3\pi n}{4}\right)$,其傅里叶级数表示为:
\[
\sin\left(\frac{3\pi n}{4}\right) = \frac{1}{2j}\left(e^{j\frac{3\pi}{4}n} - e^{-j\frac{3\pi}{4}n}\right)
\]
对于 $x[n]$,其 $\omega_0 = \frac{2\pi}{8}$,非零的傅里叶级数系数为:
\[
a_3 = \frac{1}{2j}, \quad a_{-3} = -\frac{1}{2j}
\]
故 $y[n]$ 中非零的傅里叶级数系数为:
\[
b_3 = a_3 H(e^{j\frac{3\pi}{4}}) = \frac{1}{2j} \cdot \frac{1}{1-\frac{1}{4}e^{-j\frac{3\pi}{4}}}
\]
\[
b_{-3} = a_{-3} H(e^{-j\frac{3\pi}{4}}) = -\frac{1}{2j} \cdot \frac{1}{1-\frac{1}{4}e^{j\frac{3\pi}{4}}}
\]

(b) 对于输入 $x[n] = \cos\left(\frac{\pi n}{4}\right) + 2\cos\left(\frac{\pi n}{2}\right)$,其傅里叶级数表示为:
\[
x[n] = \frac{1}{2}e^{j\frac{\pi n}{4}} + e^{j\frac{\pi n}{2}} + \frac{1}{2}e^{-j\frac{\pi n}{4}} + e^{-j\frac{\pi n}{2}}
\]
对于 $x[n]$,其 $\omega_0 = \frac{2\pi}{8}$,非零的傅里叶级数系数为:
\[
a_1 = \frac{1}{2}, \quad a_{-1} = \frac{1}{2}, \quad a_2 = 1, \quad a_{-2} = 1
\]
故 $y[n]$ 中非零的傅里叶级数系数为:
\[
b_1 = a_1 H(e^{j\frac{\pi}{4}}) = \frac{1}{2} \cdot \frac{1}{1-\frac{1}{4}e^{j\frac{\pi}{4}}}
\]
\[
b_{-1} = a_{-1} H(e^{-j\frac{\pi}{4}}) = \frac{1}{2} \cdot \frac{1}{1-\frac{1}{4}e^{-j\frac{\pi}{4}}}
\]
\[
b_2 = a_2 H(e^{j\frac{\pi}{2}}) = 1 \cdot \frac{1}{1-\frac{1}{4}e^{j\frac{\pi}{2}}}
\]
\[
b_{-2} = a_{-2} H(e^{-j\frac{\pi}{2}}) = 1 \cdot \frac{1}{1-\frac{1}{4}e^{-j\frac{\pi}{2}}}
\]




\chapter{2024.10.17}\thispagestyle{fancy}

\section{习题 3.42}

令 $x(t)$ 是一个基波周期为 $T$,傅里叶级数系数为 $a_k$ 的实值信号。

(a) 证明:$a_k = a_{-k}^*$,并且 $a_0$ 一定为实数。

\begin{proof}
由于 $x(t)$ 是实值信号,其傅里叶级数表示为:
\[
x(t) = \sum_{k=-\infty}^{\infty} a_k e^{j\frac{2\pi k}{T}t}
\]
取复共轭:
\[
x^*(t) = \left( \sum_{k=-\infty}^{\infty} a_k e^{j\frac{2\pi k}{T}t} \right)^* = \sum_{k=-\infty}^{\infty} a_k^* e^{-j\frac{2\pi k}{T}t}
\]
由于 $x(t)$ 是实值信号,$x^*(t) = x(t)$,所以:
\[
\sum_{k=-\infty}^{\infty} a_k^* e^{-j\frac{2\pi k}{T}t} = \sum_{k=-\infty}^{\infty} a_k e^{j\frac{2\pi k}{T}t}
\]
令 $k \rightarrow -k$,得到:
\[
\sum_{k=-\infty}^{\infty} a_{-k}^* e^{j\frac{2\pi k}{T}t} = \sum_{k=-\infty}^{\infty} a_k e^{j\frac{2\pi k}{T}t}
\]
因此,$a_k = a_{-k}^*$。特别地,当 $k=0$ 时,$a_0 = a_0^*$,所以 $a_0$ 一定为实数。
\end{proof}

(b) 证明:若 $x(t)$ 为偶函数,则它的傅里叶级数系数一定为实偶函数。

\begin{proof}
若 $x(t)$ 为偶函数,则 $x(t) = x(-t)$。其傅里叶级数表示为:
\[
x(t) = \sum_{k=-\infty}^{\infty} a_k e^{j\frac{2\pi k}{T}t}
\]
由于 $x(t)$ 为偶函数,$x(-t) = x(t)$,所以:
\[
\sum_{k=-\infty}^{\infty} a_k e^{-j\frac{2\pi k}{T}t} = \sum_{k=-\infty}^{\infty} a_k e^{j\frac{2\pi k}{T}t}
\]
令 $k \rightarrow -k$,得到:
\[
\sum_{k=-\infty}^{\infty} a_{-k} e^{j\frac{2\pi k}{T}t} = \sum_{k=-\infty}^{\infty} a_k e^{j\frac{2\pi k}{T}t}
\]
因此,$a_k = a_{-k}$,并且 $a_k$ 为实数。
\end{proof}

(c) 证明:若 $x(t)$ 为奇函数,则它的傅里叶级数系数是虚数且为奇函数,$a_0 = 0$。

\begin{proof}
若 $x(t)$ 为奇函数,则 $x(t) = -x(-t)$。其傅里叶级数表示为:
\[
x(t) = \sum_{k=-\infty}^{\infty} a_k e^{j\frac{2\pi k}{T}t}
\]
由于 $x(t)$ 为奇函数,$x(-t) = -x(t)$,所以:
\[
\sum_{k=-\infty}^{\infty} a_k e^{-j\frac{2\pi k}{T}t} = -\sum_{k=-\infty}^{\infty} a_k e^{j\frac{2\pi k}{T}t}
\]
令 $k \rightarrow -k$,得到:
\[
\sum_{k=-\infty}^{\infty} a_{-k} e^{j\frac{2\pi k}{T}t} = -\sum_{k=-\infty}^{\infty} a_k e^{j\frac{2\pi k}{T}t}
\]
因此,$a_k = -a_{-k}$,并且 $a_k$ 为虚数。特别地,当 $k=0$ 时,$a_0 = -a_0$,所以 $a_0 = 0$。
\end{proof}

(d) 证明:$x(t)$ 偶部的傅里叶系数等于 $\Re\{a_k\}$。

\begin{proof}
设 $x(t)$ 的偶部为 $x_e(t)$,则:
\[
x_e(t) = \frac{x(t) + x(-t)}{2}
\]
其傅里叶级数系数为:
\[
a_k^e = \frac{a_k + a_{-k}}{2} = \Re\{a_k\}
\]
\end{proof}

(e) 证明:$x(t)$ 奇部的傅里叶系数等于 $j\Im\{a_k\}$。

\begin{proof}
设 $x(t)$ 的奇部为 $x_o(t)$,则:
\[
x_o(t) = \frac{x(t) - x(-t)}{2}
\]
其傅里叶级数系数为:
\[
a_k^o = \frac{a_k - a_{-k}}{2} = j\Im\{a_k\}
\]
\end{proof}

\section{习题 3.47}

考虑信号 $x(t)$:
\[
x(t) = \cos(2\pi t)
\]
因为 $x(t)$ 是周期的,基波周期为 1,因此对任意正整数 $N$,该信号也是周期的。若将 $x(t)$ 看成周期为 3 的周期信号,那么 $x(t)$ 的傅里叶级数系数是什么?

\subsection*{解答}

考虑信号 $x(t) = \cos(2\pi t)$。将其视为周期为 3 的周期信号,我们计算其傅里叶级数系数。傅里叶级数的表达式为:

\begin{equation}
x(t) = \sum_{k=-\infty}^{\infty} c_k e^{j \frac{2\pi k t}{T}}
\end{equation}

其中 $T = 3$ 为信号的周期,$c_k$ 是傅里叶级数系数,计算公式为:

\begin{equation}
c_k = \frac{1}{T} \int_0^T x(t) e^{-j \frac{2\pi k t}{T}} \, dt
\end{equation}

对于周期为 3 的情况,即 $T = 3$,我们代入 $x(t) = \cos(2\pi t)$:

\begin{equation}
c_k = \frac{1}{3} \int_0^3 \cos(2\pi t) e^{-j \frac{2\pi k t}{3}} \, dt
\end{equation}

使用欧拉公式 $\cos(2\pi t) = \frac{e^{j 2\pi t} + e^{-j 2\pi t}}{2}$,我们可以将傅里叶系数展开为两个积分:

\begin{equation}
c_k = \frac{1}{6} \left( \int_0^3 e^{j 2\pi t} e^{-j \frac{2\pi k t}{3}} \, dt + \int_0^3 e^{-j 2\pi t} e^{-j \frac{2\pi k t}{3}} \, dt \right)
\end{equation}

计算第一个积分

第一个积分为:

\begin{equation}
\int_0^3 e^{j 2\pi t (1 - \frac{k}{3})} \, dt = \frac{e^{j 2\pi (1 - \frac{k}{3}) 3} - 1}{j 2\pi (1 - \frac{k}{3})}
\end{equation}

注意到 $e^{j 2\pi n} = 1$ 对于任意整数 $n$,因此该积分为:

\begin{equation}
\frac{1 - 1}{j 2\pi (1 - \frac{k}{3})} = 0 \quad \text{当} \quad k \neq 3
\end{equation}

如果 $k = 3$,则该积分为:

\begin{equation}
\int_0^3 e^{j 2\pi t (1 - 1)} \, dt = \int_0^3 1 \, dt = 3
\end{equation}

计算第二个积分

第二个积分为:

\begin{equation}
\int_0^3 e^{-j 2\pi t (1 + \frac{k}{3})} \, dt = \frac{e^{-j 2\pi (1 + \frac{k}{3}) 3} - 1}{-j 2\pi (1 + \frac{k}{3})}
\end{equation}

同样,由于 $e^{-j 6\pi (1 + \frac{k}{3})} = 1$,因此该积分为 0,除非 $k = -3$。当 $k = -3$ 时,积分为:

\begin{equation}
\int_0^3 e^{-j 2\pi t (1 - 1)} \, dt = \int_0^3 1 \, dt = 3
\end{equation}

总结傅里叶级数系数

结合以上结果:

\begin{itemize}
    \item 当 $k = 3$ 时,$c_3 = \frac{1}{6} \cdot 3 = \frac{1}{2}$;
    \item 当 $k = -3$ 时,$c_{-3} = \frac{1}{6} \cdot 3 = \frac{1}{2}$;
    \item 对于其他 $k$,$c_k = 0$。
\end{itemize}

因此,傅里叶级数系数为:

\begin{equation}
c_k =
\begin{cases}
\frac{1}{2}, & k = 3 \text{ 或 } k = -3, \\
0, & \text{其他}.
\end{cases}
\end{equation}

这表明,信号 $x(t) = \cos(2\pi t)$ 的傅里叶级数系数只有在 $k = 3$ 和 $k = -3$ 时为非零,且值为 $\frac{1}{2}$。












%\nocite{*}
%\bibliography{re}
%\thispagestyle{fancy} 
%\addcontentsline{toc}{chapter}{参考文献}


% ------------------------------------------------------------ %
% >> ------------------------ 附录 ------------------------ << %

%\newpage
%\appendix
% chapter 标题自定义设置
%\titleformat{\chapter}[hang]{\normalfont\huge\bfseries\centering}{}{20pt}{}
%\titlespacing*{\chapter}{0pt}{-25pt}{8pt} % 控制上方空白的大小
% section 标题自定义设置 
%\titleformat{\section}[hang]{\normalfont\centering\Large\bfseries}{\thesection}{8pt}{}

% 附录 A
%\chapter*{附录 A. Matlab 代码}\addcontentsline{toc}{chapter}{附录 A. Matlab 代码}   
%\thispagestyle{fancy} 
%\setcounter{section}{0}   
%\renewcommand\thesection{A.\arabic{section}}   
%\renewcommand{\thefigure}{A.\arabic{figure}} 
%\renewcommand{\thetable}{A.\arabic{table}}


% >> ------------------------ 附录 ------------------------ << %
% ------------------------------------------------------------ %

\end{document}

% VScode 常用快捷键:

% Ctrl + R:                 打开最近的文件夹
% F2:                       变量重命名
% Ctrl + Enter:             行中换行
% Alt + up/down:            上下移行
% 鼠标中键 + 移动:           快速多光标
% Shift + Alt + up/down:    上下复制
% Ctrl + left/right:        左右跳单词
% Ctrl + Backspace/Delete:  左右删单词    
% Shift + Delete:           删除此行
% Ctrl + J:                 打开 VScode 下栏(输出栏)
% Ctrl + B:                 打开 VScode 左栏(目录栏)
% Ctrl + `:                 打开 VScode 终端栏
% Ctrl + 0:                 定位文件
% Ctrl + Tab:               切换已打开的文件(切标签)
% Ctrl + Shift + P:         打开全局命令(设置)

% Latex 常用快捷键

% Ctrl + Alt + J:           由代码定位到PDF
% 


% Git提交规范:
% update: Linear Algebra 2 notes
% add: Linear Algebra 2 notes
% import: Linear Algebra 2 notes
% delete: Linear Algebra 2 notes
