% 若编译失败,且生成 .synctex(busy) 辅助文件,可能有两个原因:
% 1. 需要插入的图片不存在:Ctrl + F 搜索 'figure' 将这些代码注释/删除掉即可
% 2. 路径/文件名含中文或空格:更改路径/文件名即可

% ------------------------------------------------------------- %
% >> ------------------ 文章宏包及相关设置 ------------------ << %
% 设定文章类型与编码格式
\documentclass[UTF8]{report}		

% 本文特殊宏包
    \usepackage{siunitx} % 埃米单位

% 本文的特殊宏定义
\def\Im{\mathrm{\,Im\,}}
\def\Re{\mathrm{\,Re\,}}
\def\Ln{\mathrm{\,Ln\,}}
\def\Arg{\mathrm{\,Arg\,}}
\def\Arccos{\mathrm{\,Arccos\,}}
\def\Arcsin{\mathrm{\,Arcsin\,}}
\def\Arctan{\mathrm{\,Arctan\,}}

% 通用宏定义
\def\N{\mathbb{N}}
\def\F{\mathbb{F}}
\def\Z{\mathbb{Z}}
\def\Q{\mathbb{Q}}
\def\R{\mathbb{R}}
\def\C{\mathbb{C}}
\def\T{\mathbb{T}}
\def\S{\mathbb{S}}
\def\A{\mathbb{A}}
\def\I{\mathscr{I}}
\def\d{\mathrm{d}}
\def\p{\partial}


% 导入基本宏包
    \usepackage[UTF8]{ctex}     % 设置文档为中文语言
    \usepackage[colorlinks, linkcolor=blue, anchorcolor=blue, citecolor=blue, urlcolor=blue]{hyperref}  % 宏包:自动生成超链接 (此宏包与标题中的数学环境冲突)
    % \usepackage{docmute}    % 宏包:子文件导入时自动去除导言区,用于主/子文件的写作方式,\include{./51单片机笔记}即可。注:启用此宏包会导致.tex文件capacity受限。
    \usepackage{amsmath}    % 宏包:数学公式
    \usepackage{mathrsfs}   % 宏包:提供更多数学符号
    \usepackage{amssymb}    % 宏包:提供更多数学符号
    \usepackage{pifont}     % 宏包:提供了特殊符号和字体
    \usepackage{extarrows}  % 宏包:更多箭头符号
    \usepackage{multicol}   % 宏包:支持多栏 
    \usepackage{graphicx}   % 宏包:插入图片
    \usepackage{float}      % 宏包:设置图片浮动位置
    \usepackage{mathtools}  % 宏包:数学公式
    %\usepackage{article}    % 宏包:使文本排版更加优美
    \usepackage{tikz}       % 宏包:绘图工具
    \usepackage{pgffor}     % 宏包:提供 \foreach 命令
    \usepackage{pgfplots}   % 宏包:绘图工具
    \usepackage{pgffor}     % 宏包:提供 \foreach 命令

% 文章页面margin设置
    \usepackage[a4paper]{geometry}
        \geometry{top=1in}  % 1 inch= 2.46 cm, 0.75 inch = 1.85 cm
        \geometry{bottom=1in}
        \geometry{left=0.75in}
        \geometry{right=0.75in}   % 设置上下左右页边距
        \geometry{marginparwidth=1.75cm}    % 设置边注距离(注释、标记等)

% 配置数学环境
    \usepackage{amsthm} % 宏包:数学环境配置
    % theorem-line 环境自定义
        \newtheoremstyle{MyLineTheoremStyle}% <name>
            {11pt}% <space above>
            {11pt}% <space below>
            {}% <body font> 使用默认正文字体
            {}% <indent amount>
            {\bfseries}% <theorem head font> 设置标题项为加粗
            {:}% <punctuation after theorem head>
            {.5em}% <space after theorem head>
            {\textbf{#1}\thmnumber{#2}\ \ (\,\textbf{#3}\,)}% 设置标题内容顺序
        \theoremstyle{MyLineTheoremStyle} % 应用自定义的定理样式
        \newtheorem{LineTheorem}{Theorem.\,}
    % theorem-block 环境自定义
        \newtheoremstyle{MyBlockTheoremStyle}% <name>
            {11pt}% <space above>
            {11pt}% <space below>
            {}% <body font> 使用默认正文字体
            {}% <indent amount>
            {\bfseries}% <theorem head font> 设置标题项为加粗
            {:\\ \indent}% <punctuation after theorem head>
            {.5em}% <space after theorem head>
            {\textbf{#1}\thmnumber{#2}\ \ (\,\textbf{#3}\,)}% 设置标题内容顺序
        \theoremstyle{MyBlockTheoremStyle} % 应用自定义的定理样式
        \newtheorem{BlockTheorem}[LineTheorem]{Theorem.\,} % 使用 LineTheorem 的计数器
    % definition 环境自定义
        \newtheoremstyle{MySubsubsectionStyle}% <name>
            {11pt}% <space above>
            {11pt}% <space below>
            {}% <body font> 使用默认正文字体
            {}% <indent amount>
            {\bfseries}% <theorem head font> 设置标题项为加粗
            {\\ \indent}% <punctuation after theorem head>
            {0pt}% <space after theorem head>
            {\textbf{#3}}% 设置标题内容顺序
        \theoremstyle{MySubsubsectionStyle} % 应用自定义的定理样式
        \newtheorem{definition}{}

%宏包:有色文本框(proof环境)及其设置
    \usepackage[dvipsnames,svgnames]{xcolor}    %设置插入的文本框颜色
    \usepackage[strict]{changepage}     % 提供一个 adjustwidth 环境
    \usepackage{framed}     % 实现方框效果
        \definecolor{graybox_color}{rgb}{0.95,0.95,0.96} % 文本框颜色。修改此行中的 rgb 数值即可改变方框纹颜色,具体颜色的rgb数值可以在网站https://colordrop.io/ 中获得。(截止目前的尝试还没有成功过,感觉单位不一样)(找到喜欢的颜色,点击下方的小眼睛,找到rgb值,复制修改即可)
        \newenvironment{graybox}{%
        \def\FrameCommand{%
        \hspace{1pt}%
        {\color{gray}\small \vrule width 2pt}%
        {\color{graybox_color}\vrule width 4pt}%
        \colorbox{graybox_color}%
        }%
        \MakeFramed{\advance\hsize-\width\FrameRestore}%
        \noindent\hspace{-4.55pt}% disable indenting first paragraph
        \begin{adjustwidth}{}{7pt}%
        \vspace{2pt}\vspace{2pt}%
        }
        {%
        \vspace{2pt}\end{adjustwidth}\endMakeFramed%
        }

% 外源代码插入设置
    % matlab 代码插入设置
    %\usepackage{matlab-prettifier}
    %    \lstset{
    %        style=Matlab-editor,  % 继承matlab代码颜色等
    %    }
    %\usepackage[most]{tcolorbox} % 引入tcolorbox包 
    %\usepackage{listings} % 引入listings包
    %    \tcbuselibrary{listings, skins, breakable}
    %    \newfontfamily\codefont{Consolas} % 定义需要的 codefont 字体
    %    \lstdefinestyle{matlabstyle}{
    %        language=Matlab,
    %        basicstyle=\small\ttfamily\codefont,    % ttfamily 确保等宽 
    %        breakatwhitespace=false,
    %        breaklines=true,
    %        captionpos=b,
    %        keepspaces=true,
    %        numbers=left,
    %        numbersep=15pt,
    %        showspaces=false,
    %        showstringspaces=false,
    %        showtabs=false,
    %        tabsize=2
    %    }
    %    \newtcblisting{matlablisting}{
    %        arc=2pt,        % 圆角半径
    %        top=-5pt,
    %        bottom=-5pt,
    %        left=1mm,
    %        listing only,
    %        listing style=matlabstyle,
    %        breakable,
    %        colback=white   % 选一个合适的颜色
    %    }
% table 支持
    \usepackage{booktabs}   % 宏包:三线表
    \usepackage{tabularray} % 宏包:表格排版
    \usepackage{longtable}  % 宏包:长表格


%figure 设置
%    \usepackage{graphicx}  % 支持 jpg, png, eps, pdf 图片 
%    \usepackage{svg}       % 支持 svg 图片
%        \svgsetup{
%             指向 inkscape.exe 的路径
%            inkscapeexe = C:/aa_MySame/inkscape/bin/inkscape.exe, 
%            inkscapeexe = C:/aa_MySame/inkscape/bin/inkscape.exe, 
%             一定程度上修复导入后图片文字溢出几何图形的问题
%            inkscapelatex = false                 
%        }
%    \usepackage{subcaption} % subfigure 子图支持

%图表进阶设置
%    \usepackage{caption}    % 图注、表注
%        \captionsetup[figure]{name=图}  
%        \captionsetup[table]{name=表}
%        \captionsetup{labelfont=bf, font=small}
%    \usepackage{float}     % 图表位置浮动设置 

% 圆圈序号自定义
    \newcommand*\circled[1]{\tikz[baseline=(char.base)]{\node[shape=circle,draw,inner sep=0.8pt, line width = 0.03em] (char) {\small \bfseries #1};}}   % TikZ solution

% 列表设置
%    \usepackage{enumitem}   % 宏包:列表环境设置
%        \setlist[enumerate]{itemsep=0pt, parsep=0pt, topsep=0pt, partopsep=0pt, leftmargin=3.5em} 
%        \setlist[itemize]{itemsep=0pt, parsep=0pt, topsep=0pt, partopsep=0pt, leftmargin=3.5em}
%        \newlist{circledenum}{enumerate}{1} % 创建一个新的枚举环境  
%        \setlist[circledenum,1]{  
%            label=\protect\circled{\arabic*}, % 使用 \arabic* 来获取当前枚举计数器的值,并用 \circled 包装它  
%            ref=\arabic*, % 如果需要引用列表项,这将决定引用格式(这里仍然使用数字)
%            itemsep=0pt, parsep=0pt, topsep=0pt, partopsep=0pt, leftmargin=3.5em
%        }  

% 其它设置
    % 脚注设置
        \renewcommand\thefootnote{\ding{\numexpr171+\value{footnote}}}
    % 参考文献引用设置
        \bibliographystyle{unsrt}   % 设置参考文献引用格式为unsrt
        \newcommand{\upcite}[1]{\textsuperscript{\cite{#1}}}     % 自定义上角标式引用
    % 文章序言设置
        \newcommand{\cnabstractname}{序言}
        \newenvironment{cnabstract}{%
            \par\Large
            \noindent\mbox{}\hfill{\bfseries \cnabstractname}\hfill\mbox{}\par
            \vskip 2.5ex
            }{\par\vskip 2.5ex}

% 文章默认字体设置
    \usepackage{fontspec}   % 宏包:字体设置
        \setmainfont{SimSun}    % 设置中文字体为宋体字体
        \setCJKmainfont[AutoFakeBold=3]{SimSun} % 设置加粗字体为 SimSun 族,AutoFakeBold 可以调整字体粗细
        \setmainfont{Times New Roman} % 设置英文字体为Times New Roman

% 各级标题自定义设置
    \usepackage{titlesec}   
        \titleformat{\chapter}[hang]{\normalfont\huge\bfseries\centering}{第\,\thechapter\,章}{20pt}{}
        \titlespacing*{\chapter}{0pt}{-20pt}{20pt} % 控制上方空白的大小
        % section标题自定义设置 
        \titleformat{\section}[hang]{\normalfont\Large\bfseries}{§\,\thesection\,}{8pt}{}
        % subsubsection标题自定义设置
        %\titleformat{\subsubsection}[hang]{\normalfont\bfseries}{}{8pt}{}

% >> ------------------ 文章宏包及相关设置 ------------------ << %
% ------------------------------------------------------------- %

% ----------------------------------------------------------- %
% >> --------------------- 文章信息区 --------------------- << %
% 页眉页脚设置
    \usepackage{fancyhdr}   %宏包:页眉页脚设置
        \pagestyle{fancy}
        \fancyhf{}
        \cfoot{\thepage}
        \renewcommand\headrulewidth{1pt}
        \renewcommand\footrulewidth{0pt}
        \lhead{2024.8 -- 2025.1} 
        \chead{yinchao050313@gmail.com}    
        \rhead{yinchao23@mails.ucas.ac.cn}
%文档信息设置
    \title{原子物理\\Atomic Physics}
    \author{尹超\\ \footnotesize 中国科学院大学,北京 100049\\ Carter Yin \\ \footnotesize University of Chinese Academy of Sciences, Beijing 100049, China}
    \date{\footnotesize 2024.8 -- 2025.1}
% >> --------------------- 文章信息区 --------------------- << %
% ----------------------------------------------------------- %

% 开始编辑文章

\begin{document} 
\zihao{5}             % 设置全文字号大小, -4 为小四, 5 为五号

% --------------------------------------------------------------- %
% >> --------------------- 封面序言与目录 --------------------- << %
% 封面
    \maketitle\newpage  
    \pagenumbering{Roman} % 页码为大写罗马数字
    \thispagestyle{fancy}   % 显示页码、页眉等

% 序言
    \begin{cnabstract}\normalsize 
        本笔记为原子物理的笔记整理\par
        讲课教师:\par
        • 李海波\par 
        • 办公室: ⾼能物理所 多学科⼤楼 619\par
        • telephone: 88233203\par
        • email: lihb@ihep.ac.cn\par
        助教:\par
        • 胡海明 研究员 中科院高能物理所 \par
        • 办公室: ⾼能物理所 多学科⼤楼 519\par
        • telephone: 18811590054\par
        • email:huhm@ihep.ac.cn
\end{cnabstract}    
\addcontentsline{toc}{chapter}{序言} % 手动添加为目录

% 目录
    \setcounter{tocdepth}{4}                % 目录深度(为1时显示到section)
    \tableofcontents                        % 目录页
    \addcontentsline{toc}{chapter}{目录}    % 手动添加此页为目录
    \thispagestyle{fancy}                   % 显示页码、页眉等 

% 收尾工作
    \newpage    
    \pagenumbering{arabic} 


% >> --------------------- 封面序言与目录 --------------------- << %
% --------------------------------------------------------------- %
\chapter{量子力学初步}\thispagestyle{fancy} 

\section{物质的波粒⼆象性}

\subsection{光电效应}

\begin{definition}
    光电效应:当光照射到金属表面时,金属表面会发射出电子,这种现象称为光电效应。
    截止频率:当光的频率小于截止频率时,光电效应不会发生。
    光电效应的截止频率与金属的逸出功有关,与光的强度无关。
    光量子假设:光是由一束一束的光子组成的,每个光子的能量为$E=h\nu$,其中$h$为普朗克常数,$\nu$为光的频率。
    爱因斯坦方程:$E=h\nu=\frac{1}{2}m_ev^2+W$,其中$W$为逸出功。
    对同⼀种⾦属,W⼀定,动能EK ∝ν,与光强⽆关
\end{definition}

\subsection{普朗克常数的测定}

\begin{definition}
    普朗克常数的测定:通过测定光电效应的截止频率来测定普朗克常数。
  $hv = W + \frac{1}{2}m_ev^2$,其中$W$为逸出功,$m_e$为电子质量,$v$为电子的速度。
  $hv = eU_0 + W$,其中$U_0$为阻止电压。
  $U_0 = \frac{hv}{e} - \frac{W}{e}$,其中$e$为电子电荷。
  $\frac{\delta U_0}{\delta v} = \frac{h}{e}$,通过测定阻止电压与光频率的关系,可以测定普朗克常数。
  $h = e\frac{\delta U_0}{\delta v}$
\end{definition}    


\subsection{光子的动量}

\begin{definition}
$p = \frac{E}{c} = \frac{h\nu}{c} = \frac{h}{\lambda}$
\end{definition}


\subsection{康普顿散射}

\begin{definition}
康普顿散射:X射线与物质中的电子发生碰撞,X射线的波长发生变化,这种现象称为康普顿散射。

\begin{itemize}
    \item 散射出现了 \(\lambda \neq \lambda_0\) 的现象,称为康普顿散射。
    \item 散射曲线的三个特点:
    \begin{itemize}
        \item 除原波长 \(\lambda_0\) 外,出现了移向长波方面的新的散射波长 \(\lambda\);
        \item 新波长 \(\lambda\) 随散射角 \(\phi\) 的增大而增大;
        \item 当散射角增大时,原波长的谱线强度降低,而新波长的谱线强度升高。
    \end{itemize}
    \item 新散射波长 \(\lambda > \lambda_0\),波长的偏移 \(\Delta \lambda = \lambda - \lambda_0\) 只与散射角 \(\phi\) 有关,和散射物质无关。
    \item 实验规律总结:
    \[
    \Delta \lambda = \lambda_c (1 - \cos \phi) = 2 \lambda_c \sin^2 \left( \frac{\phi}{2} \right)
    \]
    \(\lambda_c = 0.0241 \text{Å} = 2.41 \times 10^{-3} \text{nm}\)(实验值),\(\lambda_c\) 称为电子的康普顿波长。
    \item 只有当入射波长 \(\lambda_0\) 与 \(\lambda_c\) 可比拟时,康普顿效应才显著。因此要用X射线才能观察到!X射线波长范围:$1\,\text{nm} \sim 10^{-3}\,\text{nm}$。
\end{itemize}

\begin{figure}[H]
    \centering
    \includegraphics[width=0.6\textwidth]{kang1.png}
    \caption{康普顿散射示意图}
\end{figure}

碰撞过程中能量与动量守恒。碰撞⇒光子把部分能量传给电子⇒光子的能量↓⇒散射X射线频率↓ 波长↑。

康普顿散射的理论推导:
\begin{itemize}
    \item 能量守恒:\(hv_0 + m_0 c^2 = hv + m_0 c^2\)
    \item 动量守恒:\(\frac{h \nu_0}{c} \hat{n}_0 = \frac{h \nu}{c} \hat{n} + mv\)
    \item 反冲电子质量:\(m = \frac{m_0}{\sqrt{1 - \frac{v^2}{c^2}}}\)
\end{itemize}

解得:
\[
\Delta \lambda = \lambda - \lambda_0 = \frac{c}{\nu} - \frac{c}{\nu_0} = \frac{h}{m_0 c}(1 - \cos \phi) = 2 \lambda_c \sin^2 \left( \frac{\phi}{2} \right)
\]
\(\lambda_c = \frac{h}{m_0 c} = 2.43 \times 10^{-12} \text{m}\)

波长位移只与散射角有关系,与入射波长 \(\lambda_0\) 无关,但实验测量中有意义的是 \(\Delta \lambda / \lambda_0\),因为 \(\Delta \lambda\) 最大为 0.0049nm,因此只有 \(\lambda < 0.1 \text{nm}\) 的X射线才能使得 \(\Delta \lambda / \lambda_0\) 大到足以被观测的程度。这就是为什么只有用X射线,我们才开始观测到康普顿效应。

\begin{itemize}
    \item 康普顿散射中还有原波长 \(\lambda_0\) 成分?
    \item 因X射线与紧紧地束缚在靶原子中的电子发生碰撞时,实际上是与整个原子相互作用,此时,\(m_0\) 是整个原子的静止质量,不是很小的电子质量(我们称这种散射为汤姆孙散射)。
    \item 为什么康普顿效应中的电子不能像光电效应那样吸收光子而是散射光子?(光电效应和康普顿散射的区别?)
    \item 为什么在光电效应中不考虑动量守恒?(原子中电子的束缚能)
    \item 为什么可见光观察不到康普顿效应?(可见光波长 ~100nm)
\end{itemize}

试推导汤姆孙散射中波长位移为:
\[
\Delta \lambda = \lambda - \lambda_0 = \lambda_A (1 - \cos \theta)
\]
其中 \(\lambda_A = \frac{h}{m_A c}\),\(m_A\) 为整个原子的静止质量,称为原子的康普顿波长,很小:\(\delta \lambda \sim 0\)。

\begin{figure}
    \centering
    \includegraphics[width=0.6\textwidth]{kang2.png}
    \caption{汤姆孙散射示意图1}
\end{figure}

\begin{figure}
    \centering
    \includegraphics[width=0.6\textwidth]{kang3.png}
    \caption{汤姆孙散射示意图2}
\end{figure}
\end{definition}


\subsection{光的波粒二象性}
\begin{definition}
    光的波粒二象性:
    \begin{itemize}
        \item 光的波动性:干涉、衍射、偏振
        \item 光的粒子性:光电效应、康普顿散射:\(E=hv\)
        \item 相对论能量和动量关系:\(E^2 = p^2c^2 + m^2c^4\)
        \item 光子 \(E_0 = 0, E = pc\)
        \[
        p = \frac{E}{c} = \frac{h\nu}{c} = \frac{h}{\lambda}
        \]
    \end{itemize}

    光的波粒二象性 \(\Rightarrow\) 实物粒子:波粒二象性
    \[
    E = h \nu 
    \]
    \[
    p = \frac{h}{\lambda}
    \]

    德布罗意公式:
    \[
    \lambda = \frac{h}{p} = \frac{h}{mv}  
    \]
    \[
    v = \frac{E}{h} = \frac{mc^2}{h}
    \]

    宏观物体的德布罗意波长小到实验难以测量的程度,因此宏观物体仅表现出粒子性。
    \[
    E = mc^2 = h\nu
    \]
    \[
    p = mv = \frac{h}{\lambda}
    \]
    \[
    m = \frac{m_0}{\sqrt{1-\frac{v^2}{c^2}}}
    \]

    相对论动量、能量、质量和动能方程:
    \[
    p^2c^2 + m_0^2c^4 = E^2 = m^2c^4
    \]
    \[
    E_k = mc^2 - m_0c^2
    \]

    引入波矢 \(k = \frac{2\pi}{\lambda}\) \(\Longrightarrow \overrightarrow{p} = \hbar \overrightarrow{k}\)
    \begin{itemize}
        \item 波动的传播方向是粒子的动量方向。
        \item 德布罗意关系式通过 \(h\) 把粒子性和波动性联系起来。
        \item 实际上,任何表达式中,只要有 \(h\) 出现,就意味其具有量子力学特征。
        \item \(h\) 的意义:量子化的量度,是不连续程度的最小量度单位。\(h\) 在物质的波性和粒子性间起着桥梁作用;在量子化和波粒二象性这两个重要概念中都起关键作用。
    \end{itemize}
\end{definition}

\subsection{德布罗意波关系式的应用}

\begin{definition}
    若将德布罗意关系式应用于氢原子上,原子定态假设便和驻波联系起来,十分自然地给出角动量量子化条件。电子要想作稳定运动,电子回转一周的周长应为其波长整数倍,即
    \[
    2\pi r = n\lambda = n\frac{h}{p} = n\frac{h}{mv}, \quad n = 1,2,\dots
    \]
    于是有
    \[
    mvr = n\frac{h}{2\pi} = n \hbar
    \]
    这正是玻尔曾用过的角动量量子化条件。

    驻波条件:
    为克服玻尔理论人为的缺陷,德布罗意把原子定态与驻波联系起来,即把粒子能量量子化问题和有限空间中驻波的波长(或频率)的分立性联系起来。
    \[
    p = \frac{h}{\lambda} = \frac{h}{2\pi r} = \frac{nh}{2\pi r} = \frac{n\hbar}{r}
    \]
    将上式改写后即得角动量量子化条件:
    \[
    L = n\hbar, \quad n = 1,2,\dots
    \]
    只有驻波可被束缚起来,而驻波条件就是角动量量子化条件!

    例:将玻尔第一速度 \(v = \alpha c\) 代入,得到
    \[
    \lambda = 2\pi \frac{\hbar}{mc\alpha}
    \]
    而 \(\frac{\hbar}{mc\alpha}\) 是折合电子康普顿波长的137倍,即第一玻尔半径 \(a_1\),故
    \[
    \lambda = 2\pi a_1
    \]
    所得结果满足驻波条件。
\end{definition}

\subsection{物质波的实验验证}

\begin{definition}
    \foreach \i in {1,...,26} {
    \begin{figure}[H]
        \centering
        \includegraphics[width=1\textwidth]{lzh\i.png}
    \end{figure}
}
\end{definition}




\section{海森堡不确定关系}

\subsection{海森堡不确定原理}

\begin{definition}
    位置—动量不确定关系:
    \begin{itemize}
        \item 在经典力学中,运动物体在任何时刻都有完全确定的位置、动量、能量和角动量等物理量。
        \item 微观粒子具有明显的波性,波不能同时有完全确定的位置、动量等物理量。
        \item 海森堡(W. Heisenberg)在1927年发表了著名的位置—动量不确定关系:
        \[
        \Delta x \Delta p \geq \frac{\hbar}{2}
        \]
        其中 \(\Delta x\) 是位置的不确定度,\(\Delta p\) 是动量的不确定度,\(\hbar = \frac{h}{2\pi}\) 是约化普朗克常数。
    \end{itemize}
\end{definition}

\begin{definition}

\begin{figure}[H]
    \centering
    \includegraphics[width=0.6\textwidth]{hsb1.png}
\end{figure}
    
\begin{figure}[H]
    \centering
    \includegraphics[width=0.6\textwidth]{hsb2.png}
\end{figure}

\end{definition}   


\begin{definition}
    电子衍射的不确定关系:
    \begin{itemize}
        \item 狭缝对电子束起了两种作用:一是将它的坐标限制在缝宽 \(d\) 的范围内,一是使电子在坐标方向上的动量发生了变化。这两种作用是相伴出现的,不可能既限制了电子的坐标,又能避免动量发生变化。
        \item 如果缝愈窄,即坐标愈确定,则在坐标方向上的动量就愈不确定。因此,微观粒子的坐标和动量不能同时有确定的值。
    \end{itemize}
    
    例题:
    \begin{itemize}
        \item 氢原子中电子的玻尔第一半径:$r_1 = 0.053 \text{nm}$
        \item 玻尔第一速度: $v_1 = \alpha c$
        \item 玻尔第一动量:$p_1 = mv_1 = m\alpha c$
        \item 假定电子在第一轨道 \(r_1\) 范围内运动:$\Delta x = r_1 = 0.053 \text{nm}$ \quad $\frac{\Delta p}{p} = \frac{\frac{h}{\Delta x}}{p} = \frac{hc}{mc^2\alpha \Delta x} = \frac{1.24 \text{nm} \cdot \text{keV}}{511 \text{keV} \cdot (137)^{-1} \cdot 0.053 \text{nm}} = 6.3$    动量不确定程度很大!
        \item 假定电子在第一轨道上,位置确定了,它的动量就完全不确定?因此“在轨道上运动”的概念就失去意义?
    \end{itemize}

    \vspace{1cm}
    
    讨论:用不确定关系讨论原子中电子的速度
    \begin{itemize}
        \item 原子的线度的数量级是 \(10^{-10} \text{m}\),原子中确定电子位置的不准确量为 \(\Delta x \approx 10^{-10} \text{m}\)。
        \item 动量的不准确量为$\Delta p \approx \frac{h}{\Delta x}$
        \item 原子中电子速度的不确定量按不确定关系:$\Delta x \Delta m v \geq \frac{\hbar}{2}$
        \item 按经典力学算氢原子的电子在轨道上速度的数量级为 $10^6 \text{m/s}$.
    \end{itemize}
\end{definition}


\begin{definition}
    \foreach \i in {1,...,11} {
    \begin{figure}[H]
        \centering
        \includegraphics[width=1\textwidth]{hsb\i.png}
    \end{figure}
}
\end{definition}


\begin{definition}
海森堡不确定原理: $\Delta x \Delta v \geq \frac{\hbar}{2m}$
不确定性原理:粒子可能性疆域
\end{definition}

\section{波函数及其统计解释}

\subsection{物质波波函数}

\begin{definition}
    如果粒子处于随时间和位置变化的力场中运动,它的动量和能量不再是常量(或不同时为常量),粒子的状态就不能用平面波描写,而必须用较复杂的波描写,一般记为:$\varPsi(\overrightarrow{r},t)$。

    描述粒子状态的波函数,通常是一个复函数。

    问题:
    \begin{enumerate}
        \item \(\psi\) 怎样描述粒子的状态?
        \item \(\psi\) 如何体现波粒二象性?
        \item \(\psi\) 描写的是什么样的波?
    \end{enumerate}

    量子力学波函数(复函数):$\Psi (x,t) = \psi_0 e^{-i\left(\omega t - kx\right)}$,其中 \(k = \frac{2\pi}{\lambda}\), \(\omega = \frac{2\pi}{T}\), \(E = \hbar \omega\), \(p = \hbar k\),波矢 \(k\) 和角频率 \(\omega\) 与粒子的动量 \(p\) 和能量 \(E\) 有关。

    可以将波函数改写为:$\Psi (x,t) = \psi_0 e^{-\frac{i}{\hbar}\left(Et - px\right)}$,其中 $\psi_0$ 为待定常数。

    若粒子为三维运动,波函数可以表示为:$\Psi (\overrightarrow{r},t) = \psi_0 e^{-\frac{i}{\hbar}\left(Et - \overrightarrow{p} \cdot \overrightarrow{r}\right)}$。

    电子是波粒二象性的统一体。电子具有质量、电荷等粒子的属性,但没有确定的运动轨道;电子具有干涉、衍射等波动现象,但不是实在物理量的周期性的变化。

    $\Psi (r,t)$ 的物理意义在于:波函数模的平方(波的强度)代表时刻 \(t\)、在空间 \(r\) 点处,单位体积元中微观粒子出现的概率。

    关于波函数的说明:
    \begin{itemize}
        \item 波函数的模的平方是概率,波函数是复数,又叫概率幅:$\rho (\overrightarrow{r},t) = |\Psi (\overrightarrow{r},t)|^2 = \Psi (\overrightarrow{r},t)^* \Psi (\overrightarrow{r},t)$
        \item 对 \(N\) 个粒子,\(N|\Psi|^2\) 给出粒子数的分布密度。
        \item 对单个粒子,\(|\Psi|^2\) 给出粒子的概率分布密度。
        \item 在时刻 \(t\)、空间 \(r\) 点处,体积单元 \(dV\) 中发现微观粒子的概率为 \(\rho (\overrightarrow{r},t)dV = \Psi (\overrightarrow{r},t)^* \Psi (\overrightarrow{r},t)dV = |\Psi|^2 dV\)。
        \item 对 \(N\) 粒子系,在体积单元 \(dV\) 中发现的粒子数为 \(dN = N\Psi (\overrightarrow{r},t)^* \Psi (\overrightarrow{r},t)dV = N|\Psi|^2 dV\)。
        \item 粒子在空间各点的概率总和应为1,即 \(\int_{\Omega} |\Psi|^2 dV = \int_{\Omega}\Psi (\overrightarrow{r},t)^* \Psi (\overrightarrow{r},t)dV = 1\)。
    \end{itemize}

    波函数应满足的条件(自然条件):
    \begin{itemize}
        \item 连续性:因概率不会在某处突变,故波函数必须处处连续。
        \item 单值性:任一点或任意体积元内只能有一个概率,故波函数一定是单值的。
        \item 有限性:因概率不可能为无限大,故波函数必须是有限的。
    \end{itemize}

    令两者所描写状态的相对几率是相同的,这里的 \(C\) 是常数。因为在 \(t\) 时刻,空间任意两点 \(r_1\) 和 \(r_2\) 处找到粒子的相对几率之比是:
    \[
    \frac{\varphi(r_1,t)}{\varphi(r_2,t)} = \frac{\Psi(r_1,t)}{\Psi(r_2,t)}
    \]
    可见,\(\varphi(r,t)\) 和 \(\Psi(r,t)\) 描述的是同一几率波。

    粒子在空间各点出现的几率只取决于波函数在空间各点强度的相对比例,而不取决于强度的绝对大小,因此,将波函数乘上一个常数后,所描写的粒子状态不变,即 \(\Psi (r, t)\) 和 \(C\Psi (r, t)\) 描述同一状态。

    与经典波不同:经典波波幅增大一倍(原来的2倍),则相应的波动能量将为原来的4倍,因此代表完全不同的波动状态。经典波无归一化问题。

    量子力学第一个基本假设—波函数假设:
    \begin{itemize}
        \item 内容:微观体系的状态可被一个波函数完全描述,从这个波函数出发可以得出体系的所有性质。波函数一般应满足连续性、有限性和单值性三个条件。
        \item 理解:这一原理给出了量子力学中反映微观体系运动状态的基本物理量——波函数,而由波函数描述运动状态的方式(几率),反映出了微观体系的运动方式(几率运动)与宏观客体有很大不同,从而决定了量子力学的理论体系与经典理论也将大不相同。另外,也需注意,这一原理是由实践总结出来的基本假设,最终也只能由实践来检验。
    \end{itemize}

    另一个问题是,既然微观粒子服从统计规律,为什么不引入几率直接进行描述,却要借用波函数——几率幅来描述呢?按照几率论,一个事件假若有两种可能发生,其几率分别是 \(P_1\) 和 \(P_2\),那么该事件出现的几率是 \(P = P_1 + P_2\)。显然这种几率相加不会出现干涉效应,不显示微观粒子的波动性,完全是经典的描述图象。代之,若一个事件有两种可能发生的几率幅 \(\Psi_1\) 和 \(\Psi_2\),该事件发生的几率幅是 \(\Psi_1\) 和 \(\Psi_2\) 之叠加,即 \(\Psi = \Psi_1 + \Psi_2\),那么相应的几率是
    \[
    P = |\Psi|^2 = |\Psi_1 + \Psi_2|^2 = |\Psi_1|^2 + |\Psi_2|^2 + \Psi_1^* \Psi_2 + \Psi_1 \Psi_2^*
    \]
    上式中的后两项代表相干项,显示出波动性。所以微观世界的统计规律是几率幅相加律(不是经典几率直接相加)。物理学大师费曼把几率幅叠加称为“量子力学的第一原理”。他这样写到:“如果一个事件可能有几种方式实现,则该事件的几率幅就是各种单独实现的几率幅之和,于是出现了干涉”。显示了波动性。

    态的叠加原理:
    \begin{itemize}
        \item 与光学中波的叠加原理一样,量子力学中也存在波叠加原理。因为量子力学中的波,即波函数决定体系的状态,称波函数为状态波函数,所以量子力学的波叠加原理称为态叠加原理。
        \item 量子力学中的态(\(\psi\))叠加不是经典波的叠加,而是概率幅的叠加。
        \item 经典物理中,波的叠加只不过是将波幅叠加(波幅代表实际物体的运动等),并在合成波中出现不同频率的波长的子波成分。
        \item 微观粒子的态叠加实质是什么呢?
    \end{itemize}

    态叠加原理的表述:
    \begin{itemize}
        \item 微观世界:事件发生概率 \(P = \Psi^* \Psi = |\Psi|^2\)
        \item 狄拉克(Dirac)符号:Bra: \(\langle A|\),Kit: \(|B\rangle\) 表达量子态;令 \(\langle A| = \Psi^*\),\(|B\rangle = \Psi\)
        \item 发生事件:从初态 \(i\) 到末态 \(f\) 的跃迁概率 \(w_{i \to f}\) 的 Dirac 符号表示形式:\(w_{i \to f} = |\langle f|i \rangle|^2\)
        \item 态叠加(概率幅)需要服从一定的规则。
    \end{itemize}
\end{definition}

\begin{definition}
    \foreach \i in {1,...,9} {
    \begin{figure}[H]
        \centering
        \includegraphics[width=1\textwidth]{wuli\i.png}
    \end{figure}
}
\end{definition}


\section{薛定谔方程}

\subsection{薛定谔方程的提出}

\begin{definition}
    按照经典波动理论,波动的物理量满足如下形式的波动方程:
    \[
    \frac{\partial^2 \Psi}{\partial x^2} = \frac{1}{v^2} \frac{\partial^2 \Psi}{\partial t^2}
    \]
\end{definition}

\begin{definition}
    \foreach \i in {1,...,54} {
    \begin{figure}[H]
        \centering
        \includegraphics[width=1\textwidth]{xde\i.png}
    \end{figure}
}
\end{definition}


\end{document}